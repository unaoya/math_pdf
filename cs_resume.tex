\documentclass[uplatex]{jsarticle}
\RequirePackage{amsmath,amssymb,amsthm, amscd, comment, multicol}
\usepackage[all]{xy}
\usepackage[dvipdfmx]{graphicx}
\usepackage{tikz-cd}
\input{../tex/theorems}
\input{../tex/symbols}

\title{Chowla-Selbergの公式}
\author{梅崎直也@unaoya}
\date{\today}
\begin{document}

\maketitle
\begin{thm}[Chowla-Selbergの公式]
\begin{align*}
\prod_{a\in Cl(k)}\Delta(a)\Delta(a^{-1})=\left(\frac{2\pi}{d}\right)^{12h}\prod_{a\in(\Z/d\Z)^\times}\Gamma\left(\frac{a}{d}\right)^{6w\epsilon(a)}
\end{align*}
\end{thm}

\begin{itemize}
\item $k=\Q(\sqrt{-d})$は虚二次体で$o_k$をその整数環
\item $Cl(k)$はイデアル類群で$h=\abs{Cl(k)}, w=\abs{o_k^\times}$
\item $\epsilon$は二次体$k$に対応するDirichlet指標(平方剰余記号)
\item $\Delta$はweight 12のcusp form
\end{itemize}

\begin{thm}[Gross]
\begin{align*}
\sqrt{\pi}\prod_{a\in(\Z/d\Z)^\times}\Gamma\left(\frac{a}{d}\right)^{w\epsilon(a)/4h}
\end{align*}
と$k$のorderで虚数乗法を持つ楕円曲線$E$の正則$1$形式の周期は$\mod $代数的数倍で一致
\end{thm}

\begin{thm}
微分形式を積分することでコホモロジーの同型が得られる。
\begin{align*}
H^i_{dR}(X/k) \otimes \C &\to H^i(X(\C), \C)\\
\omega &\mapsto (\gamma\mapsto\int_\gamma\omega)
\end{align*}

相対的な比較定理
\begin{align*}
\mathcal{H}^n_{dR}(A/S) \to R^n\pi_*\C \otimes_{\C}O_S
\end{align*}
\end{thm}

\begin{thm}
以下は同値
\begin{enumerate}
\item $\omega_s$は$\pi_1(S,s)$不変な$H^n(A_s,\C)$の元
\item $\omega$は$R^n\pi_*\C$の大域切断
\item 正則ベクトル束$\omega\in\mathcal{H}^n_{dR}(A(\C)/S(\C))$の$s\in S$に対して$\omega_s$の周期格子が一定
\item 代数的ベクトル束$\omega\in\mathcal{H}^n_{dR}(A/S)$の$s\in S$に対して$\omega_s$の周期格子が一定
\end{enumerate}
\end{thm}

モジュラー曲線の構成。上半平面$h$を考え、
$\pi:h\times \C/L \to h$を$L=\{(\tau, x), x \in \Z \oplus \tau\Z \subset \C\}$で定める。
さらに$SL_2(\Z)$(もしくはより一般のレベル)でわる。
これはファイバーにも作用$r_{T^{-1}}:(Y/L)_{T(n)} \to (Y/L)_n$をもつ。

虚二次体に対応する点を$h$の部分集合と思い、$SL_2(\Z)$が作用。
これに楕円曲線を引き戻して割る。(離散集合になる)
\end{document}