\documentclass[uplatex]{jsarticle}
\RequirePackage{amsmath,amssymb,amsthm, amscd, comment, multicol}
\usepackage[all]{xy}
\input{../tex/theorems}
\input{../tex/symbols}
\usepackage[dvipdfmx]{graphicx}
\usepackage{tikz, tikz-cd, tkz-euclide}
\usetkzobj{all}
\usetikzlibrary{intersections, calc}
\title{導来代数幾何入門}
\author{梅崎直也@unaoya}
\date{\today}
\begin{document}
\maketitle
ToenとVezzosiによる導来代数幾何とは通常の代数幾何よりも構造をもたせた空間を扱います。通常の代数幾何では主に可換環の圏から集合の圏への関手を扱うのに対して、導来代数幾何では例えば可換なdg代数の圏から単体的集合の圏への関手を扱います。

これにより、通常の代数幾何では扱うのが困難な問題についても対処可能になり、例えば交点理論、変形理論、モジュライ理論、表現論などへの応用があります。

私自身もまだ勉強を始めたばかりなのであまり詳しいことはお話できませんが、導来代数幾何ではどのような定式化がなされるのかについて単体的集合やdg代数、モデル圏などの予備知識からはじめて、通常の代数幾何との違いやそれが応用上どのように有効なのかについてをできる限りお話しようと思います。
\end{document}