\documentclass[uplatex]{jsarticle}
\RequirePackage{amsmath,amssymb,amsthm, amscd, comment, multicol}
\usepackage[all]{xy}
\input{../tex/theorems}
\input{../tex/symbols}
\usepackage[dvipdfmx]{graphicx}
\usepackage{tikz, tikz-cd, tkz-euclide}
\usetkzobj{all}
\usetikzlibrary{intersections, calc}
\title{Weil予想とエタールコホモロジー}
\author{梅崎直也@unaoya}
\date{\today}
\begin{document}
\maketitle

Weil予想とは有限体$\F_q$上定義された代数多様体$X$の合同ゼータ関数$Z(X/\F_q,T)$が持ついくつかの性質についての予想です。
合同ゼータ関数は、方程式の解の個数についての母関数として
\begin{align*}
  Z(X/\F_q,T)=\exp(\sum_{n=1}^\infty\abs{X(\F_{q^n})}\frac{T^n}{n})
\end{align*}
により定義されるもので、予想は
\begin{enumerate}
\item 有理性
\item 関数等式
\item Betti数との関係
\item Riemann予想の類似
\end{enumerate}
からなります。

この予想は、Weil自身による曲線の場合の解決、
Dworkによる一般次元の場合における有理性の証明ののち、
Grothendieckらによって整備されたエタールコホモロジーの理論により関数等式やBetti数との関係が証明され、
最終的にはDeligneによってRiemann予想の類似が証明されたことで完全に解決しました。

GrothendieckやDeligneによる証明においてはエタールコホモロジーや$\ell$進層の理論が大きな役割を果たします。
講演では、これらの理論がWeil予想の解決にどのように用いられたか、可能な限り証明の内容に踏み込んでお話ししたいと思います。
\end{document}
