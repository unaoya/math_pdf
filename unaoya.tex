\documentclass[uplatex]{jsarticle}
\RequirePackage{amsmath,amssymb,amsthm, amscd, comment, multicol}
\usepackage[all]{xy}
\input{../tex/theorems}
\input{../tex/symbols}
\usepackage[dvipdfmx]{graphicx}
\usepackage{tikz, tikz-cd, tkz-euclide}
\usepackage[dvipdfmx]{hyperref}
%\usetkzobj{all}
%\usetikzlibrary{intersections, calc}
\title{ヴェイユ予想と$\ell$進層のフーリエ変換}
\author{梅崎直也@unaoya}
\date{\today}
\begin{document}
\maketitle

ヴェイユ予想とは有限体$\F_q$上定義された代数多様体$X$の合同ゼータ関数$Z(X/\F_q,T)$が持ついくつかの性質についての予想です。
合同ゼータ関数は、方程式の解の個数についての母関数として
\begin{align*}
  Z(X/\F_q,T)=\exp(\sum_{n=1}^\infty\abs{X(\F_{q^n})}\frac{T^n}{n})
\end{align*}
により定義されるもので、予想は
\begin{enumerate}
\item 有理性
\item 関数等式
\item ベッチ数との関係
\item リーマン予想の類似
\end{enumerate}
からなります。
この予想は、最終的にドリーニュ[1, 2]によりリーマン予想の類似が証明されたことで完全に解決しました。

リーマン予想の類似については上のドリーニュによる証明ののち、$\ell$進層のフーリエ変換を用いた別証明がローモン[3]により与えられました。
この講演ではヴェイユ予想及び$\ell$進層のフーリエ変換について簡単に説明した上で、フーリエ変換を用いたリーマン予想の類似の証明の概略を紹介します。

\begin{thebibliography}{99}
\item Pierre Deligne, La conjecture de Weil : I, Publ. Math. IHES, 43 (1974), 273-307, \url{http://www.numdam.org/item/PMIHES_1974__43__273_0/}.
\item Pierre Deligne, La conjecture de Weil : II, Publ. Math. IHES, 52 (1980), 137-252, \url{http://www.numdam.org/item/PMIHES_1980__52__137_0/}.
\item G\'erard Laumon, Transformation de Fourier, constantes d'\'equations fonctionnelles et conjecture de Weil, Publ. Math. IHES, 65 (1987), 131-210, \url{http://www.numdam.org/item/PMIHES_1987__65__131_0/}.
\end{thebibliography}

\end{document}
