\documentclass[uplatex, a4paper]{jsbook}
\RequirePackage{amsmath,amssymb,amsthm, amscd, comment, multicol}
\usepackage[all]{xy}
\input{../tex/theorems}
\input{../tex/symbols}
\usepackage[dvipdfmx]{graphicx}
\usepackage{tikz,pgfplots}
\usepackage{tkz-euclide}
\usetkzobj{all}
\usetikzlibrary{intersections, calc}
\usepackage{epigraph}
\setlength{\epigraphwidth}{.6\textwidth}


\title{Dirichlet $L$の函数等式}
\author{梅崎 直也}
\date{\today}                                           % Activate to display a given date or no date

\begin{document}
\maketitle

\section{Riemann $\zeta$}
まず初めにRiemann $\zeta$の函数等式を雑に証明します。
証明のあらすじは
\begin{enumerate}
\item $\theta(t)$のMellin変換が$\zeta(s)\pi^{-s/2}\Gamma(\dfrac{s}{2})$
\item Poisson和公式から$\theta$の函数等式
\item $\theta$の函数等式のMellin変換が$\zeta$の函数等式
\end{enumerate}
ほんとは被積分関数の漸近挙動を調べるという重要な問題があるのですが、そこは省略します。

まずはMellin変換の定義
\begin{align*}
M(f)(s)=\int^\infty_0f(t)t^{s}\frac{dt}{t}
\end{align*}

\begin{eg}
$e^{-t}$のメリン変換は$\Gamma(s)$
$e^{-ct}$のメリン変換は$c^{-s}\Gamma(s)$

$e^{-n^2\pi t}$の$t$についてのメリン変換は$(n^2\pi)^{-s}\Gamma(s)$
\end{eg}

\begin{dfn}
テータ函数は$e^{-n^2\pi t}$の$n$についての和
\begin{align*}
\theta(t)=\sum_{n\in \Z}e^{-n^2\pi t}
\end{align*}
\end{dfn}

これのMellin変換は(積分の順序交換を無視すると)(漸近挙動を見る必要あり)
\begin{align*}
M(\theta)(s/2)=\zeta(s)\pi^{-s/2}\Gamma(\frac{s}{2})
\end{align*}

ポワソン和公式
\begin{align*}
\sum_nf(n)=\sum_n\hat{f}(n)
\end{align*}

$e^{-n^2\pi t}$の$n$についてのフーリエ変換は
\begin{align*}
e
\end{align*}
なので、
これを$f(n)=e^{-n^2\pi t}$とすると

\begin{align*}
\theta(t)=t^{-1/2}\theta(1/t)
\end{align*}

が成り立つ。

この左辺のMellin変換は上で見たように$\zeta$で右辺のMellin変換は
\begin{align*}
\int^\infty_0t^{-1/2}\theta(1/t)t^{s/2}\frac{dt}{t}
&=\int^0_\infty t^{1/2}\theta(t)t^{-s/2}\frac{dt}{t}\\
&=M(\theta((1-s)/2))
\end{align*}

これが函数等式です。

\section{Dirichlet $L$}
次にDirichlet $L$関数の函数等式を証明します

\begin{align*}
L(\chi,s)=\sum_n\frac{\chi(n)}{n^s}
\end{align*}

です。
$\chi=1$としたものが$\zeta$です。

関数等式の証明の準備として、いくつかの関数を定義します。
テータ関数の変形版を定義します。
\begin{align*}
\theta_a\\
\theta^a\\
\zeta(s,a)=\sum_n(n+a)^{-s}\\
\end{align*}

\section{Dedekind zeta}
参考文献
小野孝
Weil

\section{代数体}
代数体$k$とは有理数体$\Q$の有限次拡大のことをいう。

例。
二次体、円分体
$d$を平方数でない整数とする。
$\sqrt{d}$は$x^2-d=0$を満たし、$\sqrt{d}\notin\Q$である。
$k=\Q(\sqrt{d})$は$\Q$上二次拡大であり、代数体である。

$n$を正の整数とし、$\zeta_n$を$1$の原始$n$乗根、つまり$n$乗して初めて$1$になる数とする。
これは$x^n-1=0$の根で、したがって$\Q(\zeta_n)/\Q$は高々$n$次拡大である。

$k$の整数環を
$k$の元であって$\Z$上整であるもの、つまり$\Q$上の最小多項式の係数が$\Z$であるもの。

\section{類数}
$k$を代数体とする。

\begin{dfn}
$k$の分数イデアルとは、$k$の部分$O_k$加群で有限性生であり、$\otimes_{O_k}k$すると$k$に一致するものをいう。
\end{dfn}

\begin{eg}
$k=\Q$の場合、

$k$が二次体の場合、

$k$が円分体の場合、
\end{eg}

\begin{dfn}
分数イデアルの積を、と定義する。
\end{dfn}

これは可換。
生成元で書けば、

逆元の存在。
双対を用いた表示

分数イデアル全体は群になる。
これを$I(k)$と書く。

$k$の主イデアルとは、$k$の元$x$が生成するイデアル$O_kx \subset k$のことをいう。

主イデアル全体$P(k)$は分数イデアル全体の部分群

\begin{dfn}
この商群$I(k)/P(k)$を$k$のイデアル類群という。
\end{dfn}

これは有限群になる。

$k$の類数とは$k$のイデアル類群の大きさ。

\section{基本単数}

\begin{dfn}[Definition 7, p.94]
$n^{-1}\delta, l(\epsilon_1),\ldots,l(\epsilon_r)$を行ベクトルに持つ行列を$L$とし、$R=\abs{\det(L)}$を$k$のregulatorと呼ぶ。
\end{dfn}

\begin{prop}[Proposition 9, p.95]
$\gamma=\prod_v\gamma_v$で有限素点では$\gamma_v(O_v^\times)=1$で、実では$d\gamma_v(x)=\abs{x}^{-1}dx$で、複素では$d\gamma_v(x)=(x\bar{x})^{-1}\abs{dx\wedge d\bar{x}}$で定まる$\A_k^\times$のHaar測度とする。

$m>1 \in \R$にたいし、$C(m) \subset \A_k^\times / k^\times$における$1\leq \abs{z}\leq m$の像とする。

\begin{align*}
\gamma(C(m)) = \log(m)2^{r_1}(2\pi)^{r_2}hR/e
\end{align*}
となる。
\end{prop}
\section{類数公式}
Dedekind zetaの$s=1$での留数の計算

\subsection{BNT}
Weilの本に沿った証明。

\begin{thm}[Theoerm 3, p.129]
$k$を代数体とし、$r_1$を実素点の個数、$r_2$を複素素点の個数とする。
\begin{align*}
Z_k(s)=G_1(s)^{r_1}G_2(s)^{r_2}\zeta_k(s)
\end{align*}
とすると、これは$x=0, 1$で一位の極を持つ有理型関数で、関数等式
\begin{align*}
Z_k(s)=\abs{D}^{1/2-s}Z_k(1-s)
\end{align*}
をみたす。
ここで$D$は$k$のdiscriminantである。
$s=1$での留数は
\begin{align*}
\abs{D}^{-1/2}2^{r_1}(2\pi)^{r_2}hR/e
\end{align*}
である。
ここで$h$は$k$の類数、$R$はregulatorで$e$は$k$における$1$の冪根の個数。
\end{thm}

$\Phi'$を$\Phi$のFourier変換とし、$a$を$\chi$のdifferential ideleとすると、
\begin{align*}
\Phi'(y)=\abs{a}^{1/2}_{\A}\Phi(ay)
\end{align*}
となる。

特に$\omega_s(s)=\abs{x}^s_{\A}$とすると、
\begin{align}
Z(\omega_s, \Phi')=\abs{a}^{1/2-s}_{\A}Z(\omega_s,\Phi)
\end{align}

\begin{align}
Z(\omega_s,\Phi)=c_k^{-1}\prod_{w\in P_\infty}G_w(s)\prod_{v\notin P_\infty}(1-q_v^{-s})^{-1}
\end{align}

\begin{prop}[Proposition 12, p.128]
$k$を代数体とし、$\mu, \gamma$を$\A_k^\times$の適切な測度とした時、$\gamma=c_k\mu$となる。

$\gamma=\prod_v\gamma_v$で有限素点では$\gamma_v(O_v^\times)=1$で、実では$d\gamma_v(x)=\abs{x}^{-1}dx$で、複素では$d\gamma_v(x)=(x\bar{x})^{-1}\abs{dx\wedge d\bar{x}}$で定める。
\end{prop}

次はいわゆるdiffernt-discriminant formulaである。
\begin{prop}[Proposition 6, p.113]
$k$を代数体とし、$a$をdifferential ideleとすると、$\abs{a}_{\A}=\abs{D}^{-1}$である。ここで$D$は$k$のdiscriminantである。
\end{prop}

\begin{lem}[Lemma 6, p.121]
$F_1:N \to [0,1]$を可測関数とする。
$[t_0,t_1] \subset \R_+^\times$であって、$n<t_0$について$F_1(n)=1$であり、$n>t_1$について$F_1(n)=0$とする。
このとき
\begin{align*}
\lambda(s)=\int_Nn^sF_1(n)d\nu(n)
\end{align*}
は$Re(s)>0$で絶対収束し、$s\in\C$に解析接続され$s=0$での留数は
\end{lem}

\begin{thm}[Theorem 2, p.121]
$\Phi$を$\A_k$のstandard functionとする。
\begin{align*}
\omega \mapsto Z(\omega, \Phi)=\int_{\A_k^\times}\Phi(j(z))\omega(z)d\mu(z)
\end{align*}
は$\Omega(G_k)$上の有理型関数に解析接続され、関数等式
\begin{align*}
Z(\omega,\Phi)=Z(\omega_1\omega^{-1},\Phi')
\end{align*}
を満たし、$\omega_0, \omega_1$で留数$-\rho\Phi(0), \rho\Phi'(0)$をそれぞれ持つ。
\end{thm}
\begin{proof}
$\R^\times_+$上の関数$F_0, F_1$を次をみたすようにとる。
\begin{enumerate}
\item $F_0\geq0, F_1\geq0, F_0+F_1=1$
\item ある$[t_0,t_1] \subset \R^\times_+$が存在して、$F_0(t)=0, 0<t<t_0, F_1(t)=0, t>t_1$をみたす
\end{enumerate}
\end{proof}

\subsection{Tauberian theorem}
数論序説はこっち?

\chapter{Fourier変換}
Katzのtravaux de Laumonの解読
\section{背景}

$G$を有限アーベル群とし、$\check{G}$をそのPontryagin双対とする。
$f$を$G$上の関数とし、そのフーリエ変換$FT(f)$は$\check{G}$上の関数で
\begin{align*}
  \chi\mapsto\sum_{x\in G}f(x)\chi(x)
\end{align*}
で定まるもの。
逆変換により
\begin{align*}
  FT(FT(f))(x)=\abs{g}f(-x)
\end{align*}
が成立する。

特に、$G=E$が有限体$k$上の有限次元ベクトル空間である場合を考える。
$\check{E}$をその双対空間とし、$\langle x,y\rangle$を$E\times\check{E}$の内積。
$\psi:k \to \C^\times$を非自明な指標とする。
この時、$\psi(\langle x,y\rangle)$により$E$と$\check{E}$はPontryagin双対となり、
Fourier変換は
\begin{align*}
  y\mapsto\sum_{x\in E}f(x)\psi(\langle x,y\rangle)
\end{align*}
となる。$y\in\check{E}$を$x\mapsto \psi(\langle x,y\rangle)$により$E$の指標とみなす。

特に$E=k=\F_p$の場合、$\psi(x)=\exp(\dfrac{2\pi ix}{p})$とすると、
\begin{align*}
  FT(f)(y)=\sum_{x\in\F_p}f(x)\exp(\frac{2\pi ixy}{p})
\end{align*}
となる。

$P(x_1,\ldots,x_n)\in\Z[x_1,\ldots,x_n]$をとる。
$N\in\F_p$に対して$P(x)=N$の解の個数を$f(N)$とする。
これに対し$FT(f)$は$\ell$進コホモロジーの理論を用いて調べることができる。

このため、$\ell$進コホモロジーの理論について復習する。
$X$を$\Z[\dfrac{1}{\ell}]$上有限型のスキームで、連結とする。
$\xi$を$X$の適当な幾何的点として基本群$\pi_1=\pi_1(X,\xi)$とする。
$X$のlisse $\overline{\Q}_\ell$層$\mathcal{F}$とは、$\pi_1$の有限次元$\overline{\Q}_\ell$
表現で、ある局所コンパクト部分体上定義されるもの。
構成可能層や複体については省略。
$k$を有限体とし、$x\in X(k)$に対し、
\begin{align*}
  \trace_{k,\mathcal{F}}(x)=\trace(F_k\vert(\phi_{x,k})^*\mathcal{F})
\end{align*}
同様に複体$K$に対しては$\mathcal{H}^i=\mathcal{H}^i(K)$とおいて
\begin{align*}
  \trace_{k,K}=\sum_i(-1)^i\trace_{k,\mathcal{H}^i}
\end{align*}
と定める。

$\phi:X \to Y$が与えられた時、$(\phi_k)^*$と$(\phi_k)_!$が定まる。
traceは複体に対するpullbackと可換になる。
一方で、pushとの整合性はtrace formulaから次が従う。
\begin{align*}
  \trace_{k,R\phi_!(K)}=(\phi_k)_!(\trace_{k,K})
\end{align*}

$\ell$進コホモロジーにおける積分変換を定義する。
$X\times_SY$上の$\ell$進層$\mathcal{F}$を核関数とする積分変換$T_{\mathcal{F},!}$を、
$X$上の複体$K$に対して
\begin{align*}
  T_{\mathcal{F},!}(K)=Rpr_{2!}(pr_1^*(K)\otimes\mathcal{F})
\end{align*}
で定める。
この時trace formulaにより、$T_{\mathcal{F},!}(K)$のtrace functionは
\begin{align*}
  y\mapsto\sum_{x\mapsto s}\trace_{k,K}(x)\trace_{k,\mathcal{F}}(x,y)
\end{align*}
を満たす。

$\psi$を加法群$k$の非自明$\overline{\Q}_\ell$値指標とし、
Artin-Schereier被覆$\A^1\to\A^1$を$\psi$で押すと$\mathcal{L}_\psi$が定まる。
これのtrace functionは$\psi$自身になる。
$\A^n\times\A^n\to \A^1$を$\sum_ix_iy_i$で定め、これで $\mathcal{L}_\psi$を引き戻す。
これを核関数として積分変換$FT_{\psi,!}, FT_{\psi,*}$を定義する。

ここでVerdierの定理は、この二つが一致することを主張する。

\section{Stationary Phase}
$\phi\in C^\infty_c(\R^n), f\in C^\infty$とし、$t\in\R$に対し
\begin{align*}
  \int\phi(x)\exp(itf(x))dx
\end{align*}
を考える。
principle of stationary phaseとは、$grad(f)$が$Supp(\phi)$で消えないとすると、
積分は$t$の関数として$\infty$で急減少。
このことから、$f$が$Supp(\phi)$で有限個のcritical pointを持つなら、$t\to \infty$で
積分がcritical pointの寄与で定まる有限和に漸近する。

これの$p$進類似。


\chapter{関数等式}

参考
http://www-users.math.umn.edu/\~garrett/m/mfms/notes\_c/analytic\_continuations.pdf

Riemann$\zeta$のテータ関数を使った解析接続と関数等式を別の状況で。

$\chi$をDirichlet指標$\mod N$とし、
\begin{align*}
  L(s,\chi)=\sum_n\chi(n)n^{-s}
\end{align*}
と定める。

$\chi$がeven、つまり$\chi(-1)=1$とする。
$\chi$の導手が$N$とする。つまり$\mod N$で原始的と仮定する。
\begin{align*}
  \theta_\chi(iy)=\sum_{n\in\Z}\chi(n)e^{-\pi n^2y}
\end{align*}
と定める。($\chi$がoddだとこれは恒等的に$0$になってしまう。)

積分表示は
\begin{align*}
  \int^\infty_0y^{s/2}\frac{\theta_\chi(iy)}{2}\frac{dy}{y}
  &=\sum_{n\geq1}\chi(n)\int^\infty_0y^{s/2}e^{-\pi n^2y}\frac{dy}{y}\\
  &=\sum_{n\geq1}\frac{\chi(n)}{\pi^{s/2}n^s}\int^\infty_0y^{s/2}e^{-y}\frac{dy}{y}\\
  &=\pi^{-s/2}\Gamma(\frac{s}{2})L(s,\chi)
\end{align*}
で得られる。($\zeta$の場合と比較せよ。)

Poisson和公式を用いることで
  

\end{document}
