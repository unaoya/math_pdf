群のコホモロジー

群の拡大との関連について
https://ncatlab.org/nlab/show/group+extension

Gを群、Aをアーベル群とする。
H^2(G,A)とExt(G,A)は自然同値。
ここで、Ext(G,A)はGのAによる中心拡大のなす群。

中心拡大とは1 \to A \to \hat{G} \to G \to 1という完全列であって、Aの像が\hat{G}の中心に入るものをいう。

この対応を記述する。まずcocycle c:G^2 \to Aに対し、(g,a)(g’,a’)= (gg’, a+a’+c(g,g’))いよりG \times Aに群構造を定める。さらにこれがH^2(G,A)からExt(G,A)への射を誘導する。
逆に中心拡大に対し、c:G^2 \to Aをつぎのように定める。
まず集合としての切断G \to \hat{G}をとり、sとする。c(g,g’)=-s(g)^{-1}s(g’)^{-1}s(gg’)と定めると、これはsの選び方によらず定まる。
この二つの対応がH^2とExtの同型を定める。
