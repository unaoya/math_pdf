\documentclass[uplatex]{jsarticle}
\setlength{\textwidth}{\fullwidth}
\setlength{\evensidemargin}{\oddsidemargin}
\RequirePackage{amsmath,amssymb,amsthm, amscd, comment, multicol}
\usepackage[all]{xy}
\input{../tex/theorems}
\input{../tex/symbols}
\usepackage[dvipdfmx]{graphicx}
\usepackage{tikz, tikz-cd, tkz-euclide}
\usetkzobj{all}
\usetikzlibrary{intersections, calc}

%\newcommand{\Xb}{\overline{X}}
%\newcommand{\Gm}{\mathbb{G}m}


\title{直交多項式}
\author{梅崎直也@unaoya}
\date{\today}

\begin{document}
\maketitle

\section{Chebyshev多項式}
三角関数の倍角公式
三角関数の積分と直交関係
指標の直交性
第二種Chebyshevを指標で捉える(第1種は?)

直交多項式としての他の性質
微分方程式、母関数、漸化式など?
これらの表現論的解釈はできる?

母関数から微分方程式と漸化式を出せる?
\begin{align*}
\frac{1-tx}{1-2tx+t^2}&=\sum^\infty_{n=0}T_n(x)t^n\\
\frac{1}{1-2tx+t^2}&=\sum^\infty_{n=0}U_n(x)t^n\\
\end{align*}

これを$t$で微分すると、
\begin{align*}
\frac{-x(1-2tx+t^2)-(1-tx)(-2x+2t)}{(1-2tx+t^2)^2}&=\sum^\infty_{n=0}nT_n(x)t^{n-1}\\
\frac{-(-2x+2t)}{(1-2tx+t^2)^2}&=\sum^\infty_{n=0}nU_n(x)t^{n-1}\\
(2x-2t)\sum^\infty_{n=0}U_n(x)t^n&=(1-2tx+t^2)\sum^\infty_{n=0}nU_n(x)t^{n-1}\\
\sum^\infty_{n=0}2xU_n(x)t^n-\sum^\infty_{n=0}2U_n(x)t^{n+1}
&=\sum^\infty_{n=0}nU_n(x)t^{n-1}-\sum^\infty_{n=0}2nxU_n(x)t^{n}+\sum^\infty_{n=0}nU_n(x)t^{n+1}\\
\sum^\infty_{n=0}2xU_n(x)t^n-\sum^\infty_{n=1}2U_{n-1}(x)t^n
&=\sum^\infty_{n=0}(n+1)U_{n+1}(x)t^n-\sum^\infty_{n=0}2nxU_n(x)t^{n}+\sum^\infty_{n=1}(n-1)U_{n-1}(x)t^n\\
2xU_n(x)-2U_{n-1}(x)&=(n+1)U_{n+1}(x)-2nxU_n(x)+(n-1)U_{n-1}(x)\\
(n+1)U_{n+1}(x)-2(n+1)xU_n(x)+(n+1)U_{n-1}(x)&=0
\end{align*}
となるので、漸化式
\begin{align*}
U_{n+1}(x)-2xU_n(x)+U_{n-1}(x)&=0
\end{align*}
が得られる。
これを$x$で微分すると、
\begin{align*}
U_{n+1}'(x)-2U_n(x)-2xU_n'(x)+U_{n-1}'(x)&=0
\end{align*}
となる。

母関数を$t$で二階微分すると、
\begin{align*}
\frac{-2t(1-2tx+t^2)^2-2(2x-2t)(-2x+2t)(1-2tx+t^2)}{(1-2tx+t^2)^4}&=\sum^\infty_{n=0}n(n-1)U_n(x)t^{n-2}\\
\frac{-2t(1-2tx+t^2)-2(2x-2t)(-2x+2t)}{(1-2tx+t^2)^3}&=\sum^\infty_{n=0}n(n-1)U_n(x)t^{n-2}\\
\frac{-2t+4t^2x-2t^3+8(x^2-2xt+t^2)}{(1-2tx+t^2)^3}&=\sum^\infty_{n=0}n(n-1)U_n(x)t^{n-2}\\
\end{align*}

母関数を$x$で微分すると、
\begin{align*}
\frac{(-t)(1-2tx+t^2)-(1-tx)(-2t)}{(1-2tx+t^2)^2}&=\sum^\infty_{n=0}T_n'(x)t^n\\
\frac{-(-2t)}{(1-2tx+t^2)^2}&=\sum^\infty_{n=0}U_n'(x)t^n\\
2t\sum^\infty_{n=0}U_n(x)t^n&=(1-2tx+t^2)\sum^\infty_{n=0}U_n'(x)t^n\\
\sum^\infty_{n=0}2U_n(x)t^{n+1}&=\sum^\infty_{n=0}U_n'(x)t^n-\sum^\infty_{n=0}2xU_n'(x)t^{n+1}+\sum^\infty_{n=0}U_n'(x)t^{n+2}\\
\sum^\infty_{n=1}2U_{n-1}(x)t^{n}&=\sum^\infty_{n=0}U_n'(x)t^n-\sum^\infty_{n=1}2xU_{n-1}'(x)t^{n}+\sum^\infty_{n=2}U_{n-2}'(x)t^{n}\\
2U_{n-1}(x)&=U_n'(x)-2xU_{n-1}'(x)+U_{n-2}'(x)
\end{align*}

\section{Laguerre多項式}
まず母関数を用いて定義する。$(1-s)^{-n}$の展開は、$(1-s)^{-n+1}$の展開を項別微分すればよい。
\begin{align*}
U(\rho,s)&=(1-s)^{-1}\exp(-\rho\frac{s}{1-s})\\
&=(1-s)^{-1}(1-\rho\frac{s}{1-s}+\frac{1}{2}(-\rho\frac{s}{1-s})^2+\frac{1}{3!}(-\rho\frac{s}{1-s})^3+\cdots)\\
&=(1-s)^{-1}-\rho s(1-s)^{-2}+\frac{1}{2}\rho^2s^2(1-s)^{-3}-\frac{1}{3!}\rho^3s^3(1-s)^{-4}+\cdots\\
&=1+s+s^2+s^3+s^4+\cdots\\
&-\rho s(1+2s+3s^2+4s^3+5s^4+\cdots)\\
&+\frac{1}{2}\rho^2s^2(1+3s+6s^2+10s^3+\cdots)\\
&-\frac{1}{3!}\rho^3s^3(1+4s+10s^2+\cdots)\\
&=1+(1-\rho)s+(1-2\rho+\frac{1}{2}\rho^2)s^2+(1-3\rho+\frac{3}{2}\rho^2-\frac{1}{6}\rho^3)s^3+\cdots
\end{align*}
の係数を用いて
\begin{align*}
U(\rho,s)=\sum^\infty_{q=0}\frac{1}{q!}L_q(\rho)s^q
\end{align*}
と定義し、$L_q(\rho)$をLaguerre多項式という。

両辺を$\rho$で微分すると
\begin{align*}
(1-s)^{-1}(-\frac{s}{1-s})\exp(-\rho\frac{s}{1-s})&=\sum^\infty_{q=0}\frac{1}{q!}L_q'(\rho)s^q
\end{align*}
となるので、整理すると
\begin{align*}
U(\rho,s)&=-\frac{1-s}{s}\sum^\infty_{q=0}\frac{1}{q!}L_q'(\rho)s^q\\
&=(1-\frac{1}{s})\sum^\infty_{q=0}\frac{1}{q!}L_q'(\rho)s^q\\
\end{align*}
となるので、$s$についての係数を比較すると
\begin{align*}
L_q'-qL_{q-1}'=-qL_{q-1}
\end{align*}
となる。

また、母関数の両辺を$s$で微分すると

これらの漸化式から、Laguerreの微分方程式を導くことができる。

Laguerreの陪多項式$L_q^p(\rho)$を$L_q(\rho)$の$p$回微分
\begin{align*}
L_q^p(\rho)=\frac{d^p}{d\rho^p}L_q(\rho)
\end{align*}
で定義する。
これの母関数と微分方程式は、Laugerreの母関数と微分方程式を$p$回微分すれば得られる。

この微分方程式が、水素原子の動径方程式である。

\section{Legendre多項式}
まず母関数で定義する。
\begin{align*}
T(z,s)&=\frac{1}{\sqrt{1-2sz+s^2}}\\
&=\sum^\infty_{l=0}P_l(z)s^l
\end{align*}
と定義する。

$z$や$s$で微分して係数比較すると、漸化式と微分方程式が出てくる。
これが角運動量演算子の固有値問題の微分方程式に対応する。

陪多項式
\begin{align*}
P^m_l(z)=(1-z^2)^{m/2}\frac{d^m}{dz^m}P_l(z)
\end{align*}
で定める。
これの母関数と微分方程式もある。


\section{Hermite多項式}
参考
\begin{enumerate}
\item http://www.nr.titech.ac.jp/~chiba/en/pdf/harmonic\_osc.pdf
\item http://www.u-gakugei.ac.jp/~nitta/operator.pdf
\end{enumerate}

\subsection{調和振動子のシュレディンガー方程式}
シュレディンガー方程式
\begin{align*}
(-\frac{\hbar^2}{2m}\frac{d^2}{dx^2}+V(x))\phi=E\phi
\end{align*}
で、ポテンシャルが$V(x)=\frac{k}{2}x^2$となる調和振動子を考える。

変数変換して式を単純化する。
\begin{align*}
\omega=\sqrt{\frac{k}{m}}, \xi=\sqrt{\frac{m\omega}{\hbar}}x
\end{align*}
とすることで、方程式は
\begin{align*}
\frac{\hbar\omega}{2}(-\frac{d^2}{d\xi^2}+\xi^2)\phi=\phi
\end{align*}
となる。

さらに
\begin{align*}
\phi(\xi)=u(\xi)e^{-\frac{\xi^2}{2}}, \frac{2E}{\hbar\omega}=2n+1
\end{align*}
とおくと、
\begin{align*}
\phi'&=u'e^{-\frac{\xi^2}{2}}-\xi ue^{-\frac{\xi^2}{2}}\\
\phi''&=(u''-2\xi u'-(u-\xi^2u))e^{-\frac{\xi^2}{2}}
\end{align*}
となるので、$u$に対する方程式をかくと
\begin{align*}
(\frac{d^2}{d\xi^2}-2\xi\frac{d}{d\xi}+2n)u=0
\end{align*}
となる。
これの解がエルミート多項式である。

\subsection{母関数による定義}
エルミート多項式の母関数による定義を紹介し、それが微分方程式の解になっていることを確かめる。
\begin{align*}
e^{-t^2+2\xi t}=\sum_{n=0}^\infty\frac{1}{n!}H_n(\xi)t^n
\end{align*}
により多項式$H_n(\xi)$を定める。
ここで左辺は$e^x=1+x+\frac{1}{2!}x^2+\cdots$に$x=-t^2+2\xi t$を代入すれば右辺が得られる。

まずはこれが上の$u$の微分方程式の解になることを確かめる。
この母関数の両辺を$\xi$で$2$回微分すると
\begin{align*}
2te^{-t^2+2\xi t}&=\sum_{n=0}^\infty\frac{1}{n!}H_n(\xi)'t^n\\
4t^2e^{-t^2+2\xi t}&=\sum_{n=0}^\infty\frac{1}{n!}H_n(\xi)''t^n
\end{align*}
となる。
また母関数の両辺を$2t\dfrac{d}{dt}$すると
\begin{align*}
(-4t^2+4\xi t)e^{-t^2+2\xi t}=\sum_{n=0}^\infty\frac{1}{n!}2nH_n(\xi)t^n
\end{align*}
となる。
これらをうまく組み合わせて左辺を消去すると
\begin{align*}
0=\sum_{n=0}^\infty\frac{1}{n!}(H_n(\xi)''-2\xi H_n(\xi)'+2nH_n(\xi))t^n=0
\end{align*}
となり、$H_n(\xi)$の満たす微分方程式が得られる。
これが上で出てきた方程式と同じ形になっているので、
$H_n(\xi)$がシュレディンガー方程式の解を与えることがわかる。

以下では母関数を使ってエルミート多項式の直交関係、漸化式や別の表示(教科書の定義)などを証明する。

\subsection{直交関係式}
直交関係式を示し、固有関数を求める。
母関数を二乗すると
\begin{align*}
e^{-(\xi-t-s)^2+2st}=\sum_m\sum_nH_mH_ne^{-\xi^2}\frac{s^mt^n}{m!n!}
\end{align*}
となる。
これを$\int^\infty_{-\infty}d\xi$すると
\begin{align*}
\sqrt{\pi}e^{2st}=\sum_m\sum_n\int^\infty_{-\infty}H_mH_ne^{-\xi^2}dx\xi\frac{s^mt^n}{m!n!}
\end{align*}
となり、さらに$e^{2st}$をテイラー展開$e^{2st}=\sum_n\frac{1}{n!}2^ns^nt^n$して$t, s$の係数比較をすると、
\begin{align*}
\sum_m\int^\infty_{-\infty}H_mH_ne^{-\xi^2}d\xi=\sqrt{\pi}2^nn!\delta_{mn}
\end{align*}
となり、直交関係式が得られる。
このことから
\begin{align*}
\phi_n(\xi)=\frac{1}{\pi^{\frac{1}{4}}\sqrt{2^nn!}}e^{-\frac{\xi^2}{2}}H_n(\xi)
\end{align*}
がシュレディンガー方程式の固有関数であることがわかる。

\subsection{漸化式}
\begin{align*}
2te^{-t^2+2\xi t}&=\sum_{n=0}^\infty\frac{1}{n!}H_n(\xi)'t^n\\
e^{-t^2+2\xi t}&=\sum_{n=0}^\infty\frac{1}{n!}H_n(\xi)t^n
\end{align*}
から
\begin{align*}
2t\sum_n\frac{1}{n!}H_nt^n=\sum_n\frac{1}{n!}H_n't^n
\end{align*}
となるので
\begin{align*}
H_n'=2nH_{n-1}
\end{align*}
となる。
さらに、上の式を$t$で微分して
\begin{align*}
(-2t+2\xi)e^{-t^2+2\xi t}=\sum_n\frac{1}{(n-1)!}H_nt^{n-1}
\end{align*}
となるので、
\begin{align*}
(-2t+2\xi)\sum_n\frac{1}{n!}H_nt^n=\sum_n\frac{1}{(n-1)!}H_nt^{n-1}
\end{align*}
から係数比較して
\begin{align*}
-2nH_{n-1}+2\xi H_n&=H_{n+1}\\
-H_n'+2\xi H_n&=H_{n+1}
\end{align*}
が得られる。

\subsection{教科書の定義式}
\begin{align*}
\frac{d}{d\xi}e^{-\frac{\xi^2}{2}}f=e^{-\frac{\xi^2}{2}}(-\xi f+f')
\end{align*}
より
\begin{align*}
(\xi+\frac{d}{d\xi})e^{-\frac{\xi^2}{2}}f&=e^{-\frac{\xi^2}{2}}f'\\
(\xi-\frac{d}{d\xi})e^{-\frac{\xi^2}{2}}f&=e^{-\frac{\xi^2}{2}}(2\xi f-f')
\end{align*}
となる。
$f=H_n$として、上の漸化式を用いると
\begin{align*}
(\xi+\frac{d}{d\xi})e^{-\frac{\xi^2}{2}}H_n&=2ne^{-\frac{\xi^2}{2}}H_{n-1}\\
(\xi-\frac{d}{d\xi})e^{-\frac{\xi^2}{2}}H_n&=e^{-\frac{\xi^2}{2}}H_{n+1}\\
\end{align*}
となる。(ここからさらに頑張って昇降演算子の話に持っていくがそれはあとで)
さらに二つ目の式から
\begin{align*}
H_{n+1}=e^{\frac{\xi^2}{2}}(\xi-\frac{d}{x\xi})e^{-\frac{\xi^2}{2}}H_n
\end{align*}
となる。
これと$H_1=1$を使うと、
\begin{align*}
H_n=e^{\frac{\xi^2}{2}}(\xi-\frac{d}{d\xi})^ne^{-\frac{\xi^2}{2}}
\end{align*}
となる。
ここで、一般に
\begin{align*}
\frac{d}{d\xi}(e^{-\frac{\xi^2}{2}}f)=e^{-\frac{\xi^2}{2}}(\frac{d}{d\xi}-\xi)f
\end{align*}
となることから
\begin{align*}
\xi-\frac{d}{d\xi}=-e^{-\frac{\xi^2}{2}}\frac{d}{d\xi}e^{-\frac{\xi^2}{2}}
\end{align*}
となる。
これを使うと
\begin{align*}
H_n=(-1)^ne^{\xi^2}\frac{d^n}{d\xi^n}e^{-\xi^2}
\end{align*}
となる。これが求めたかった式。

\subsection{昇降演算子?}
\begin{align*}
(\xi-\frac{d}{d\xi})e^{-\frac{\xi^2}{2}}H_n&=e^{-\frac{\xi^2}{2}}H_{n+1}
\end{align*}
を$\phi_n$を用いて書くと、
\begin{align*}
\frac{1}{\sqrt{2}}(\xi-\frac{d}{d\xi})\phi_n=\sqrt{n+1}\phi_{n+1}
\end{align*}
となる。
同様に
\begin{align*}
\frac{1}{\sqrt{2}}(\xi+\frac{d}{d\xi})\phi_n=\sqrt{n}\phi_{n-1}
\end{align*}
となる。
これらから、
\begin{align*}
a^\dagger&=\frac{1}{\sqrt{2}}(\xi-\frac{d}{d\xi})\\
a&=\frac{1}{\sqrt{2}}(\xi+\frac{d}{d\xi})
\end{align*}
と定めると、
\begin{align*}
a^\dagger\phi_n&=\sqrt{n+1}\phi_{n+1}\\
a\phi_n&=\sqrt{n}\phi_{n-1}\\
aa^\dagger\phi_n&=(n+1)\phi_n\\
a^\dagger a\phi_n&=n\phi_n\\
aa^\dagger-a^\dagger a&=1
\end{align*}
が得られる。(これが昇降演算子?)
\begin{align*}
a^\dagger a&=\frac{1}{2}(-\frac{d^2}{d\xi^2}+\xi^2-1)\\
H&=\hbar\omega(a^\dagger a+\frac{1}{2})\\
H\phi_n&=\hbar\omega(n+\frac{1}{2})\phi_n
\end{align*}
と計算できる。
\end{document}