\documentclass{jsarticle}
\setlength{\textwidth}{\fullwidth}
\setlength{\evensidemargin}{\oddsidemargin}
\RequirePackage{amsmath,amssymb,amsthm, amscd, comment, multicol}
\usepackage[all]{xy}
\input{../tex/theorems}
\input{../tex/symbols}
\usepackage[dvipdfmx]{graphicx}
\usepackage{tikz}
\usepackage{tkz-euclide}
\usetkzobj{all}
\usetikzlibrary{intersections, calc}
\title{Gauss和}
\author{@unaoya}
\date{\today}
\begin{document}

\section{The Fourier Transform and Equations over Finite Abelian Groups}
参考文献
http://people.cs.uchicago.edu/~laci/reu02/fourier.pdf

\begin{thm}
$k$を整数$q$を素数$q\geq k^4+4$とする。
このとき
\[
x^k+y^k=z^k
\]
は$\F_q$に非自明な解を持つ。
\end{thm}
(Weilの予想の数え上げとの関係は?)

\begin{thm}
$k$を整数$A_1, A_2\subset\F_q$と$l_i=\dfrac{q-1}{\abs{A_i}}$とする。
\[
q\geq k^2i_1i_2+4
\]
のとき、
\[
x+y=z^k
\]
は$x \in A_1, y\in A_2, z\in \F_q$の解を少なくとも一つ持つ。
\end{thm}
上の定理は、$A_i=\{a^k,a\in\F_q\}$とすれば出る。

$G$を位数$n$のアーベル群とする。
$G$の指標とは群準同型$\chi: G\to \C^\times$のこと。
自明な指標を$\chi_0$と書くことにする。
\begin{prop}
$\chi$が非自明なら
\[
\sum_{a\in G}\chi(a)=0
\]
が成り立つ
\end{prop}

\begin{cor}
指標の直交性
\end{cor}

$\hat{G}$を$G$の指標のなす群とする。

$A\subset G$に対し$\Phi(A)$を$A$の特性関数$f_A$の$\chi_0$以外のFourier係数の絶対値の最大値とする。
\begin{thm}
$A\subset \F_q$をcyclotomic classとする。
このとき
\[
\Phi(A) < \sqrt{q}
\]
\end{thm}

$a\in G$と$A_1,\ldots,A_k\subset G$に対し、
\[
x_1+\cdots+x_k=a~(x_i\in A_i)
\]
の解の個数を考えると、平均は
\[
\frac{m_1\cdots m_k}{n}
\]
となる。
分散は?

$a=0$の場合の解の個数を$N$とする。

\begin{thm}[Theorem 3.1]
\[
N=\frac{m_1\cdots m_k}{n}+R
\]
であり、
\[
R=\frac{1}{n}\sum_{\chi\in\hat{G}, \chi\neq\chi_0}\prod_{i=1}^k\hat{f}_{A_i}(\chi)
\]
と計算できる。
\end{thm}

$\delta\in\C^G$を$0$に台を持つデルタ関数とする。
これに対し、
\[
\delta=\frac{1}{n}\sum_{\chi\in\hat{G}}\chi
\]
が成り立つ。

このことから
\begin{align*}
N=\sum_{(x_i)\in\prod_iA_i}\delta(x_1+\cdots+x_k)=\frac{1}{n}\sum_{\chi\in\hat{G}}\sum_{(x_i)\in\prod_iA_i}\chi(x_1+\cdots+x_k)
\end{align*}
と計算できる。

$\chi$が指標であるから$\chi(x_1+\cdots+x_k)=\chi(x_1)\cdots\chi(x_k)$であり、
和を因数分解すると
\begin{align*}
\sum_{(x_iin\prod_iA_i}\chi(x_1+\cdots+x_k)=\prod_i\sum_{x_i\in A_i}\chi(x_i)=\prod_i\hat{f}_{A_i}(\chi)
\end{align*}
と計算できる。

\begin{thm}[Theorem 4.1]
\begin{align*}
\abs{N-\frac{\abs{A_1}\abs{A_2}\abs{A_3}}{n}}<\Phi(A_3)\sqrt{\abs{A_1}\abs{A_2}}
\end{align*}
\end{thm}


\begin{thm}[Theorem 7.1]
$k\vert q-1$を整数とし$A_1,A_2\subset \F_q$とする。
$N$を方程式
\begin{align*}
x+y=z^k~(x\in A_1,y\in A_2, z\in\F_q^\times)
\end{align*}
の解の個数とする。

このとき
\begin{align*}
\abs{N-\frac{\abs{A_1}\abs{A_2}(q-1)}{q}}<k\sqrt{\abs{A_1}\abs{A_2}q}
\end{align*}
が成立する。
\end{thm}
\begin{proof}
$A_3=H(q,k)$とし、$N'$を
\begin{align*}
x+y=u~(x\in A_1, y\in A_2, u\in A_3)
\end{align*}
の解の個数とする。
$k\vert q-1$なので$z^k=u$は$k$個解を持つ。
したがって$N=kN'$となる。

Theorem 4.1から
\begin{align*}
\abs{N-\frac{\abs{A_1}\abs{A_2}\abs{A_3}}{n}}<\Phi(A_3)\sqrt{\abs{A_1}\abs{A_2}}
\end{align*}
であり、
Theorem 6.7から
\begin{align*}
\Phi(A_3)<\sqrt{q}
\end{align*}
\end{proof}

\begin{thm}[Theorem 6.8]
$A$をcyclotomic classすなわち$bH(q,k)\subset \F_q$とする。
このとき
\begin{align*}
\Phi(A)<\sqrt{q}
\end{align*}
\end{thm}
\begin{proof}
特性関数$f_A$のFourier係数$\hat{f}_A(\chi)$の絶対値を計算する。
これは次のようにGauss和を用いて記述できる。
\begin{align*}
\hat{f}_A(\chi)=\frac{1}{k}\sum_{i=0}^{k-1}S(\chi,\psi_i)
\end{align*}
またGauss和の絶対値は
\begin{align*}
\abs{S(\chi,\psi)}=\sqrt{q}
\end{align*}
と計算できる。
これを合わせて
\end{proof}

分布の話
$q, A$などをいろいろ変えて分布を見る。
Sato-Tate?

\section{AC}
\subsection{セミナーと日曜}
\begin{dfn}
additive character $\chi:\F_p\to\C^\times$とmultiplicative character $\psi:\F_p^\times\to\C^\times$にたいし
(記号逆の方がいいきがする)
\begin{align*}
S(\chi,\psi)=\sum_{a\in\F_p}\chi(a)\psi(a)=\sum_{a\in\F_p^\times}\chi(a)\psi(a)
\end{align*}
ここで$\psi(0)=0$としておく。
\end{dfn}
multiplicative charは例えばLegendre symbolなど。

\begin{prop}
Gauss和の評価
\begin{align*}
\abs{S(\chi,\psi)}=\sqrt{q}
\end{align*}
\end{prop}
\begin{proof}
\begin{align*}
S(\chi,\psi)\overline{S(\chi,\psi)}=\sum_{a\in\F_p^\times}\chi(a)\psi(a)\sum_{b\in\F_p^\times}\overline{\chi(b)}\overline{\psi(b)}\\
=\sum_{a,b}\chi(a-b)\psi(ab^{-1})
\end{align*}
ここで$a\in\F_p$に対して$ba^{-1}=c$とすると
\begin{align*}
\sum_{a,b}\chi(a-b)\psi(ab^{-1})=\sum_{a,c}\chi(a-ac)\psi(c)\\
=\sum_c\psi(c)\sum_a\chi(a(1-c))
\end{align*}
次に$c\in\F_p^\times$に対して$\sum_a\chi(a(1-c))$を計算する。
$c=1$の場合、
$c\neq1$の場合、
\end{proof}
$\F_q$の部分群と、その商の指標のGauss和の関係を与える。
Gauss和とGauss周期の関係?と解釈できる?
\begin{prop}
$A=H(k,q)\subset \F_q^\times$を指数$k$の部分群とする。
$\psi_0,\ldots,\psi_{k-1}$を$\F_q^\times$の指標から誘導される$\F_q^\times$の指標とする。
$\F_q$の指標$\chi$にたいし
\begin{align*}
\hat{f}_A(\chi)=\frac{1}{k}\sum^{k-1}_{i=0}S(\chi,\psi_i)
\end{align*}
\end{prop}
\begin{proof}
まず$\F_q^\times/A$の指標の性質から
\begin{align*}
\sum_{i=0}^{k-1}\psi_i(a)=\begin{cases}0~(a\not\in A\mbox{もしくは}a=0)\\k~(a\in A)\end{cases}
\end{align*}
\end{proof}
である。

\subsection{セミナー原稿}
\begin{thm}
$A_1, A_2, A_3 \subset G$とし$N$を$x_1+x_2+x_3=0, x_i\in A_i$の解の個数とする。
この時
\begin{align*}
\abs{N-\frac{\abs{A_1}\abs{A_2}\abs{A_3}}{n}}<\Phi(A_3)\sqrt{\abs{A_1}\abs{A_2}}
\end{align*}
が成り立つ。
\end{thm}
\begin{proof}
\begin{align*}
N=\sum_{x_i\in A_i}\delta(x_1+x_2+x_3)\\
=\frac{1}{n}\sum_{x_i\in A_i}\sum_{\chi\in\hat{G}}\chi(x_1+x_2+x_3)\\
=\frac{1}{n}\sum_{x_i\in A_i}\chi_0(x_1+x_2+x_3)+\frac{1}{n}\sum_{\chi\neq\chi_0}\sum_{x_i\in A_i}\chi(x_1+x_2+x_3)
\end{align*}
となる。
\begin{align*}
\sum_{x_i\in A_i}\chi_0(x_1+x_2+x_3)=\sum_{x_i\in A_i}\chi_0(x_1)\chi_0(x_2)\chi_0(x_3)=\abs{A_1}\abs{A_2}\abs{A_3}
\end{align*}
となる。

第二項を評価していく。
\begin{align*}
\abs{\sum_{\chi\neq\chi_0}\sum_{x_i\in A_i}\chi(x_1)\chi(x_2)\chi(x_3)}&
=\abs{\sum_{\chi\neq\chi_0}(\sum_{x\in G}\chi1_{A_1}(x))(\sum_{x\in G}\chi1_{A_2}(x))(\sum_{x\in G}\chi1_{A_3}(x))}\\
&=\abs{\sum_{\chi\neq\chi_0}\hat{1}_{A_1}(\chi)\hat{1}_{A_2}(\chi)\hat{1}_{A_3}(\chi)}\\
&\leq\Phi(A_3)\sum_{\chi\in\hat{G}}\abs{\hat{1}_{A_1}(\chi)}\abs{\hat{1}_{A_2}(\chi)}\\
\end{align*}
と計算できる。

さらに$\C^G$がHilbert空間なので、Cauchy-Schwarzより
\begin{align*}
\sum_{\chi\in\hat{G}}\abs{\hat{1}_{A_1}(\chi)}\abs{\hat{1}_{A_2}(\chi)}
&\leq\sqrt{(\sum_{\chi\in\hat{G}}\abs{\hat{1}_{A_1}(\chi)}^2)(\sum_{\chi\in\hat{G}}\abs{\hat{1}_{A_2}(\chi)}^2)}\\
&=\sqrt{n^2\abs{A_1}\abs{A_2}}
\end{align*}
と計算できる。
\end{proof}


\begin{dfn}[Gauss和]
$G=\Z/p$とし$\chi:G \to \C^\times$と$\psi:G^\times\cup\{0\}\to\C^\times$に対し
\begin{align*}
S(\chi,\psi)=\sum_{x\in G}\chi(x)\psi(x)
\end{align*}
と定義する。
\end{dfn}
三角和

Gauss和とGauss周期の関係について。
Fourier変換
\begin{align*}
\hat{1}_A(\chi)=\sum_{x\in A}\chi(x)
\end{align*}
において$\chi(x)=\exp(2\pi ix/p)$で$A$を部分群のcosetとすればよい。

$\hat{1}_A$の評価の実解析での類似は?
$\Gamma$関数のFourier係数の評価とか?

\begin{prop}
$A\subset G^\times$を指数$k$の部分群とし$\psi_0,\ldots,\psi_{k-1}$を$G^\times/A$の指標全体とする。
\begin{align*}
\sum^{k-1}_{i=0}S(\chi,\psi_i)&=\sum^{k-1}_{i=0}\sum_{x\in G}\chi(x)\psi_i(x)\\
&=\sum_{x\in G}\chi(x)(\sum_{i=0}^{k-1}\psi_i(x))
\end{align*}
ここで指標の和に関する公式
\begin{align*}
\sum_{\psi\in\hat{H}}\psi(x)=\begin{cases}\abs{H}&(x=0)\\0&(x\neq0)\end{cases}
\end{align*}
を思い出すと、
上の計算の続きは
\begin{align*}
=\sum_{x\in G}\chi(x)k1_A(x)=\hat{1}_A(x)
\end{align*}
とできる。
\end{prop}

\begin{prop}
Gauss和の評価。
$\psi$が非自明な時
\begin{align*}
\abs{S(\chi,\psi)}&=\sum_{x\in G}\chi(x)\psi(x)\sum_{y\in G}\overline{\chi(y)}\overline{\psi(y)}\\
&=\sum_{x,y\in G}\chi(x)\chi(-y)\psi(x)\psi(y^{-1})\\
&=\sum_{x,y\in G^\times}\chi(x-y)\psi(xy^{-1}
\end{align*}
ここで$z=xy^{-1}$と変数変換すると
\begin{align*}
=\sum_{z,y\in G^\times}\chi(yz-y))\psi(z)\\
=\sum_{z\in G^\times}\psi(z)\sum_{y\in G^\times}\chi((z-1)y)
\end{align*}
となる。
さらに$z=1$かどうかで場合分けすると、$\chi:G\to\C^\times$が指標なので
\begin{align*}
\sum_{y\in G^\times}\chi((z-1)y)=\begin{cases}q-1&(z=1 \mbox{or} \chi=\chi_0)\\-1&\mbox{o.w.}\end{cases}
\end{align*}
となる。
したがって、
\begin{align*}
=(q-1)\psi(1)+\sum_{z\neq1}\psi(z)(-1)\\
=q-1+\sum_{z\in G^\times}\psi(z)(-1)+1\\
=q
\end{align*}
となる。
\end{prop}

上の二つを合わせて
\begin{prop}
部分群$A\subset G^\times$について
\begin{align*}
\Phi(A)<\sqrt{q}
\end{align*}
\end{prop}

\begin{proof}
\begin{align*}
\abs{\hat{1}_A(\chi)}&\leq\frac{1}{k}\sum_{\psi}\abs{S(\chi,\psi)}\\
&\leq\frac{1}{k}(\abs{S(\chi,\psi_0)}+\sqrt{q}(k-1))
\end{align*}
であり、
\begin{align*}
S(\chi,\psi_0)=\sum_{x\in G^\times}\chi(x)=-1
\end{align*}
であることから、
\begin{align*}
\leq\frac{1}{k}(1+\sqrt{q}(k-1))\leq\sqrt{q}
\end{align*}
となる。
\end{proof}

\subsubsection{巡回群の指標}
\begin{lem}
$G$を巡回群で$\chi$をその指標としたとき
\begin{align*}
\sum_{x\in G}\chi(x)=\begin{cases}0&(\chi\neq\chi_0)\\\abs{G}&(\chi=\chi_0)\end{cases}
\end{align*}
\end{lem}
\begin{proof}
$S=\sum_{x\in G}\chi(x)$とする。
$\chi(y)S=S$より
\end{proof}

$\C^G$を$G\to\C$の集合とし、$\C$線形空間と思う。
この空間に内積を
\begin{align*}
(f,g)=\frac{1}{n}\sum_{x\in G}f(x)\overline{g(x)}
\end{align*}
とすることでHilbert空間となる。
特にCauchy-Schwarz

\begin{lem}
$G$の指標はHilbert空間$\C^G$の正規直交基底となる。
\end{lem}
\begin{proof}
 $\hat{G}\cong G$なので個数はあう。
直交することは、
\end{proof}

$f\in\C^G$のFourier変換を
\begin{align*}
\hat{f}(\chi)=\sum_{x\in G}\chi(x)f(x)
\end{align*}
と定める。
$\hat{f}\in\C^{\hat{G}}$となる。
また逆変換を
\begin{align*}
g(x)=\frac{1}{n}\sum_{\chi\hat{G}}g(\chi)\chi(-x)
\end{align*}
と定める。

$\delta\in\C^G$を$0$に台を持つ特性関数とすると、$\hat{\delta}(\chi)=1$であり、
Fourier逆変換公式から
\begin{align*}
\delta=\sum_{\chi\in\hat{G}}\chi
\end{align*}
となる。
これとtrace formulaの関係は?

\begin{thm}[Plancherel公式]
\begin{align*}
(\hat{f},\hat{g})=n(f,g)
\end{align*}
である。特に
\begin{align*}
\norm{f}=\norm{\hat{f}}
\end{align*}
\end{thm}
\begin{proof}
一点に台を持つ$\delta_a, \delta_b$を用いて確かめると
\begin{align*}
(\delta_a,\delta_b)=\begin{cases}\frac{1}{n}&a=b\\0&a\neq b\end{cases}\\
\hat{\delta}_a(\chi)=\sum_{x\in G}\delta_a(x)\chi(x)=\chi(a)\\
(\hat{\delta}_a,\hat{\delta}_b)0\frac{1}{n}\sum_{\chi\in\hat{G}}\chi(a)\chi(b)=\begin{cases}1&a=b\\0&a\neq b\end{cases}
\end{align*}


\end{proof}

\subsection{数学カフェ}
関数解析とtrace formula
$L(\Gamma\backslash G)$への$R(f)$の作用のtrace
表現の分解を用いてtraceを計算、軌道積分を用いて計算、この二つの一致
\subsection{日曜2}
Selberg zeta、函数等式など
\subsection{コンピュータ}
実験。Sato-Tate
\subsection{表現論のノート}
前に作ったやつ。
一回セミナーする?
分割してアップする。

\section{Deirichlet $L$}
https://www.dpmms.cam.ac.uk/~rdh46/lecture\_notes/lecture\_4.pdf

Dirichlet $L$の函数等式。
$\theta$の保型性。

\begin{thm}[Poisson和公式]
$f\in C^1(\R)$が無限遠で適当な条件を満たすとする。
このとき
\begin{align*}
\sum_{n\in\Z}f(n)=\sum_{m\in\Z}\hat{f}(m)
\end{align*}
\end{thm}

\begin{proof}
$F(x)=\sum_{n\in\Z}f(n+x)$を$\R/\Z$上の関数として定義する。
これのFourier変換から
\begin{align*}
F(x)=\sum_m\hat{F}e(mx)
\end{align*}
となる。
この両辺の$x=0$での値を見ればよい
\end{proof}

\section{Quadratic Reciprocity and the Sign of the Gauss Sum via the Finite Weil Representation}
https://www.math.wisc.edu/~shamgar/QR-IMRN.pdf

数学のたのしみにも似た話あった?

\begin{dfn}[Gauss和]
$a\in(\Z/n\Z)^\times$にたいし
\begin{align*}
G_n(a)=\sum_{x\in\Z/n\Z}e^{\frac{2\pi i}{n}ax^2}
\end{align*}
とし、$G_n=G_n(1)$とする。
\end{dfn}

\begin{lem}[Lemma 4.1]
\begin{align*}
\left(\frac{p}{q}\right)\left(\frac{q}{p}\right)=\frac{G_pG_q}{G_{pq}}
\end{align*}
\end{lem}
\begin{proof}
\begin{align*}
\sum_{x\in\F_p}e^{\frac{2\pi i}{p}ax^2}=\sum_{x\in\F_p}e^{\frac{2\pi i}{p}ax}\left(\frac{x}{p}\right)
\end{align*}
\end{proof}


\begin{prop}[Proposition 4.2]
\begin{align*}
Tr(\rho_{pq}(w))=Tr(\rho_p(w))Tr(\rho_q(w))
\end{align*}
\end{prop}

DFTとWeil表現の関係について。
$\psi$をadditive character $\Z/n \to \C^\times$とする。
これによるFourier変換$F_\psi:\C^G\to\C^{\hat{G}}$を
\begin{align*}
F_\psi(f)(y)=\sum_{g\in G}\psi(yg)f(g)
\end{align*}
と定義する。
特に$\psi:z\mapsto\exp(\frac{2\pi i}{n}z$に対するものを$F_n$と書く。
\begin{prop}
\begin{align*}
\det(F_n)=i^{\frac{n(n-1)}{2}}n^{\frac{n}{2}}
\end{align*}
\end{prop}

\begin{align*}
\rho_n:SL_2(\Z/n\Z)\to \GL(H)
\end{align*}
をWeil表現とする。
\begin{lem}
\begin{align*}
F_n=i^{\frac{n(n-1)}{2}}\rho_n(w)
\end{align*}
\end{lem}
Weil表現の定義を確認。
$\pi(h)$との整合性の条件から定数倍を除いて決まってしまう?

この辺りを詳しく。
$\pi$と$\pi^g$は同型になるか?
central characterは一致する。
これだけで同型がわかる?


\section{Dirichlet's calculation of Gauss sum}
https://www.math.ubc.ca/~cass/research/pdf/dirichlet.pdf

\section{Fourier transform over finite field and identities between Gauss sums}
https://arxiv.org/abs/math/0003011

\section{DOUBLE AFFINE HECKE ALGEBRAS AND DIFFERENCE FOURIER TRANSFORMS}
https://arxiv.org/pdf/math/0110024.pdf
https://mathoverflow.net/questions/11053/whats-the-relationship-between-gauss-sums-and-the-normal-distribution

\end{document}