\documentclass[uplatex]{jsarticle}
\RequirePackage{amsmath,amssymb,amsthm, amscd}
\usepackage[all]{xy}
\input{../tex/theorems}
\input{../tex/symbols}
\usepackage[dvipdfmx]{graphicx}
\usepackage{tikz, tikz-cd, tkz-euclide}
\usetkzobj{all}
\usetikzlibrary{intersections, calc}
\title{結び目とエタールコホモロジー}
\author{梅崎直也@unaoya}
\date{\today}
\begin{document}
\maketitle

\begin{abstract}
この講演ではエタールコホモロジーについて、それがどのような性質を持つのかを中心にお話しします。特にそれらの性質を用いることで、ある種の群や代数系の表現を幾何学的に調べることができます。具体的にはKazhdan-Lusztig多項式やSpringer対応といった話題について紹介します。また、そのような対象を通してエタールコホモロジーが結び目の研究と交わる可能性について考えてみたいと思います。
\end{abstract}
\end{document}