\documentclass[uplatex]{jsarticle}
\RequirePackage{amsmath,amssymb,amsthm, amscd}
\usepackage{mathrsfs}
\usepackage[all]{xy}
\input{../tex/theorems}
\input{../tex/symbols}
\usepackage[dvipdfmx]{graphicx}
\usepackage{tikz, tikz-cd, tkz-euclide}
\usetkzobj{all}
\usetikzlibrary{intersections, calc}


\title{結び目とエタールコホモロジー}
\author{梅崎直也@unaoya}
\date{\today}
\begin{document}
\maketitle

\begin{abstract}
この講演ではエタールコホモロジーについて、それがどのような性質を持つのかを中心にお話しします。特にそれらの性質を用いることで、ある種の群や代数系の表現を幾何学的に調べることができます。具体的にはKazhdan-Lusztig多項式やSpringer対応といった話題について紹介します。また、そのような対象を通してエタールコホモロジーが結び目の研究と交わる可能性について考えてみたいと思います。
\end{abstract}

\section{幾何学的表現論}
Weyl群
$G=GL_n$なら$W=S_n$
リー代数$\fg$と$\fh$をそのCartanとしたとき、$W\subset GL(\fh^*)$は以下のような$S$で生成される。
$s_i(\lambda)=\lambda-\lambda(h_i)\alpha_i$で定める。
ここで$h_i\in \fh, \alpha_i\in\fh^*$でこれらが生成系
例えば$W=S_n$の時、$S$は$i,i+1$の互換。
これをいくつかけるかで$w\in W$に対し$l(w)$を定めることができる。

\subsection{Springer対応}
Weyl群$W$の表現と$G$の冪単共役類$u$
冪単元$u$から多様体$\sB_u$を$u$を含むBorel全体のなす多様体とする。
$\sB_u$のコホモロジーに表現を実現

$Z$を$(u,B,B')$のなす多様体とする。
$u$は冪単元、$B, B'$は$u$を含むBorel

コホモロジーの二つの基底
これの変換にKazhdan-Lusztig多項式が現れる。
advance80の定義
ホモロジーはBorel-Moore、proper contiなら射を誘導する。
$H_{4v}(\hat{E}_\phi)$の基底として、$[\bar{C}_w]$と$\Gamma_w$がある。


\subsection{Kazhdan-Lusztig多項式}
無限次元表現の指標としての解釈

Kazhdan-Lusztig多項式の三つの解釈
Verma加群の組成列
旗多様体のコホモロジー
Hecke環

$\fg$を半単純Lie代数
$\fg$のweight $\lambda$に対して、そのVerma加群 $\Delta(\lambda)=U(\fg)\otimes_{U(\fb)}\C_\lambda$と定義。
これはただ一つの既約商 $L(\lambda)$をもつ。
$\Delta(\lambda)$の組成列とその重複度の記述
Mの指標を$ch(M)=\sum\dim M_\lambda e^\lambda$と定める。

Hecke環の定義
Coxeter系$S$とWeyl群$W$。
生成元と関係式による記述。
$w\in W$に対する$\Z[q^{1/2}, q^{-1/2}]$上の基底$T_w$をもち、関係式は
\begin{align*}
T_yT_w=T_{yw}, l(yw)=l(y)+l(w)\\
(T_s+1)(T_s-q)=0, s\in S
\end{align*}

例$W=S_n$の時。

$q\to 1$とすると$W$の群環$\Z[W]$である。

$G(\F_q)$の両側$B$不変関数の環としての記述。


旗多様体のコホモロジー、Bruhat胞体から定まる基底?$W$の作用から定まる基底?
$w\in W$に対して$G/B$のSchubert多様体と呼ばれる部分多様体$X_w$が定まる。
これのintersection homologyを用いてKazhdan-Lusztig多項式を記述できる。

\section{結び目の不変量}

\subsection{Webster-Williamson}
colored braid$\beta$から$GL(N)$の$P_\beta$両側同変複体を作る。
さらにそれのcohomologyをとることで三重次数複体を作る。
一方、colored linkから$X_D$の$G_D$同変複体を作り、
そのcohomologyからも三重次数複体を作る。
これらが一致し、HOMFLYPT多項式の圏化になっている。

$D$
\subsection{}
\end{document}