\documentclass[dvipdfmx]{beamer}
\usetheme{metropolis}
%\usecolortheme{beaver} 

\input{../tex/theorems}
\input{../tex/symbols}
\usepackage{bxdpx-beamer}
\usepackage{tikz-cd}
\usepackage[all]{xy}
\RequirePackage{amsmath,amssymb,amsthm, amscd, comment, multicol}

%https://tex.stackexchange.com/questions/178800/creating-sections-each-with-title-pages-in-beamers-slides
\AtBeginSection[]{
  \begin{frame}
  \vfill
  \centering
  \begin{beamercolorbox}[sep=8pt,center,shadow=true,rounded=true]{title}
    \usebeamerfont{title}\insertsectionhead\par%
  \end{beamercolorbox}
  \vfill
  \end{frame}
}

\newcommand{\QC}{\mathrm{QC}}
\newcommand{\spec}{\mathrm{Spec}}
\newcommand{\Mod}{\mathrm{Mod}}

\title{直交多項式と表現論}
\author{梅崎直也@unaoya}
\date{2019/5/11~数理空間$\tau\acute{o}\pi o\zeta$(トポス)新歓イベント}
\begin{document}

\begin{frame}
\maketitle
\end{frame}

\begin{frame}{三角関数の倍角公式}
\begin{align*}
\sin2\theta&=2\sin\theta\cos\theta\\
\sin3\theta&=\sin2\theta\cos\theta+\cos2\theta\sin\theta\\
&=2\sin\theta\cos^2\theta+(2\cos^2\theta-1)\sin\theta\\
&=\sin\theta(4\cos^2\theta-1)\\
\sin4\theta&=\sin3\theta\cos\theta+\cos3\theta\sin\theta\\
&=\sin\theta(4\cos^2\theta-1)\cos\theta+(4\cos^3\theta-3\cos\theta)\sin\theta\\
&=\sin\theta(8\cos^3\theta-4\cos\theta)
\end{align*}
\end{frame}

\begin{frame}{Chebyshev多項式}
\begin{align*}
\sin\theta&=\sin\theta\times1&U_0(x)&=1\\
\sin2\theta&=\sin\theta\times2\cos\theta&U_1(x)&=2x\\
\sin3\theta&=\sin\theta\times(4\cos^2\theta-1)&U_2(x)&=4x^2-1\\
\sin4\theta&=\sin\theta\times(8\cos^3\theta-4\cos\theta)&U_3(x)&=8x^3-4x
\end{align*}

とすると
\begin{align*}
\sin(n+1)\theta=\sin\theta\times U_n(\cos\theta)
\end{align*}
となる。この$U_n(x)$を第2種Chebyshev多項式という。
\end{frame}

\begin{frame}{漸化式}

加法定理から
\begin{align*}
\sin(n+1)\theta=\cos\theta\sin n\theta+\cos n\theta\sin\theta\\
\sin(n-1)\theta=\cos\theta\sin n\theta-\cos n\theta\sin\theta\\
\sin(n+1)\theta+\sin(n-1)\theta=2\sin n\theta\cos\theta
\end{align*}

ここで$\sin(n+1)\theta=\sin\theta\times U_n(\cos\theta)$を思い出す。
両辺を$\sin\theta$で割って$x=\cos\theta$とすると、
\begin{align*}
U_{n+1}(x)+U_{n-1}(x)=2xU_n(x)
\end{align*}
が成り立つ。
\end{frame}


\begin{frame}{母関数}
漸化式$U_{n}(x)-2xU_{n-1}(x)+U_{n-2}(x)=0, U_0(x)=1, U_1(x)=2x$を使って
\begin{align*}
\sum^\infty_{n=0}U_n(x)t^n=U_0(x)+U_1(x)t+U_2(x)t^2+U_3(x)t^3+\cdots
\end{align*}
を変形する。$t$倍すると$a_nt^n$は$a_nt^{n+1}$となるから、$t^n$の係数は$a_{n-1}$になることに注意すると、
\begin{align*}
-2xt\sum^\infty_{n=0}U_n(x)t^n&=-2xU_0(x)t-2xU_1(x)t^2-2xU_2(x)t^3-2xU_3(x)t^4-\cdots\\
t^2\sum^\infty_{n=0}U_n(x)t^n&=U_0(x)t^2+U_1(x)t^3+U_2(x)t^4+U_3(x)t^5+\cdots
\end{align*}
\begin{align*}
(1-2tx+t^2)\sum^\infty_{n=0}U_n(x)=1
\end{align*}
\end{frame}

\begin{frame}{母関数}
\begin{align*}
\frac{1}{1-2tx+t^2}&=U_0(x)+U_1(x)t+U_2(x)t^2+U_3(x)t^3+\cdots
\end{align*}

逆に左辺を展開することで
\begin{align*}
&1+(2tx-t^2)+(2tx-t^2)^2+(2tx-t^2)^3+\cdots\\
&=1+2tx-t^2+4t^2x^2-4t^3x+t^4+8t^3x^3-12t^4x^2+\cdots\\
&=1+(2x)t+(-1+4x^2)t^2+(-4x+8x^3)t^3+\cdots\\
&=U_0(x)+U_1(x)t+U_2(x)t^2+U_3(x)t^3+\cdots
\end{align*}
と定義することもできる。
\end{frame}


\begin{frame}{昇降演算子}
$\dfrac{1}{1-2tx+t^2}$を$t, x$でそれぞれ微分すると、
\begin{align*}
\frac{2x-2t}{(1-2tx+t^2)^2}=\frac{2(x-t)}{(1-2tx+t^2)^2},~\frac{2t}{(1-2tx+t^2)^2}
\end{align*}
となる。
\begin{align*}
t(x-t)\frac{2(x-t)}{(1-2tx+t^2)^2}&=2t\frac{(x-t)^2}{(1-2tx+t^2)^2}=2t\frac{x^2-2xt+t^2}{(1-2tx+t^2)^2}\\
(1-x^2)\frac{2t}{(1-2tx+t^2)^2}&=2t\frac{1-x^2}{(1-2tx+t^2)^2}
\end{align*}
\begin{align*}
(t(x-t)\frac{d}{dt}+(1-x^2)\frac{d}{dx})\frac{1}{1-2tx+t^2}&=2t\frac{1-2xt+t^2}{(1-2tx+t^2)^2}\\
&=2t\frac{1}{1-2tx+t^2}
\end{align*}
となる。
\end{frame}

\begin{frame}{昇降演算子}
つまり
\begin{align*}
(t(x-t)\frac{d}{dt}+(1-x^2)\frac{d}{dx}-2t)\frac{1}{1-2tx+t^2}=0
\end{align*}
が成り立つ。
\begin{align*}
\frac{1}{1-2tx+t^2}&=U_0(x)+U_1(x)t+U_2(x)t^2+U_3(x)t^3+\cdots\\
\end{align*}
の右辺にも同じ計算をする。
$a_nt^n$を$t$倍と$t$での微分によって$t^n$の係数が$a_{n-1}, (n+1)a_{n+1}$となることに注意すると、
\begin{align*}
nxU_n(x)-(n-1)U_{n-1}(x)+(1-x^2)\frac{d}{dx}U_n(x)-2U_{n-1}(x)=0\\
\{(1-x^2)\frac{d}{dx}+nx\}U_n(x)=(n+1)U_{n-1}
\end{align*}
\end{frame}

\begin{frame}{昇降演算子}
\begin{align*}
\{(1-x^2)\frac{d}{dx}+nx\}U_n(x)&=(n+1)U_{n-1}(x)\\
(n+1)2xU_n(x)&=(n+1)U_{n+1}(x)+(n+1)U_{n-1}(x)
\end{align*}
両辺引くと
\begin{align*}
\{(1-x^2)\frac{d}{dx}-(n+2)x\}U_n(x)&=-(n+1)U_{n+1}(x)\\
\end{align*}

この二つが昇降演算子と呼ばれる。
\begin{align*}
\{(1-x^2)\frac{d}{dx}+nx\}U_n(x)&=(n+1)U_{n-1}\\
\{(1-x^2)\frac{d}{dx}-(n+2)x\}U_n(x)&=-(n+1)U_{n+1}(x)\\
\end{align*}
\end{frame}

\begin{frame}{微分方程式}
昇降演算子を続けて作用させると、
\begin{align*}
\{(1-x^2)\frac{d}{dx}-(n+1)x\}\{(1-x^2)\frac{d}{dx}+nx\}U_n(x)=-n(n+1)U_{n-1}\\
(\{(1-x^2)\frac{d}{dx}-(n+1)x\}\{(1-x^2)\frac{d}{dx}+nx\}+n(n+1)\})U_n(x)=0\\
\end{align*}
左辺を計算すると、
\begin{align*}
&\{(1-x^2)^2\frac{d^2}{dx^2}+(1-x^2)(-2x)\frac{d}{dx}-(n+1)x(1-x^2)\frac{d}{dx}\\
&+nx(1-x^2)\frac{d}{dx}+n(1-x^2)-n(n+1)x^2+n(n+1)\}U_n(x)=0
\end{align*}
\begin{align*}
(1-x^2)\frac{d^2}{dx}U_n(x)-3x\frac{d}{dx}U_n(x)+n(n+2)U_n(x)=0
\end{align*}
がえられる。
\end{frame}

\begin{frame}{微分方程式}
\begin{align*}
&\frac{d^2}{d\theta^2}(\sin\theta U_n(\cos\theta))\\
&=\sin\theta(-U_n(\cos\theta)-3\cos\theta\frac{d}{dx}U_n(\cos\theta)+(1-\cos^2\theta)\frac{d^2}{dx^2}U_n(\cos\theta))
\end{align*}
となるので、
\begin{align*}
(1-x^2)\frac{d^2}{dx}U_n(x)-3x\frac{d}{dx}U_n(x)+n(n+2)U_n(x)=0
\end{align*}
と合わせると、
\begin{align*}
\frac{d^2}{d\theta^2}\sin\theta U_n(\cos\theta)=-(n+1)^2\sin\theta U_n(\cos\theta)
\end{align*}
これは両端が固定された波の方程式(無限の高さの井戸型ポテンシャル)
\end{frame}

\begin{frame}{直交関係}
\begin{align*}
\int_0^\pi\sin (n+1)\theta\sin (m+1)\theta d\theta=\begin{cases}\pi&(n=m)\\0&(n\neq m)\end{cases}
\end{align*}
左辺を
\begin{align*}
\int_0^\pi\frac{\sin (n+1)\theta}{\sin\theta}\frac{\sin (m+1)\theta}{\sin\theta}\sin\theta \sin\theta d\theta
\end{align*}
と変形してから$x=\cos\theta$と置換積分すると
\begin{align*}
\int^1_{-1}U_n(x)U_m(x)\sqrt{1-x^2}dx=\begin{cases}\pi&(n=m)\\0&(n\neq m)\end{cases}
\end{align*}
これを使うと関数をChebyshev多項式で展開でき、数値計算などに応用される。
\end{frame}

\begin{frame}{$SU(2)$の表現}
$SU(2)=\left\{\begin{pmatrix}a&b\\-\bar{b}&\bar{a}\end{pmatrix},\abs{a}^2+\abs{b}^2=1, a,b\in\C\right\}$の$2$変数多項式への作用を
$\begin{pmatrix}a&b\\-\bar{b}&\bar{a}\end{pmatrix}f(x,y)=f(\bar{a}x-by,\bar{b}x+ay)$で定める。
$\begin{pmatrix}e^{i\theta}&0\\0&e^{-i\theta}\end{pmatrix}f(x,y)=f(e^{-i\theta}x,e^{i\theta}y)$を計算すると
\begin{align*}
\begin{pmatrix}e^{i\theta}&0\\0&e^{-i\theta}\end{pmatrix}x=e^{-i\theta}x,
\begin{pmatrix}e^{i\theta}&0\\0&e^{-i\theta}\end{pmatrix}y=e^{i\theta}y,\\
\begin{pmatrix}e^{i\theta}&0\\0&e^{-i\theta}\end{pmatrix}x^2=e^{-2i\theta}x^2,
\begin{pmatrix}e^{i\theta}&0\\0&e^{-i\theta}\end{pmatrix}xy=xy,\\
\begin{pmatrix}e^{i\theta}&0\\0&e^{-i\theta}\end{pmatrix}y^2=e^{2i\theta}y^2
\end{align*}
\end{frame}

\begin{frame}{表現の指標}
\begin{align*}
\begin{pmatrix}e^{i\theta}&0\\0&e^{-i\theta}\end{pmatrix}x^3=e^{-3i\theta}x^3,
\begin{pmatrix}e^{i\theta}&0\\0&e^{-i\theta}\end{pmatrix}x^2y=e^{-i\theta}x^2y,\\
\begin{pmatrix}e^{i\theta}&0\\0&e^{-i\theta}\end{pmatrix}xy^2=e^{i\theta}xy^2,
\begin{pmatrix}e^{i\theta}&0\\0&e^{-i\theta}\end{pmatrix}y^3=e^{3i\theta}y^3
\end{align*}
同じ次数の部分について係数だけ足す。
等比数列の和の公式とEulerの公式で計算すると
\begin{align*}
e^{-i\theta}+e^{i\theta}=\frac{e^{-2i\theta}-e^{2i\theta}}{e^{-i\theta}-e^{i\theta}}=\frac{\sin2\theta}{\sin\theta}\\
e^{-2i\theta}+1+e^{2i\theta}=\frac{e^{-3i\theta}-e^{3i\theta}}{e^{-i\theta}-e^{i\theta}}=\frac{\sin3\theta}{\sin\theta}\\
e^{-3i\theta}+e^{-i\theta}+e^{i\theta}+e^{3i\theta}=\frac{e^{-4i\theta}-e^{4i\theta}}{e^{-i\theta}-e^{i\theta}}=\frac{\sin4\theta}{\sin\theta}\\
\end{align*}
\end{frame}

\begin{frame}{指標の直交性}
群の表現の指標には直交性がある。
例えば$n\neq m$なら
\begin{align*}
\sin(n+m)\frac{2\pi}{N}+\sin2(n+m)\frac{2\pi}{N}+\cdots+\sin(N-1)(n+m)\frac{2\pi}{N}=0
\end{align*}
が成り立つということ。

同じように$n\neq m$のとき、
\begin{align*}
\int_0^\pi\frac{\sin (n+1)\theta}{\sin\theta}\frac{\sin (m+1)\theta}{\sin\theta}\sin\theta \sin\theta d\theta&=0\\
\int_{-1}^0U_n(x)U_m(x)\sqrt{1-x^2}dx&=0
\end{align*}
が成り立つ。
\end{frame}

\begin{frame}{直交多項式}
上で見たような直交関係、漸化式、微分方程式などの性質を満たす多項式の系列がいくつかある。

例えば量子力学に出てくるものとして
\begin{itemize}
\item Hermite多項式は調和振動子
\item Legendre多項式は角運動量の量子化
\item Laguerre多項式は水素原子の動径方程式
\end{itemize}
がある。

直交多項式の性質を表現論で理解できる(と思っているのでこれから勉強する)。
\end{frame}

\end{document}