圏論の歩き方

タングルの不変量とリボン圏

リボン圏とその対象を与えると、タングルの圏からの関手が得られる。
リボン圏の例としては、単純リー代数gの量子群U_q(g)の有限次元左加群のなす圏がある。
つまり、U_q(g)の有限次元表現を与えると、そこからタングルの不変量が得られることになる。

この構成は、タングルの圏の普遍性(リボン圏としての自由性)を用いている。
Cをリボン圏とし、Vをその対象とする。
このとき、リボン圏の構造を保つ関手F_V:T \to CでF_V(↓)=Vであるものがただ一つ存在する。

リボン圏というのは特別なテンソル圏で、タングルの圏はリボン圏である。
これはR行列の満たす関係式を抽象化したような関係式を持つもの?

