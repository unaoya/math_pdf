\documentclass[dvipdfmx]{beamer}
\usetheme{metropolis}

\input{../tex/theorems}
\input{../tex/symbols}
\usepackage{bxdpx-beamer}
\title{関数等式と双対性}
\author{梅崎直也@unaoya}
\date{ロマンティックゼータナイト}
\begin{document}

\begin{frame}
\maketitle
\end{frame}

\begin{frame}{Riemann $\zeta$}
  Euler積表示
  \begin{align*}
    \zeta(s)=\sum^{\infty}_{n=1}n^{-s}=\prod_p(1-p^{-s})^{-1}
  \end{align*}
  関数等式
  \begin{align*}
    \pi^{-s/2}\Gamma(\frac{s}{2})\zeta(s)=\pi^{-(1-s)/2}\Gamma(\frac{1-s}{2})\zeta(1-s)
  \end{align*}

  関数等式の応用:  ゼロ点の情報
\end{frame}

\begin{frame}{Poisson和公式}
  証明:Poisson和公式、$\theta$関数、Mellin変換
\end{frame}

\begin{frame}{Dirichlet $L$}
  Dirhchlet指標$\chi:\Z\to\C$、$\chi(nm)=\chi(n)\chi(m), \exists N, (n,N)\neq1, \chi(n)=0$

  eg:Legendre記号
  
  \begin{align*}
    L(\chi,s)=\sum^{\infty}_{n=1}\chi(n)n^{-s}=\prod_p(1-\chi(p)p^{-s})^{-1}
  \end{align*}

  全ての$n$で$\chi(n)=1$とするとRiemann $\zeta$
\end{frame}

\begin{frame}{関数等式}
  \begin{align*}
    \Lambda(s,\chi)=(\frac{\pi}{m})^{-s/2}\Gamma(\frac{s+\delta}{2})L(s,\chi)
  \end{align*}
  として、
  \begin{align*}
    \Lambda(1-s,\overline{\chi})=\frac{i^\delta m^{1/2}}{\tau(\chi)}\Lambda(s,\chi)
  \end{align*}
  (補正項の存在、root number, $\varepsilon$因子)
\end{frame}

\begin{frame}{Dedekind $\zeta$}
  代数体$K$
  \begin{align*}
    \zeta_K(s)=\sum_{\mathfrak{a}}(N\mathfrak{a})^{-s}
    =\prod_{\mathfrak{p}}
  \end{align*}

  幾何的な解釈
  $K=\Q[x]/f(x)$とした時の点の数

  関数等式
  \begin{align*}
    \Lambda_K(s)=(\frac{\abs{d_K}}{4^{r_2}\pi^n})^{s/2}\Gamma^{r_1}(s/2)\Gamma^{r_2}(s)\zeta_K(x)
  \end{align*}
  とすると、
  \begin{align*}
    \Lambda_K(s)=\Lambda_K(1-s)
  \end{align*}
  
  $K=\Q$がRiemann $\zeta$

  Dirichlet $L$との関係
\end{frame}

\begin{frame}{Artin $L$}
  Galois表現
  $\sigma:\Gal(L/K) \to GL_n(\C)$

  $n=1, \sigma=1$の時がDedekind $\zeta$
  $n=1$のときDirichlet(類体論)
\end{frame}

\begin{frame}{Hecke $L$}
  Hecke指標

  位数が有限でないものも考える

  係数は?$\C$ vs $\Q_\ell$

  なぜ?
\end{frame}

\begin{frame}{合同$\zeta$関数}
  有限体上の多様体$X/\F$

  およそ多項式、
  解の個数を数えることで、

  \begin{align*}
    Z=\exp(\sum\frac{\abs{X(\F_{q^m})t^m}}{m})
  \end{align*}  

  点ごとの寄与で表したものとの比較

  曲線の場合、関数体と代数体の類似

  関数等式
\end{frame}

\begin{frame}{コホモロジー}
  跡公式によりコホモロジーで解釈できる。
  コホモロジーの双対性が関数等式に
\end{frame}

\begin{frame}{Hasse-Weil $\zeta$関数}
  代数体上の多様体

  よい素点と悪い素点
\end{frame}

\begin{frame}{関数等式}
  予想。

  証明されたケース、$\Q$上の楕円曲線$E$の時。
  保型形式$f_E$であって$L$関数が一致するものを作る(難しい。Wilesなど)
  $f_E$の関数等式を証明する。
\end{frame}

\begin{frame}{保型形式}
  保型形式の$L$関数
\end{frame}

\begin{frame}{Fourier変換}
  関数等式、テータ、Tate論文
\end{frame}

\begin{frame}{Langlands対応}
  保型表現
\end{frame}

\begin{frame}{気になること}
  $\varepsilon$因子
  Fesenko
\end{frame}

\end{document}
