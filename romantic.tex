\documentclass[dvipdfmx]{beamer}
\usetheme{metropolis}

\input{../tex/theorems}
\input{../tex/symbols}
\usepackage{bxdpx-beamer}
\usepackage{comment}

\title{関数等式と双対性}
\author{梅崎直也@unaoya}
\date{ロマンティックゼータナイト}
\begin{document}

\begin{frame}
\maketitle
\end{frame}

\begin{frame}{Riemann $\zeta$}
  Euler積表示
  \begin{align*}
    \zeta(s)=\sum^{\infty}_{n=1}n^{-s}=\prod_p(1-p^{-s})^{-1}
  \end{align*}
  関数等式
  \begin{align*}
    \pi^{-s/2}\Gamma(\frac{s}{2})\zeta(s)=\pi^{-(1-s)/2}\Gamma(\frac{1-s}{2})\zeta(1-s)
  \end{align*}
  Poisson和公式を用いて示す。
\end{frame}

\begin{frame}{Dirichlet $L$}
  導手$N$のDirhchlet指標$\chi:\Z\to\C$
  $\chi(nm)=\chi(n)\chi(m)$、$n$が$N$と互いに素なら$\chi(n)=0$

  Legendre記号など。
  
  \begin{align*}
    L(\chi,s)=\sum^{\infty}_{n=1}\chi(n)n^{-s}=\prod_p(1-\chi(p)p^{-s})^{-1}
  \end{align*}
  全ての$n$で$\chi(n)=1$とするとRiemann $\zeta$
  \begin{align*}
    L(1,s)=\sum^{\infty}_{n=1}n^{-s}=\prod_p(1-p^{-s})^{-1}
  \end{align*}
\end{frame}

\begin{frame}{関数等式}
  \begin{align*}
    \hat{L}(s,\chi)=N^{s/2}\Gamma(s,\chi)L(s,\chi)
  \end{align*}
  として、
  \begin{align*}
    \hat{L}(1-s,\overline{\chi})=W(\chi)\hat{L}(s,\chi)
  \end{align*}
  補正項$W(\chi)$が存在する。
  Gauss和を用いて記述できる。

  Fourier変換、Poisson和公式を用いて示す。
\end{frame}

\begin{frame}{Dedekind $\zeta$}
  代数体$K$に対して、
  \begin{align*}
    \zeta_K(s)=\sum_{\mathfrak{a}}(N\mathfrak{a})^{-s}=\prod_\mathfrak{p}(1-(N\mathfrak{p})^{-s})^{-1}
  \end{align*}

  %幾何的な解釈
  %$K=\Q[x]/f(x)$とした時の点の数

  $K=\Q$の時、$\zeta_K(s)=\zeta(s)$
\end{frame}

\begin{frame}{関数等式}
  $D_K$を$K$の判別式とする。$D_\Q=1$である。
  \begin{align*}
    \hat{\zeta}_K(s)=\abs{D_K}^{s/2}\Gamma_K(s)\zeta_K(x)
  \end{align*}
  とすると、
  \begin{align*}
    \hat{\zeta}_K(s)=\hat{\zeta}_K(1-s)
  \end{align*}
\end{frame}

\begin{frame}{Hecke $L$}
  同種$f$のHecke指標$\chi:\A_K \to \C^\times$

  \begin{align*}
    L(s,\chi)=\prod_p(1-\chi(\pi_p)N(p)^{-s})^{-1}
  \end{align*}
  (悪い素点では修正する。)

  %位数が有限でないものも考える
  %係数は?$\C$ vs $\Q_\ell$
\end{frame}

\begin{frame}{関数等式}    
  \begin{align*}
    \hat{L}=\abs{D_K}^{s/2}f_\chi^{s/2}\Gamma(s,\chi)L(s,\chi)
  \end{align*}
  とすると、関数等式
  \begin{align*}
    \hat{L}(s,\chi)=W(\chi)\hat{L}(1-s,\overline{\chi})
  \end{align*}
  を満たす。  アデールでの積分、Fourier変換を用いて示す。
\end{frame}

\begin{comment}
  
\begin{frame}{Artin $L$}
  Galois表現$\sigma:\Gal(L/K) \to GL_n(\C)$に対し
  \begin{align*}
    L(s,\sigma)=\prod_p\det(1-N(p)^{-s}\sigma(\Frob_p))
  \end{align*}
  (悪い素点では修正する。)

  $\sigma=1$の時が$K$のDedekind $\zeta$ $L(s,1)=\zeta_K(s)$である。
  $n=1$のときDirichlet $L$($K=\Q$なら円分体$\Gal(\Q(\zeta_m)/\Q)\cong(\Z/m\Z^\times$、類体論)
\end{frame}

\begin{frame}{関数等式}
  未解決?
\end{frame}

\end{comment}

\begin{frame}{合同$\zeta$関数}
  有限体上の多様体$X/\F_q$はだいたい多項式$f=0$で定まる図形。
  これの解の個数$\abs{X(\F_{q^m})}$を数えることで、
  \begin{align*}
    Z(X,t)=\exp\left(\sum_{m=1}^\infty\frac{\abs{X(\F_{q^m})t^m}}{m}\right)
  \end{align*}
  を定める。
  \begin{align*}
    \zeta_X(s)=\prod_{x\in X}(1-(Nx)^{-s})^{-1}=Z(X,q^{-s})
  \end{align*}
  と表示できる。
\end{frame}

\begin{frame}{関数等式}
  $X$が$n$次元の時、整数係数の多項式$P_i(t)$が存在して
  \begin{align*}
    Z(t)=\frac{P_1(t)P_3(t)\cdots P_{2n-1}(t)}{P_0(t)P_2(t)\cdots P_{2n}(t)}
  \end{align*}
  となる。
  
  関数等式
  \begin{align*}
    Z(\frac{1}{q^nt})=\pm q^{nE/2}t^EZ(t)
  \end{align*}
  が成立。
  
  $X$のコホモロジー$H^i(X)$のLefschetz跡公式により、$Fr$の作用の固有多項式$P_i(t)$を用いて$Z(X,t)$を記述できる。
  コホモロジーのPoincare双対性から関数等式を示す。
\end{frame}

\begin{frame}{Hasse-Weil $\zeta$関数}
  代数体上の多様体$X$に対し、その$i$次部分$H^i(X)$に対して

  \begin{align*}
    L(s,H^i(X))=\prod_pP_i(X_p,q^{-s})
  \end{align*}
  (悪い素点では修正する。)

\end{frame}

\begin{frame}{関数等式}
  \begin{align*}
    \hat{L}(s,H^i(X))=N^{s/2}\Gamma(H^i(X),s)L(s,H^i(X))
  \end{align*}
  
  関数等式(予想)
  \begin{align*}
    \hat{L}(s,H^i(X))=\pm\hat{L}(i+1-s,H^i(X))
  \end{align*}
  
  $\Q$上の楕円曲線$E$ではWilesなどにより証明された。
  
  保型形式$f_E$であって$L$関数が一致するものを作る。
  保型形式$f_E$の$L$関数の関数等式はHeckeなどによりFourier変換などを用いて証明されていた。
\end{frame}

\begin{frame}{$\varepsilon$因子}
  関数等式に現れる補正項について。

  $X$が有限体上の多様体、$\mathcal{F}$を$\ell$進層とする。
  関数等式
  \begin{align*}
    L(X,\mathcal{F},t)=\varepsilon(X,\mathcal{F})t^{-\chi(\overline{X},\mathcal{F})}L(X,D(\mathcal{F}),t^{-1})
  \end{align*}
  の$\varepsilon(X,\mathcal{F})$を特性類を用いて幾何的に解釈

  \begin{thm}[U.-Yang-Zhao]
    \begin{align*}
      \det\rho(-cc_X\mathcal{F})=\frac{\varepsilon(X,\mathcal{F}\otimes\rho)}{\varepsilon(X,\mathcal{F})^{\dim\rho}}
    \end{align*}    
  \end{thm}  
\end{frame}

\end{document}
