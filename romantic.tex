\documentclass[dvipdfmx]{beamer}
\usetheme{metropolis}

\input{../tex/theorems}
\input{../tex/symbols}
\usepackage{bxdpx-beamer}
\usepackage{comment}

\title{関数等式と双対性}
\author{梅崎直也@unaoya}
\date{2019年10月20日ロマンティック数学ナイトプライム@ゼータ}
\begin{document}

\begin{frame}
\maketitle
\end{frame}

\begin{frame}{Riemann $\zeta$}
  \begin{align*}
    \zeta(s)=\sum^{\infty}_{n=1}n^{-s}=\prod_p(1-p^{-s})^{-1}
  \end{align*}

  \begin{align*}
    \hat{\zeta}(s)=\pi^{-s/2}\Gamma(\dfrac{s}{2})\zeta(s)
  \end{align*}
  とおくと、関数等式
  \begin{align*}
    \hat{\zeta}(s)=\hat{\zeta}(1-s)
  \end{align*}
  が成立。Fourier変換(Poisson和公式)を用いて示せる。
\end{frame}

\begin{frame}{Dirichlet $L$}
  導手$f$のDirhchlet指標$\chi:\Z\to\C$
  
  $\chi(nm)=\chi(n)\chi(m)$、$n$が$f$と互いに素なら$\chi(n)=0$。

  Legendre記号などが例。
  
  \begin{align*}
    L(\chi,s)=\sum^{\infty}_{n=1}\chi(n)n^{-s}=\prod_p(1-\chi(p)p^{-s})^{-1}
  \end{align*}
  全ての$n$で$\chi(n)=1$とするとRiemann $\zeta$
  \begin{align*}
    L(1,s)=\sum^{\infty}_{n=1}n^{-s}=\prod_p(1-p^{-s})^{-1}
  \end{align*}
\end{frame}

\begin{frame}{関数等式}
  \begin{align*}
    \hat{L}(\chi,s)=f_\chi^{s/2}\Gamma(\chi,s)L(s,\chi)
  \end{align*}
  とする。$f_1=1, \Gamma(s,1)=\pi^{-s/2}\Gamma(s)$である。
  \begin{align*}
    \hat{L}(\overline{\chi},1-s)=W(\chi)\hat{L}(\chi,s)
  \end{align*}
  補正項$W(\chi)$が存在する。Fourier変換(Poisson和公式)を用いて示せる。
\end{frame}

\begin{frame}{Dedekind $\zeta$}
  代数体$K$に対して、
  \begin{align*}
    \zeta_K(s)=\sum_{\mathfrak{a}}(N_{K/\Q}\mathfrak{a})^{-s}=\prod_\mathfrak{p}(1-(N_{K/\Q}\mathfrak{p})^{-s})^{-1}
  \end{align*}

  %幾何的な解釈
  %$K=\Q[x]/f(x)$とした時の点の数

  $K=\Q$の時、$N_{\Q/\Q}(p)=p$なので$\zeta_K(s)=\zeta(s)$となる。
\end{frame}

\begin{frame}{関数等式}
  \begin{align*}
    \hat{\zeta}_K(s)=\abs{D_K}^{s/2}\Gamma_K(s)\zeta_K(x)
  \end{align*}
  とする。$D_K$は$K$の判別式で$D_\Q=1$。
  $\Gamma_\Q(s)=\pi^{-s/2}\Gamma(\dfrac{s}{2})$である。
  \begin{align*}
    \hat{\zeta}_K(s)=\hat{\zeta}_K(1-s)
  \end{align*}
\end{frame}

\begin{frame}{Hecke $L$}
  導手$f$のHecke指標$\chi:\A_K \to \C^\times$。
  これの特別な場合がDirichlet指標。

  \begin{align*}
    L(\chi,s)=\prod_p(1-\chi(\pi_p)N(p)^{-s})^{-1}
  \end{align*}
      (悪い素点では修正する。)

  %位数が有限でないものも考える
  %係数は?$\C$ vs $\Q_\ell$
\end{frame}

\begin{frame}{関数等式}    
  \begin{align*}
    \hat{L}(\chi,s)=\abs{D_K}^{s/2}f_\chi^{s/2}\Gamma(\chi,s)L(\chi,s)
  \end{align*}
  とすると、関数等式
  \begin{align*}
    \hat{L}(\chi,s)=W(\chi)\hat{L}(\overline{\chi},1-s)
  \end{align*}
  を満たす。
  アデール上のFourier変換を用いて示す。
\end{frame}

\begin{frame}{合同$\zeta$}
  有限体上の多様体$X/\F_q$はだいたい多項式$f=0$で定まる図形。
  これの解の個数$\abs{X(\F_{q^m})}$を数えることで、
  \begin{align*}
    Z(X,t)=\exp\left(\sum_{m=1}^\infty\frac{\abs{X(\F_{q^m})}t^m}{m}\right)
  \end{align*}
  を定める。
  \begin{align*}
    \frac{d}{dt}\log(Z(X,t))=\sum_m\abs{X(\F_{q^m})}t^m
  \end{align*}
  である。
  \begin{align*}
    \zeta_X(s)=\prod_{x\in X}(1-(Nx)^{-s})^{-1}=Z(X,q^{-s})
  \end{align*}
  と表示できる。
\end{frame}

\begin{frame}{関数等式}
  $X$のコホモロジー$H^i(X)$のLefschetz跡公式により、Frobenius作用の固有多項式を用いて$Z(X,t)$を記述できる。
  \begin{align*}
    Z(X,t)=\frac{\det(1-\Frob t\mid H^1(X))\cdots \det(1-\Frob t\mid H^{2n-1}(X))}
    {\det(1-\Frob t\mid H^0(X))\cdots \det(1-\Frob t\mid H^{2n}(X))}
  \end{align*}
  
  関数等式
  \begin{align*}
    Z(X,\frac{1}{q^nt})=\pm q^{n\chi(X)/2}t^{\chi(X)}Z(X,t)\\
    \zeta_X(n-s)=\pm q^{n\chi(X)/2-\chi(X)s}\zeta_X(s)
  \end{align*}
  が成立。コホモロジーのPoincare双対性。
\end{frame}

\begin{frame}{Hasse-Weil $\zeta$}
  代数体$K$上の多様体$X$に対し、その$i$次部分$H^i(X)$に対して
  \begin{align*}
    L(H^i(X),s)=\prod_p\det(1-\Frob_pp^{-s}\mid H^i(X))^{-1}
  \end{align*}
  (悪い素点では修正する。)

  \begin{align*}
    \hat{L}(H^i(X),s)=N^{s/2}\Gamma(H^i(X),s)L(H^i(X),s)
  \end{align*}
 
\end{frame}

\begin{frame}{関数等式}
  関数等式(予想)
  \begin{align*}
    \hat{L}(H^i(X),s)=\pm\hat{L}(H^i(X),i+1-s)
  \end{align*}
  
  $\Q$上の楕円曲線$E$ではWilesなどにより証明された。
  
  保型形式$f_E$であって$L$関数が一致するものを作る。
  保型形式$f_E$の$L$関数の関数等式はHeckeなどによりFourier変換などを用いて証明されていた。
\end{frame}

\begin{frame}{$\ell$進層の$L$}
  $X$が有限体上の多様体、$\mathcal{F}$を$\ell$進層とする。
  \begin{align*}
    &L(X,\mathcal{F},t)=\prod_x\det(1-t^{\deg(x)}F_x,\mathcal{F}_{\bar{x}})^{-1}\\
    &=\frac{\det(1-\Frob t\mid H^1(X,\mathcal{F}))\cdots \det(1-\Frob t\mid H^{2n-1}(X,\mathcal{F}))}
    {\det(1-\Frob t\mid H^0(X,\mathcal{F}))\cdots \det(1-\Frob t\mid H^{2n}(X,\mathcal{F}))}
  \end{align*}

  $\mathcal{F}$が定数層$\Lambda$のとき、合同ゼータ。

  曲線$X$上の族$f:Y\to X$に対して、$\mathcal{F}=H^i(Y_x)$も$\ell$進層の例。

  関数等式
  \begin{align*}
    L(X,\mathcal{F},t)=\varepsilon(X,\mathcal{F})t^{-\chi(\overline{X},\mathcal{F})}L(X,D(\mathcal{F}),t^{-1})
  \end{align*}
\end{frame}


\begin{frame}{分岐と$\varepsilon$因子}
  悪い素点での様子、判別式、導手、関数等式に現れる補正項などの情報が重要。
  (不変量としても強力。)

  分岐の幾何的な不変量として特性サイクルというものがある。
  特性サイクルは元々は微分方程式($D$加群)の理論で考えられたもので、分岐の様子を記述する。

  関数等式の$\varepsilon(X,\mathcal{F})$と特性サイクルの関係
  \begin{thm}[U.-Yang-Zhao]
    \begin{align*}
      \det\rho(-cc_X\mathcal{F})=\frac{\varepsilon(X,\mathcal{F}\otimes\rho)}{\varepsilon(X,\mathcal{F})^{\dim\rho}}
    \end{align*}    
  \end{thm}
\end{frame}

\end{document}
