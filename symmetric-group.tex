誘導指標の計算

S_4なら、4の分割が1^4, 1^22, 13, 2^2, 4なので既約表現が5個。自明表現と符合表現のみが1次元表現。4!-1-1=22なので、1より大きい平方数3つの和で表すには4+4+9で残りの既約表現は2次元2つと3次元1つ。S_3から自明表現を誘導すると1次元成分は自明表現1つで、残りが3次元既約表現。S_2^2の1(x)1を誘導すると、1次元成分が自明表現1つで残り5次元が2次元表現と3次元表現。S_2^2のsgn(x)sgnを誘導すると、1次元成分がsgn1つで残り2次元と3次元の和。ここで現れる2つの2次元表現が同型か?
Hom(Ind1(x)1, Indsgn(x)sgn)を計算すると、S_2^2の表現としてResInd(sgn(x)sgn)は
指標を計算すると、誘導指標の計算から
