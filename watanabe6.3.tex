\documentclass[uplatex, slide, papersize]{jsarticle}
\usepackage[dvipdfmx]{graphicx, color}
\RequirePackage{amsmath,amssymb,amsthm, amscd, comment, multicol}
\input{../tex/theorems}
\input{../tex/symbols}
\title{Watanebe, 6.3}
\date{\today}
\author{@unaoya}
\begin{document}
\maketitle

\section*{統計的学習}
\begin{enumerate}
\item 真の分布$q(x)$またはそこから生成されるサンプルを予測したい
\item モデル$p(x\vert w)$とパラメータ空間$W$を設定
\item 与えられたサンプルからパラメータ空間上の測度および予測分布$\hat{p}(x)$を決定
\end{enumerate}

\begin{center}
\Large{モデルを評価したい}
\end{center}

\section*{Bayes quartet}
四つのモデルの評価基準

\begin{table}[!h]
\begin{tabular}{ccc}
&Bayes推測&Gibbs推測\\
分布&$B_g$&$G_g$\\
サンプル&$B_t$&$G_t$
\end{tabular}
\end{table}

これらはサンプル$D_n$に依存した確率変数

\section*{6.3の目標}
\begin{itemize}
\item サンプル数$n\to\infty$における漸近挙動
\item サンプルについての期待値
\item これらの4つの間の関係
\end{itemize}
について調べる。

\newpage
\vspace*{\stretch{1}}
\begin{center}
\huge{今日は$G_g$について}
\end{center}
\vspace{\stretch{2}}
\newpage
\subsection*{Gibbs推測}
事後分布に従ってパラメータ$\hat{w}$をサンプリングし、$\hat{p}(x)=p(x\vert \hat{w})$を予測分布とする。
\subsection*{汎化誤差$G_g$}
$q(x)$と$\hat{p}(x)$のKL divergenceを$w$について事後分布$p(w\vert \textcolor{red}{D_n})$で積分したもの
\[
G_g=\int_WK(w)p(w\vert \textcolor{red}{D_n} )dw
\]

\newpage
\subsection*{問題}
$n\to\infty$で$nG_g$がどのような確率変数に収束するか?

\subsection*{主要項}
\[
G_g(\epsilon)=\frac{\int_{K(w)\leq\epsilon}K(w)p(w\vert D_n)dw}{\int_{K(w)\leq\epsilon}p(w\vert D_n)dw}
\]

\begin{lem}[Lemma 6.3]
$nG_g-nG_g(\epsilon)$は$0$に確率収束する。
\end{lem}

\section*{主要項の評価}
\begin{align*}
G_g(\epsilon)&=\frac{\int_{K(w)\leq\epsilon}K(w)p(w\vert D_n)dw}{\int_{K(w)\leq\epsilon}p(w\vert D_n)dw}\\
&=E[K(w)\vert_{K(w)\leq\epsilon}]
\end{align*}
の評価をしたいが直接は難しい。
\begin{center}
\large{特異点解消を使う}
\end{center}

\section*{標準形}
$f(x,g(u))=\log(\frac{q(x)}{p(x\vert g(u))})=a(x,u)u^k$とし
\begin{align*}
K(g(u))&=u^{2k}\\
K_n(g(u))&=u^{2k}-\frac{1}{\sqrt{n}}u^k\xi_n(u)\\
\xi_n(u)&=\frac{1}{\sqrt{n}}\sum_{i=1}^n\{a(X_i,u)-E_X[a(X,u)]\}
\end{align*}
$\xi_n(u)$はサンプル$D_n$に依存した確率過程。
$\xi_n$はGauss過程$\xi$に法則収束する。

\begin{lem}[6.51]
\[
G_g^*(\xi_n)=E_{y,t}[t\vert\xi_n]
\]
と定義すると
\[
nG_g(\epsilon)-G_g^*(\xi_n)\to^P0
\]
\end{lem}
\newpage
\subsection*{特異点解消$M$上での積分$E_{y,t}$と$E_u$}
\begin{itemize}
\item $\xi(u)$: $M$上の$C^1$級関数(サンプルの確率過程)
\item$f(u)$: $M$上の関数($K(w)$に対し$f(u)= u^{2k}$)
\item $0\leq\sigma\leq1$
\end{itemize}
\[
E^\sigma_u[f(u)\vert\xi]=\frac{\displaystyle\sum_{\alpha\in A}\int_{[0,b]^d}f(u)Z(u,\xi)du}{\displaystyle\sum_{\alpha\in A}\int_{[0,b]^d}Z(u,\xi)du}
\]
$A$は座標近傍の(有限)集合。

$Z(u,\xi)$は
\[
u^h\phi^*(u)\exp(-\beta nu^{2k}+\beta\sqrt{n}u^k\xi(u)-\sigma u^ka(X,u))
\]
事後分布$p(w\vert D_n)$との関係。
\begin{itemize}
\item $u^h\phi^*(u)$が事前分布$\phi(w)$に対応。
\item $\sigma=0$として
\begin{align*}
Z_n^0p(w\vert D_n)&=\exp(-n\beta K_n(w))\\
&=\exp(-\beta nu^{2k}+\beta\sqrt{n}u^k\xi_n(u))
\end{align*} 
\end{itemize}
\begin{lem}[6.41]
\[
G_g(\epsilon)=E^0_u[u^{2k}\vert\xi_n]
\]
\end{lem}

$u^{2k}=K(g(u))$であった。
\newpage
\subsection*{本質的部分}
座標$u=(x,y)$と本質的部分$A^*\subset A$($K(w)$の特異点解消から決まる)

$E_{y,t}[f(y,t)\vert\xi]= $
\begin{align*}
\frac{\displaystyle\sum_{\alpha\in A^*}\int dt\int_{[0,b]^{d-m}}f(y, t)Z_0(y, t,\xi)du}{\displaystyle\sum_{\alpha\in A^*}\int dt\int_{[0,b]^{d-m}}Z_0(y, t,\xi)du}
\end{align*}

\[
Z_0(y,t,\xi)=\gamma_by^\mu t^{\lambda-1}\exp(-\beta t+\beta\sqrt{t}\xi_0(y))\phi^*_0(y)
\]

\begin{lem}[Lemma 6.6, $p=1, f=1, \xi=\xi_n$]
\[
\abs{E^0_u[nu^{2k}\vert\xi_n]-E_{y,t}[t\vert\xi_n]}\leq\frac{D(\xi_n,1,\phi^*)}{\log n}
\]
\end{lem}
これの証明に4章での分配関数の計算を用いる。

\newpage
\subsection*{収束先の構成}
\begin{dfn}[6.46]
$M$上の関数$\psi(u)$に対し
\[
G_g^*(\psi)=E_{y,t}[t\vert\psi]
\]
\end{dfn}

これを使ってさっきの補題を書き直すと
\begin{lem}
\[
\abs{nG_g(\epsilon)-G_g^*(\xi_n)}\leq\frac{D(\xi_n,1,\phi^*)}{\log n}
\]
\end{lem}

\subsection*{結論}
$\xi_n$が$\xi$に法則収束することから
\begin{itemize}
\item 補題4を用いて$nG_g(\epsilon)-G_g^*(\xi_n)\to 0$
\item $G_g^*(\xi_n)-G^*_g(\xi)\to 0$
\end{itemize}
が言える。

全て合わせて$nG_g-G^*_g(\xi)\to0$が証明できた。
$G^*_g(\xi)$は$\xi$に依存した確率変数である。

\newpage
\section*{4章の復習}
\subsection*{ゼータ関数}
開集合$U\subset \R^d$上の非負解析的関数$K(w)$とコンパクト台$C^\infty$関数$\phi(w)$に対し、
\[
\zeta(z)=\int K(w)^z\phi(w)dw
\]
と定義する。
これの極の位置とその位数はどのような情報を持つか?
\newpage
\subsection*{状態密度関数}
$\zeta(z)$の逆Mellin変換は状態密度関数
\[
v(t)=\int\delta(t-K(w))\phi(w)dw
\]
である。Mellin変換の理論により、これの発散のオーダーが$\zeta(z)$の極の位置と対応。
\newpage
\subsection*{分配関数}
状態密度関数$v(t)$のLaplace変換
\[
Z(n)=\int\exp(-nK(w))\phi(w)dw
\]
を分配関数という。
これが6章前半で調べていたもの。
\newpage
\subsection*{特異学習理論}
Remark 4.4にあるように
\[
Z=\int\exp(-n\beta K(w)+\beta\sqrt{nK(w)}\xi(w))\phi(w)dw
\]
の$n\to\infty$での挙動を調べたい。
\newpage

$K$についての特異点解消により、normal crossingの場合の積分$Z(n,\xi,\phi)$を用いて
\[
Z=\sum_\alpha Z(n,\xi\circ g_\alpha,\phi\circ g_\alpha\abs{g'_\alpha})
\]
と書けるので、$Z(n,\xi,\phi)$について調べるのが4.4の目標。

\newpage
\begin{align*}
&Z^p(n,\xi,\phi)=\\
&\int_{[0,b]^r}dx\int_{[0,b]^s}dyK(X,y)^px^hy^{h'}\phi(x,y)\\ &\exp(-n\beta K(x,y)^2+\sqrt{n}\beta K(x,y)\xi(x,y))
\end{align*}
と定義する。
\newpage

さらにこれで$\xi=0, \phi=1$と置いたものを
\begin{align*}
Z^p(n)=&\int_{[0,b]^r}dx\int_{[0,r]^s}dy\\&K(x,y)^px^h,y^{h'}\exp(-n\beta K(x,y)^2)
\end{align*}
と書くことにする。

\newpage

\begin{thm}[Theorem 4.7]
\[
\frac{h_i+1}{2k_i}=\lambda
\]
が一定で
\[
\frac{h_j'+1}{2k_j'}>\lambda
\]
とする。
$K(x,y)=x^ky^{k'}$のときに、ある$a_1,a_2>0$が存在して任意の$n$に対して
\[
a_1\frac{(\log n)^{r-1}}{n^{\lambda+p}}\leq Z^p(n)\leq a_2\frac{(\log n)^{r-1}}{n^{\lambda+p}}
\]
\end{thm}

平均誤差関数$K(w)$と事前分布$\phi(w)$に対してゼータ関数
\[
\zeta(z)=\int K(w)^z\phi(w)dw
\]
を定義する。これの極の情報から$K$の?特異点の実対数閾値がもとまる。
これが自由エネルギーおよび汎化損失の理論値が明らかになる。

\newpage

\subsection{疑問点}
正則な場合の計算をやる。特にこの時正則性をどこで使うか。
正則な場合、サンプルが多いことが仮定される?
\end{document}