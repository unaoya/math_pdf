\documentclass{jsarticle}
\RequirePackage{amsmath,amssymb,amsthm, amscd, comment, multicol}
\usepackage[all]{xy}
\input{../tex/theorems}
\input{../tex/symbols}
\newcommand{\fC}{\mathfrak{C}}
\usepackage[dvipdfmx]{graphicx}
\title{BSD予想について}
\author{@unaoya}
\date{\today}
\begin{document}
\maketitle
参考、Birch and Swinnerton-Dyer, Notes on elliptic curves II.

話の流れ
\begin{enumerate}
\item $E$を$y^2=x^3+ax+b$としておく。
\item $E(\Q)$をいくつかの例で考察する。
\item 直線との交点を利用して、有理点を作っていく(群構造)
\item 有限で終わる場合、無限に続く場合、どちらもある。
\item $E(\F_p)$を計算する
\item $E(\F_p)$の$p \to \infty$での振る舞いと有理点の個数の関係
\item $L$関数を導入する
\item 虚数乗法?
\end{enumerate}


BirchとSwinnerton-Dyerの数値計算の話。
楕円曲線の$\mod p$での有理点を数えた。

$E$を楕円曲線とし、$E(\Q)$の$\rank$が既知とする。
この時、$N_p=\abs{E(\F_p)}$とし、
\begin{align*}
\prod_{p\leq x}\frac{N_p}{p}
\end{align*}
の$x \to \infty$での振る舞いを調べた。

Mordellの定理。$E(\Q)$が有限生成であること。($\Q, \R$は群として有限生成ではない。)
群構造の説明、有限ランク?
有限か否か簡単に分かる例は?
$y^2=x^3-4x, y^2=x^3+x$は有限、$y^2=x^3-4$は?CMの場合は?

Mazurの定理($E(\Q)_{tors}$のboundがある(cyclicなら$12$以下)
位数$12$以上の点を見つけたら無限

$y^2=x^3-x$は$E(\Q)=\{(0,0), (1,0), (-1,0), \infty\}$であることが、
Fermatにより無限降下法で示された。

Hasseの定理。$\abs{E(\F_p)}$の評価。

$y^2=x^3-1$は$\mod 6$で$a_p=0$が判定できる。(CMもつ?)

$L$関数の定義。
上の$\prod_{p\leq x}\frac{N_p}{p}$が「形式的に」$L$関数と関係すること。

\begin{align*}
L_\Gamma(s)=\prod_p\frac{1}{1+(N_p-p-1)p^{-s}+p^{1-2s}}\\
\end{align*}
とする。(実は有限個の$p$で修正が必要。)

形式的に
\begin{align*}
L_\Gamma(1)&=\prod_p\frac{1}{1+(N_p-p-1)p^{-1}+p^{-1}}\\
&=\prod_p\frac{1}{1+N_pp^{-1}-1}\\
&=\prod_p\frac{p}{N_p}
\end{align*}


虚数乗法を持つ場合にはより正確に計算できる。
\begin{align*}
\Gamma:y^2=x^3-Dx
\end{align*}
に対して、
\begin{align*}
\zeta_{\Gamma}(s)=\frac{\zeta(s)\zeta(s-1)}{L_D(x)}
\end{align*}
となる。

The Birch and Swinnerton-Dyer Conjecture, a Computational Approach
William A. Stein
%https://williamstein.org/books/bsd/bsd.pdf

Distinguished Lecture Series I: Shou-wu Zhang
%https://terrytao.wordpress.com/tag/bsd-conjecture/
\end{document}