\documentclass{jsarticle}
\setlength{\textwidth}{\fullwidth}
\setlength{\evensidemargin}{\oddsidemargin}
\RequirePackage{amsmath,amssymb,amsthm, amscd, comment, multicol}
\usepackage[all]{xy}
\input{../tex/theorems}
\input{../tex/symbols}
\usepackage[dvipdfmx]{graphicx}
\usepackage{tikz}
\usepackage{tkz-euclide}
\usetkzobj{all}
\usetikzlibrary{intersections, calc}
\title{Gauss和とFourier変換}
\author{@unaoya}
\date{\today}
\begin{document}

\maketitle
\section{はじめに}
http://people.cs.uchicago.edu/\verb|~|laci/reu02/fourier.pdf
に従って、次の定理の証明を紹介する。

\begin{thm}
$k$を整数$q$を素数$q\geq k^4+4$とする。
このとき
\[
x^k+y^k=z^k
\]
は$\F_q$に非自明な解を持つ。
\end{thm}

\section{有限巡回群の指標}
$G$を有限巡回群とする。
\begin{dfn}[指標]
$G$の指標とは群準同型$\chi: G\to \C^\times$のことをいう。

自明な指標を$\chi_0$と書くことにする。
つまり任意の$x\in G$について$\chi_0(x)=1$である。
\end{dfn}

$\chi_1, \chi_2$が共に$G$の指標であるとき、$\chi_1\chi_2(x)=\chi_1(x)\chi_2(x)$と定めることで指標の積を定義できる。
このようにして定まる$G$の指標全体の群を$\hat{G}$とかく。

$G$が有限巡回群$\Z/n$であるとき、指標は$\chi_k:x\mapsto\exp(2\pi i\dfrac{k}{n}x)$で与えられるもので全て。
これにより$\hat{G}$も位数$n$の巡回群になることがわかる。

$G$の元$x$を一つ取ると、これは$\hat{G}$の指標$f_x:\hat{G} \to \C^\times$を$f_x(\chi)=\chi(x)$により定めることができる。

\begin{lem}
$G$を巡回群で$\chi$をその指標としたとき
\begin{align*}
\sum_{x\in G}\chi(x)=\begin{cases}0&(\chi\neq\chi_0)\\\abs{G}&(\chi=\chi_0)\end{cases}
\end{align*}
\end{lem}
\begin{proof}
$S=\displaystyle\sum_{x\in G}\chi(x)$とする。
$\chi(y)S=\displaystyle\sum_{x\in G}\chi(y)\chi(x)=\displaystyle\sum_{x\in G}\chi(yx)=S$より$\chi(y)\neq1$なら$S=0$となる。
\end{proof}

上で見たことから
\begin{align*}
\sum_{\chi\in\hat{G}}\chi(x)=\begin{cases}0&(\chi\neq\chi_0)\\\abs{G}&(\chi=\chi_0)\end{cases}
\end{align*}
も成り立つ。

$\C^G$を$G$上の複素数値関数全体の集合とし、$\C$線形空間とみなす。
この空間に内積を
\begin{align*}
(f,g)=\frac{1}{n}\sum_{x\in G}f(x)\overline{g(x)}
\end{align*}
とすることでこれは有限次元のHilbert空間となる。
特にCauchy-Schwarz不等式が成り立つ。

\begin{lem}
$G$の指標はHilbert空間$\C^G$の正規直交基底となる。
\end{lem}
\begin{proof}
$\hat{G}$は位数が$n$で$\C^G$は$n$次元なので個数はあう。
直交することは
\begin{align*}
(\chi_1, \chi_2)&=\frac{1}{n}\sum_{x\in G}\chi_1(x)\overline{\chi_2(x)}\\
&=\frac{1}{n}\sum_{x\in G}\chi_1\chi_2^{-1}(x)
\end{align*}
と計算すれば、上の補題から証明できる。
\end{proof}

\section{Fourier変換}
$G$を位数$n$の巡回群とする。
\begin{dfn}[Fourier変換]
$f\in\C^G$のFourier変換$\hat{f}:\hat{G}\to \C^\times$を
\begin{align*}
\hat{f}(\chi)=\sum_{x\in G}\chi(x)f(x)
\end{align*}
と定める。
また$g\in\C^{\hat{G}}$のFourier逆変換$\hat{g}:G \to \C$を
\begin{align*}
g(x)=\frac{1}{n}\sum_{\chi\hat{G}}g(\chi)\chi(-x)
\end{align*}
と定める。
\end{dfn}

前の補題を用いると、Fourier逆変換公式を証明することができる。
$\delta_0\in\C^G$を$0$に台を持つ特性関数とすると、$\hat{\delta}_0(\chi)=1$であり、
Fourier逆変換公式から
\begin{align*}
\delta_0=\sum_{\chi\in\hat{G}}\chi
\end{align*}
となる。

\begin{thm}[Plancherel公式]
\begin{align*}
(\hat{f},\hat{g})=n(f,g)
\end{align*}
である。特に
\begin{align*}
\norm{f}=\norm{\hat{f}}
\end{align*}
\end{thm}
\begin{proof}
一点$a, b$に台を持つ$\delta$関数$\delta_a, \delta_b$を用いて確かめると
\begin{align*}
(\delta_a,\delta_b)=\begin{cases}\dfrac{1}{n}&a=b\\0&a\neq b\end{cases}
\end{align*}
であり、また
\begin{align*}
\hat{\delta}_a(\chi)=\sum_{x\in G}\delta_a(x)\chi(x)=\chi(a)
\end{align*}
であることから
\begin{align*}
(\hat{\delta}_a,\hat{\delta}_b)&=\frac{1}{n}\sum_{\chi\in\hat{G}}\chi(a)\chi(b)=\begin{cases}1&a=b\\0&a\neq b\end{cases}
\end{align*}
と計算できる。
\end{proof}

\section{Gauss和}
ここでは$q$を素数とし、$\Z/q=\F_q$と書く。
また$\F_q$のかけ算を考えることで$\F_q-\{0\}=\F_q^\times$も巡回群になることがわかる。

\begin{dfn}[Gauss和]
加法的指標$\chi:\F_p\to\C^\times$と乗法的指標$\psi:\F_p^\times\to\C^\times$にたいし
\begin{align*}
S(\chi,\psi)=\sum_{a\in\F_p}\chi(a)\psi(a)=\sum_{a\in\F_p^\times}\chi(a)\psi(a)
\end{align*}

$\psi(0)=0$とすることで$\psi:\F_p \to \C^\times$とみなす。
\end{dfn}

\begin{prop}[Gauss和の評価]
$\psi$が非自明な時
\begin{align*}
\abs{S(\chi,\psi)}=\sqrt{q}
\end{align*}
\end{prop}
\begin{proof}
\begin{align*}
\abs{S(\chi,\psi)}&=\sum_{x\in G}\chi(x)\psi(x)\sum_{y\in G}\overline{\chi(y)}\overline{\psi(y)}\\
&=\sum_{x,y\in G}\chi(x)\chi(-y)\psi(x)\psi(y^{-1})\\
&=\sum_{x,y\in G^\times}\chi(x-y)\psi(xy^{-1})
\end{align*}
ここで$z=xy^{-1}$と変数変換すると
\begin{align*}
\abs{S(\chi,\psi)}&=\sum_{z,y\in G^\times}\chi(yz-y))\psi(z)\\
&=\sum_{z\in G^\times}\psi(z)\sum_{y\in G^\times}\chi((z-1)y)
\end{align*}
となる。
さらに$z=1$かどうかで場合分けすると、$\chi:G\to\C^\times$が指標なので
\begin{align*}
\sum_{y\in G^\times}\chi((z-1)y)=\begin{cases}q-1&(z=1~\mbox{or}~\chi=\chi_0)\\-1&\mbox{それ以外}\end{cases}
\end{align*}
となる。
したがって、
\begin{align*}
\abs{S(\chi,\psi)}&=(q-1)\psi(1)+\sum_{z\neq1}\psi(z)(-1)\\
&=q-1+\sum_{z\in G^\times}\psi(z)(-1)+1=q
\end{align*}
となる。
\end{proof}

Gauss和とGauss周期の関係について。
Fourier変換
\begin{align*}
\hat{1}_A(\chi)=\sum_{x\in A}\chi(x)
\end{align*}
において$\chi(x)=\exp(2\pi ix/p)$で$A$を部分群のcosetとすればよい。

\begin{prop}
$A\subset \F_q^\times$を指数$k$の部分群とし$\psi_0,\ldots,\psi_{k-1}$を$\F_q^\times/A$の指標全体とする。
この指標から誘導される$\F_q^\times$の指標も同じく$\psi_0,\ldots,\psi_{k-1}$とする。
$\F_q$の指標$\chi$にたいし
\begin{align*}
\hat{1}_A(\chi)=\frac{1}{k}\sum^{k-1}_{i=0}S(\chi,\psi_i)
\end{align*}
\end{prop}
\begin{proof}
\begin{align*}
\sum^{k-1}_{i=0}S(\chi,\psi_i)&=\sum^{k-1}_{i=0}\sum_{x\in G}\chi(x)\psi_i(x)\\
&=\sum_{x\in G}\chi(x)(\sum_{i=0}^{k-1}\psi_i(x))
\end{align*}
ここで指標の和に関する公式
\begin{align*}
\sum_{\psi\in\hat{H}}\psi(x)=\begin{cases}\abs{H}&(x=0)\\0&(x\neq0)\end{cases}
\end{align*}
を思い出すと、
\begin{align*}
\sum_{i=0}^{k-1}\psi_i(a)=\begin{cases}0~(a\not\in A\mbox{もしくは}a=0)\\k~(a\in A)\end{cases}
\end{align*}
となる。
よって
\begin{align*}
\sum^{k-1}_{i=0}S(\chi,\psi_i)&=\sum_{x\in G}\chi(x)k1_A(x)=\hat{1}_A(x)
\end{align*}
とできる。
\end{proof}

部分集合$A\subset G$に対し
\begin{align*}
\Phi(A) = \max_{\chi\neq\chi_0}\abs{\hat{1}(\chi)}
\end{align*}
とする。
上の二つを合わせて
\begin{thm}
部分群$A\subset G^\times$について
\begin{align*}
\Phi(A)<\sqrt{q}
\end{align*}
\end{thm}

\begin{proof}
\begin{align*}
\abs{\hat{1}_A(\chi)}&\leq\frac{1}{k}\sum_{\psi}\abs{S(\chi,\psi)}\\
&\leq\frac{1}{k}(\abs{S(\chi,\psi_0)}+\sqrt{q}(k-1))
\end{align*}
となる。
前に示したように$\abs{S(\chi,\psi)}=\sqrt{q}$であり
\begin{align*}
S(\chi,\psi_0)=\sum_{x\in G^\times}\chi(x)=-1
\end{align*}
であることから、
\begin{align*}
\frac{1}{k}(\abs{S(\chi,\psi_0)}+\sqrt{q}(k-1))\leq\frac{1}{k}(1+\sqrt{q}(k-1))\leq\sqrt{q}
\end{align*}
となる。
\end{proof}

\section{方程式の解の個数}
\begin{thm}
部分集合$A_1, A_2, A_3 \subset G=\Z/n$に対し、$N$を$x_1+x_2+x_3=0, x_i\in A_i$の解の個数とする。
この時
\begin{align*}
\abs{N-\frac{\abs{A_1}\abs{A_2}\abs{A_3}}{n}}<\Phi(A_3)\sqrt{\abs{A_1}\abs{A_2}}
\end{align*}
が成り立つ。
\end{thm}
\begin{proof}
\begin{align*}
N&=\sum_{x_i\in A_i}\delta(x_1+x_2+x_3)\\
&=\frac{1}{n}\sum_{x_i\in A_i}\sum_{\chi\in\hat{G}}\chi(x_1+x_2+x_3)\\
&=\frac{1}{n}\sum_{x_i\in A_i}\chi_0(x_1+x_2+x_3)+\frac{1}{n}\sum_{\chi\neq\chi_0}\sum_{x_i\in A_i}\chi(x_1+x_2+x_3)
\end{align*}
となる。
$\chi$が指標であるから$\chi(x_1+\cdots+x_k)=\chi(x_1)\cdots\chi(x_k)$であり、
和を因数分解すると
\begin{align*}
\sum_{x_i\in A_1\i}\chi_0(x_1+x_2+x_3)=\sum_{x_i\in A_i}\chi_0(x_1)\chi_0(x_2)\chi_0(x_3)=\abs{A_1}\abs{A_2}\abs{A_3}
\end{align*}
となる。

第二項も同様に計算でき、これを評価していく。
\begin{align*}
\abs{\sum_{\chi\neq\chi_0}\sum_{x_i\in A_i}\chi(x_1)\chi(x_2)\chi(x_3)}&
=\abs{\sum_{\chi\neq\chi_0}(\sum_{x\in G}\chi1_{A_1}(x))(\sum_{x\in G}\chi1_{A_2}(x))(\sum_{x\in G}\chi1_{A_3}(x))}\\
&=\abs{\sum_{\chi\neq\chi_0}\hat{1}_{A_1}(\chi)\hat{1}_{A_2}(\chi)\hat{1}_{A_3}(\chi)}\\
&\leq\Phi(A_3)\sum_{\chi\in\hat{G}}\abs{\hat{1}_{A_1}(\chi)}\abs{\hat{1}_{A_2}(\chi)}\\
\end{align*}
と計算できる。

さらに$\C^G$がHilbert空間なので、Cauchy-Schwarzより
\begin{align*}
\sum_{\chi\in\hat{G}}\abs{\hat{1}_{A_1}(\chi)}\abs{\hat{1}_{A_2}(\chi)}
&\leq\sqrt{(\sum_{\chi\in\hat{G}}\abs{\hat{1}_{A_1}(\chi)}^2)(\sum_{\chi\in\hat{G}}\abs{\hat{1}_{A_2}(\chi)}^2)}\\
&=\sqrt{n^2\abs{A_1}\abs{A_2}}
\end{align*}
と計算できる。
\end{proof}

\begin{thm}
$k\vert q-1$および部分集合$A_1,A_2\subset \F_q$に対し、$N$を方程式
\begin{align*}
x+y=z^k~(x\in A_1,y\in A_2, z\in\F_q^\times)
\end{align*}
の解の個数とする。

このとき
\begin{align*}
\abs{N-\frac{\abs{A_1}\abs{A_2}(q-1)}{q}}<k\sqrt{\abs{A_1}\abs{A_2}q}
\end{align*}
が成立する。
\end{thm}
\begin{proof}
$A_3=\{z^k\mid z\in\F_q^\times\}$とし、$N'$を
\begin{align*}
x+y=u~(x\in A_1, y\in A_2, u\in A_3)
\end{align*}
の解の個数とする。
$k\vert q-1$なので$z^k=u$は$k$個解を持つ。
したがって$N=kN'$となる。
\begin{align*}
\abs{N-\frac{\abs{A_1}\abs{A_2}\abs{A_3}}{n}}<\Phi(A_3)\sqrt{\abs{A_1}\abs{A_2}}
\end{align*}
であり、$\Phi(A_3)<\sqrt{q}$を使えば証明できる。
\end{proof}

\section*{参考文献}
http://people.cs.uchicago.edu/\verb|~|laci/reu02/fourier.pdf

\end{document}