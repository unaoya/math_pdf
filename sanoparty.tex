\documentclass[dvipdfmx]{beamer}
\usetheme{metropolis}

\input{../tex/theorems}
\input{../tex/symbols}
\usepackage{bxdpx-beamer}
\title{結び目とエタールコホモロジー}
\author{梅崎直也@unaoya}
\date{佐野さん3年間お疲れ様セミナー}
\begin{document}

\begin{frame}
\maketitle
\end{frame}

\begin{frame}
\begin{enumerate}
\item Khovanov triply graded homology
\item Kazhdan-Lusztig conjecture
\item geometric interpretation of invariants by Webstar-Williamson
\end{enumerate}
\end{frame}

\begin{frame}{昨日の話}

  trefoilとKhovanov homologyの図式

  \begin{align*}
    0 \to C^0 \to C^1 \to C^2 \to C^3 \to 0
  \end{align*}
  これが二重次数つき複体。
  これの次元をとることでJones多項式が得られる。

  三重次数つきの複体を作り、その次元をとることでHOMFLYPT多項式をえるものを作る。
  (最初はKhovanov-Rozansky?)
\end{frame}

\begin{frame}{別の構成}
  trefoilからbraidの図をかく。
  $n=2$のbraidで$\sigma=\sigma_1^3$と表すことができる。

  $F(\sigma_1):0\to R\{2\} \to B_1 \to0$と$F(\sigma_1^{-1}):0 \to B_1\{-2\} \to R\{-2\}\to0$で定める。
  ここで、コホモロジー次数はそれぞれ$B_1, R\{-2\}$を$0$次にする。

  上では$R=\Q[y], R_1=\Q[y^2], B_1=R\otimes_{R_1}R$とし、
  $rb_1:R\{2\} \to B_1; 1 \mapsto y\otimes1+1\otimes y$で定める。
  また$br_1:B_1\{-2\} \to R\{-2\}; 1\otimes 1\to1$で定める。

  これらは次数つき $R$-bimoduleの射。

  さらに$F(\sigma)=F(\sigma_1)^{\otimes 3}$で定義。
  ここで複体のテンソル積は
\end{frame}

\begin{frame}{Braid群}
  m本のbraid群とは、紐の図をかく

  $\sigma_i$を$i$番目と$i+1$番目の入れ替えで$i$番目が下を通るようにする。

  これに対し、前と同様な複体を
  $R=\Q[x_1-x_2, \ldots, x_{m-1}-x_m], R_i=R^{(i, i+1)}, B_i=R\otimes_{R_i}R$とし、$rb_i, br_i$を定め。
  $F(\sigma_i)=o\to R\{2\} \to B_i \to0, F(\sigma_i^{-1}):0 \to B_i\{-2\} \to R\{-2\} \to 0$
  とする。コホモロジーの次数は前と同様。

  これを用いて$F(\sigma)$をテンソル積で定義する。
  これはup to homotopyでwell-defined
\end{frame}

\begin{frame}{HHH}
  $F(\sigma)$のHochshild homologyをとる。

  $R$-bimod $M$の$HH$とは$HH_i(R,M)=Tor_i^{R\otimes R}(R,M)$なるもの。
  これは$M \mapsto M_R=R\otimes_{R\otimes R}M=M/[R,M]$のderived functorである。

  $HHH(\sigma): \to HH(R,F^i(\sigma))\to HH(R,F^{i+1}(\sigma))\to$

  すると$HH$の次数、$F$の次数、$R$-bimodの次数と三つの次数がつき、
  これはKovanov-Rozanskyで定義したものと同じ
\end{frame}

\begin{frame}{Soergel bimodule}
  上に出てきた$R_i, B_i$は何か?
  Soergel bimoduleとは、categorification of Hecke algebraである。

  $\B_i=B_i\{-1\}$とすると、これは以下の関係式を満たす。
  \begin{align*}
    \B_i\otimes_R\B_i&=\B_i\{1\}\oplus\B_i\{-1\}\\
    (\B_i\otimes\B_{i1+}\otimes \B_i)\oplus\B_{i+1}&=(\B_{i+1}\otimes\B_i\otimes\B_{i+1})\oplus\B_i\\
    \B_i\otimes\B_j&=\B_j\otimes\B_i&i\neq j\pm1
  \end{align*}
\end{frame}

\begin{frame}{Kazhdan-Lusztig basis}
  Soergel bimoduleが定義された背景にはKazhdan-Lusztig予想という表現論の問題があった。

  上の関係式で$\B_i=C_i', \{1\}=q$とすると、
  \begin{align*}
    C_i'^2&=(q+q^{-1})C_i'\\
    C_i'C_{i+1}'C_i'+C_{i+1}'&=C_{i+1}'C_i'C_{i+1}'+C_i'\\
    C_i'C_j'&=C_j'C_i'&i\neq j\pm1
  \end{align*}
  これはHecke algebraのKazhdan-Lusztig basisというものを与えている。

  この解釈のもと、$\B_w$がindecomposableであり、
  このことから$F(\sigma)$を分解して自明なところを消去することで$F_{min}(\sigma)$を得る。
  これは計算がだいぶ楽になる。
\end{frame}

\begin{frame}{$sl_2$の表現}
  $\fg$の表現について、Weyl群、最高ウェイト加群

  $\fg=sl_2$のとき。
  $h=\begin{pmatrix}1&0\\0&-1\end{pmatrix},
  e=\begin{pmatrix}0&1\\0&0\end{pmatrix},
  f=\begin{pmatrix}0&0\\1&0\end{pmatrix}$
  とする。
  交換関係は$h=[e,f], [h,e]=2e, [h,f]=-2f$である。
  $\fg$の表現を$h$の固有空間分解して調べる。

  $S_2$が$h \mapsto -h$で$\fg$(ほんとはCartanにのみ?)に作用する。
\end{frame}

\begin{frame}{$sl_3$の表現}
  $\fg=sl_3$のとき。
  $h_1=\begin{pmatrix}1&0&0\\0&-1&0\\0&0&0\end{pmatrix},
  h_2=\begin{pmatrix}0&0&0\\0&1&0\\0&0&-1\end{pmatrix}$
  とし、$e_1, e_2, f_1,f_2$も適切に定める。
  $\fg$の表現$V$があれば$h$について同時固有空間分解$V=\oplus_{\nu\in h^*}V_\nu$とできる。

  $W=S_3$である。
\end{frame}

\begin{frame}{Varma加群}
weightとroot

$h^\vee$の基底$\alpha_i$を$(\alpha_i,h_j)=a_{ij}$となるように定義。
$Q^+$を$\alpha_1,\ldots,\alpha_n$で貼られるもの、
$e_i,h_i,f_i$をそれぞれ次数$\alpha_i,0,-\alpha_i$とし、
$\fg=\oplus_\beta\fg_\beta$とした時、$\beta$がrootとは$\fg_\beta\neq0$となること。

positive root

$\rho=\sum_{\alpha>0}\frac{\alpha}{2}$とし、
$w\cdot0=w\rho-\rho$とする。

$w$に対応するVerma加群とは、$w\cdot0$を最高ウェイトに持つ中で普遍性を持つもの。

これが唯一の既約商を持つ。
\end{frame}

\begin{frame}{Hecke環}
  $W=S_n$とする。
  $H_n$を$T_w, w\in W$を基底に持つ$\Z[q^{1/2},q^{-1/2}]$代数で、以下の関係式を満たすもの。
  ここで$s_i=(i, i+1)$に対応する$T_{s_i}$を$T_i$と書いた。
  \begin{align*}
    (T_i-q^2)(T_i+1)&=0\\
    T_iT_{i+1}T_i&=T_{i+1}T_iT_{i+1}\\
    T_iT_j&=T_jT_i&i\neq j\pm1
  \end{align*}

  $W$の群環とIwahori-Hecke代数
  $T_w$を基底にもち$\Z[q^{1/2}, q^{-1/2}]$上生成される環で、積は
  \begin{align*}
    T_yT_w=T_{yw}& l(yw)=l(y)+l(w)\\
    (T_s+1)(T_s-q)=0 & s\in S
  \end{align*}
  で乗法が定まる。としても定義可能。
\end{frame}

\begin{frame}{involutionの存在}
  これはinvolution $\iota:H_n \to H_n, q^{1/2} \mapsto q^{-1/2}, T_w\mapsto(T_{w^{-1}})^{-1}$
  を持つ。

  Braid群の群環の商であり、$q=1$とすると$W$の群環になる。
  また$G(\F_q)$の両側$B$不変$\C$値関数のなすconvolution代数と同型である。
\end{frame}

\begin{frame}{Kazhdan-Lusztig基底}
  Hecke代数の基底として、次のような性質を満たすものがある。
  これをKazhdan-Lusztig基底という
  \begin{prop}
    次を満たす$\iota$不変な要素からなる基底$\{C'_w\}_{w\in W}$が存在する。
    
    \begin{align*}
      C'_w=q^{-l(w)/2}\sum_{y\leq q}P_{y,w}T_y
      P_{y,w}(q)\in\Z[q], P_{w,w}=1\\
      \deg P_{y,w}<l(w)-l(y)&y\neq w
    \end{align*}
  \end{prop}
  これは組み合わせ的に証明できる。
  この$P$をKazhdan-Lusztig多項式と呼ぶ。
  これは次で見るように、表現論において重要な対象である。
\end{frame}

\begin{frame}{Kazhdan-Lusztig予想}
  $w\in W$に対し、$w(\rho)-\rho$を最高weightに持つVerma module $M_w$と
  $L_w$を最高weight加群とする($M_w$の唯一の既約商)。
  この時、これらの指標の関係式がKazhdan-Lusztig多項式を用いて
  \begin{align*}
    ch(L_w)=\sum_{y\leq w}(-1)^{l(w)-l(y)}P_{y,w}(1)ch(M_w)
  \end{align*}
  と書ける。
\end{frame}

\begin{frame}{Schubert多様体}
  Kazhdan-Lusztig多項式はSchubert多様体のintersectionコホモロジーを用いて
  \begin{align*}
    P_{y,w}(q)=\sum_iq^i\dim IH_{X_y}^{2i}(\overline{X}_w)
  \end{align*}
  と書ける(Kazhdan-Lusztig)
  (柏原谷崎のKazhdan-Lusztig予想をめぐってを参考)

  証明の方針

  $j_x:O_w\to X$として
  \begin{align*}
    h(j_{w,!}(\Lambda_{O_w})[l(w)])=T_w\\
    h(j_{w,*}(IC_{\bar{O}_w})=C_w'\\
  \end{align*}
  である。
  BBDGやtrace formula, dualityを使う。

  ここで設定として$IC_w=j_{!*}\overline{\Q}_\ell, j_w:Y(w)\to X(w)$とし、
  $X=G/B, Y(w)=BwB/B \sim \A^{l(w)}$と、$X(w)=\overline{Y(w)}$とする。
  $X(w)$は$v\leq w$なる$v$についての$Y(v)$でstratified。
\end{frame}

\begin{frame}{Schubert多様体のintersection cohomology}
  purityとdecomposition theoremを使う。

  $X=G/B, Y(w)=BwB/B\simeq \A^{l(w)}\subset X(w)=\overline{Y(w)}$とする。
  $X(w)$は$v\leq w$なる$Y(w)$でstratified。

  $j_w:Y(w) \to X(w)$をBruhat cellとして
  $IC_w=j_{w,!*}\overline{\Q}_\ell$とする。

  この辺はKiehl-Weissauerを参照
  \begin{align*}
    h(j_{w,!}(\Lambda_{O_w})[l(w)])&=T_w\\
    f(j_{w,*}(IC_{O_w}))&=C'_w
  \end{align*}
  BBDG, trace formula, dualityなど
\end{frame}

\begin{frame}{Kazhdan-Lusztig予想の証明}
  Beilinson-BernsteinとBrylinski-Kashiwaraによる。

  Bruhat分解$G=\coprod_{w\in W}BwB$とSchubert多様体$G/B=\coprod_{w\in W}X_w$

  Beilinson-Bernstein localization
  $\fg$の表現と旗多様体の(twisted) $D$加群の対応
  $\lambda$を整ウェイトで任意の$i\in I$について$\lambda(h_i)\in\Z_{>0}$を満たすとする。
  この時$\Gamma(X,-):RH^0_l(D^\lambda_X) \to M(\fg)$は完全関手で
  $B_w(\lambda)\mapsto M(w\circ\lambda)^*, M_w(\lambda)\mapsto M(w\circ\lambda), L_w(\lambda)\mapsto(w\circ\lambda)$を満たす。

  Riemann-Hilbert対応
  正則ホロノミック$D$加群とperverse sheafの対応
  $Sol$が正則ホロノミック$D$加群をperverse sheafに移す。
  $B_w(\lambda)\mapsto \C_{X^w}[-l(w)], M_w(\lambda)\mapsto D(\C_{X^w}[-l(w)],L_w(\lambda)\mapsto {}^\pi\C_{X^w}$となる。
\end{frame}

\begin{frame}{perverse sheafとは}
  $D^b(X,\overline{\Q}_\ell)$の$t$-structureをつぎのように定める。
  ${}^pD^{\leq0}(X)$を$\dim supp(\mathcal{H}^{-i}B)\leq i$であるもの
  ${}^pD^{\geq0}(X)$を$\dim supp(\mathcal{H}^{-i}DB)\leq i$であるものとし、
  $Perv(X)={}^pD^{\leq0}(X)\cap{}^pD^{\geq0}(X)$とさだめる。

  $j:U \to X$をopen immとしたとき、$j_{!*}:Perv(U) \to Perv(X)$が定まり、
  $Dj_{!*}B=J_{!*}DB$を満たす。
  (これはcohomologyがdualityをみたすということ?_)

  Gabberの定理
  $B_0$を$\tau$-mixed perverse sheafとした時、
  $w(B_0)\leq w$であることは、すべての$Y_0\subset X_0$既約$d$次元としたとき、
  あるopen dense $U_0\subset Y_0$が存在して、$w(\mathcal{H}^{-d}B_0\vert_{U_0})\leq w-d$をみたす。

  これの系として
  $w(B_0)\leq w$であることと、$w({}^pH^v(B_0))\leq w+v$と同値
\end{frame}

\begin{frame}{intersection cohomologyとperverse sheaf}

  動機。特異点のある場合のPoincare双対など

  derived categoryとperverse $t$-structure
  (この$t$-structureについてコホモロジーを取るとdualityを満たす?)

  semi-small mapとpush-forward

  decomposition theorem

  weight filtration

  perverse sheafとweight

  Weil予想、ゼータ関数のゼロ点とFrobenius固有値
\end{frame}

\begin{frame}{perverse sheaf}
  $D^b(X,\overline{\Q}_\ell)$の$t$-structureを次で定める。
  \begin{align*}
    {}^pD^{\leq 0}(X)=\{\dim supp(H^{-i}B)\leq i\}\\
    {}^pD^{\geq 0}(X)=\{\dim supp(H^{-i}DB)\leq i\}\\
    Perv(X)={}^pD^{\leq0}(X)\cap{}^pD^{\geq0}(X)
  \end{align*}
  と定義する。
\end{frame}

\begin{frame}{Riemann予想とWeil予想}
  Riemann zeta
  \begin{align*}
    \zeta(s)=\sum\frac{1}{n^s}=\prod\frac{1}{1-p^{-s}}
  \end{align*}

  合同zetaは$X/\F_q$に対して、
  \begin{align*}
    Z(X,t)=\exp(\sum_m\frac{N_m}{m}t^m)\\
    N_m=\abs{X(\F_{q^m})}\\
    \zeta_X(s)=\log(Z)=N_1t+\frac{t^2}{2}N_2+\cdots=
    \prod_{x\in\abs{X_0}}\det(1-t^{d(x)}F_x;\mathcal{G}_0
  \end{align*}
  とする。
\end{frame}

\begin{frame}{\'etale cohomology}
  これは次の有理性を満たす(Grothendieck, trace formula)
  \begin{align*}
    Z(X,t)=\prod_{i=0}^{\dim X}\det(1-tFrob_q\vert H^i_c(X,\Q_\ell))^{(-1)^{i+1}}
  \end{align*}

  リーマン予想の類似。
  $t=q^{-s}$とした時のゼロ点。
  Frobeniusの$H^i_c(X,\Q_\ell)$の固有値が$\dfrac{i}{2}$となる。
\end{frame}

\begin{frame}{local system}
  $\ell$進層、局所系、基本群
\end{frame}

\begin{frame}{weight}    
    $\F_0$が$\tau$-pure of weight $\beta$とは
    全ての$x\in\abs{X_0}$について、その$f_x:\G_{0,\bar{x}}\to \G_{0,\bar{x}}$の固有値が$\abs{\tau(\alpha)}=N(x)^\beta$なること。

    poncturement pureとは?

    $\F_0$が$\tau$-mixedとは、
    ある有限filtration $W$が存在して$Gr^W$が$\tau$-pure of weight $\beta$であること。
    $w(\F_0)=\sup_{x\in\abs{X_0}}\sup_\alpha\log\abs{\tau(\alpha)}^2$

    mixed complexとは全てのcohomology sheafが$\tau$-mixedであること。

    $K_0\in D^b_c(X_0,\overline{Q}_\ell)$に対し、
    $K_0\in D^b_{\leq w}(X_0)$であるとは$w(K_0)=\max_\nu(w(H^\nu(K_0)))\leq w$であること。
    $K_0$がpureとは$D^b_{\leq w}(X_0)\cap D^b_{\geq w}(X_0)$に入ること。
\end{frame}

\begin{frame}{Deligneの定理}
  \begin{thm}[Deligne]
    $f_0:X_0\to Y_0$と$\F_0/X_0$に対し、
    \begin{enumerate}
    \item $\F_0$が$\tau$-pure of weight $\beta$なら$R^if_{0!}\F_0$の$\tau$-weightはある$n$に対し$\beta+i-n$となる。
    \item $\F_0$がmixedなら$R^if_{0!}\F_0$はmixed
    \end{enumerate}
  \end{thm}

  perverse sheafにおいてもweight filtrationが存在する。(これはDeligneではない?BBD?)
\end{frame}


\begin{frame}{不変量の幾何的定義}
  結び目の図式からある多様体とその上の層の複体を定義する。

  weight filtrationからspectre系列を作る。

  $E_2$-pageが二重複体で、さらにここにweightでもう一つ次数が入って、三重次数複体。
\end{frame}

\begin{frame}{Webster-Williamson}
  Braid $\beta$やlink $L$から$\Phi_\beta\in D_{B\times B}(GL(N)), \F_D\in D_{G_D}(X_D)$を作る。
  これに対し、
  $\mathbb{H}^*_B(GL(N),\Phi_\beta), \mathbb{H}^*_{G_D}(X_D,\F_D)$はtriply graded

  $\F_D$のweight filtrationから定まるspectral sequenceにより$A_2^{p,q}$を定める。
  \begin{thm}
    \begin{align*}
      \mathcal{E}(L)=\mathcal{E}_{G_D}(X_D,\F_D)=\sum_{l,j,k}(-1)^lq^jt^kA_2^{j,k,l}(L)
    \end{align*}
    はHOMFLYPT多項式
  \end{thm}

  $\mathbb{H}^{j-l,j-k}(gr_l^W\F_D)$の部分商が$A^{j,k,l}(L)$に入る。
  ここで、
  \begin{align*}
    \mathbb{H}^{*,i}(\F)=\oplus_{\abs{\alpha}=q^{i/2}}\mathbb{H}^*_\alpha(\F)
  \end{align*}
\end{frame}

\begin{frame}{Markov? traceの幾何的な記述}
  $\sigma\in B_n$に対し$[\Phi_\sigma]$が定まる。
  $K:D^b_{B\times B}(G)\to H_n$が存在し、
  trace $H_n \to \C(t,q^{1/2})$と$\dim \mathbb{H}_B:D^b_{B\times B}(G) \to \C(t,q^{1/2})$が存在。
  さらに$IC_w \in D^b_{B\times B}(G) \mapsto S_w=\mathbb{H}^*_{B\times B}(G, IC_w)$という$R$-bimodが定まる。
  これの$\dim HH$が$\C(t,q^{1/2})$の元。
  これらの相互の関係は?

  \begin{align*}
    \Phi_\sigma=j_{w,*}k_{BwB}<l(w)>\mapsto q^{1/2}\sigma_i\in H_n\\
    IC_w=j_{w,!*}k_{BwB}<l(w)/2>\mapsto \sigma_i\in H_n
  \end{align*}
  で定める?
\end{frame}

\begin{frame}
  $\F_D\in D_{G_D}(X_D)$の構成

  結び目からグラフを作って
  $G_D=\prod_e GL(1), X_D=\prod_v GL(2)$とする。
  (colored linkの場合は行列のサイズが変わる)
  
  \begin{align*}
    (a,b,c,d)x=\begin{pmatrix}a&0\\0&b\end{pmatrix}x\begin{pmatrix}c^{-1}&0\\0&d^{-1}\end{pmatrix}
  \end{align*}
  
  Bruhat分解$GL(2)=B\amalg BsB=B\amalg U$とし、$k:U \to GL(2)$とする。
  結び目の交差の上下に対応して、$k_*\Lambda_U<2>$または$k_!\Lambda_u<2>$を$\F_v$とする。
  ここで$k_*, k_!$の関係はPoincare duality$Rf_*(D_X(K))\sim D_S(Rf_!(K))$がある。
  
  \begin{align*}
    \F_D=res^G_{G'}(\boxtimes\F_v)\in D^b_{G_D}(X_D)
  \end{align*}
  と定義する。
\end{frame}

\begin{frame}
$W$をWeyl群、$S$を単純鏡映
例$W=S_n, S=\{(i,i+1),i\}$

$S_n$が$sl_n$のWeyl群になっているということを理解する。

$W$の表現

有限群$G$の表現
$G$の共役類と$G$の既約表現は個数が等しい。

$S_n$の共役類は$n$の分割に対応する。

この対応を幾何的に構成する。
$n$の分割は$GL_n$もしくは$SL_n$のJordan標準形に対応。
\end{frame}

\begin{frame}
Fourier変換、Springer対応

$W$をGalois群に持つ被覆
正則表現の分解
intermediate extension
Fourier変換
するとSpringer fiberのコホモロジーが出てくる
\end{frame}

\begin{frame}
convolution代数としての$\Z[W]$の構成。

Lusztig
Steinberg多様体の上の構成可能関数とその合成石で定まる代数が$\Z[W]$と同型。

Fourier変換との関係は?
\end{frame}

\begin{frame}
KhovanovのSpringer多様体。
Grassmannianとの関係、geometric Satake
\end{frame}
\end{document}
