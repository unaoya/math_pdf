\documentclass[dvipdfmx]{beamer}
\input{../tex/theorems}
\input{../tex/symbols}
\usepackage{bxdpx-beamer}
\title{結び目とエタールコホモロジー}
\author{梅崎直也@unaoya}
\date{\today}
\begin{document}

\begin{frame}
\maketitle
\end{frame}

\begin{frame}
\begin{enumerate}
\item 幾何学的表現論
\begin{enumerate}
\item Springer対応
\item Kazhdan-Lusztig多項式
\end{enumerate}
\item 結び目の不変量
\begin{enumerate}
\item Khovanovによる構成
\item Stroppel
\item Webster-Williamson
\item Cautis-Kamnitzer?
\end{enumerate}
\end{enumerate}
\end{frame}

\begin{frame}
$W$をWeyl群、$S$を単純鏡映
例$W=S_n, S=\{(i,i+1),i\}$

$S_n$が$sl_n$のWeyl群になっているということを理解する。

$W$の表現

有限群$G$の表現
$G$の共役類と$G$の既約表現は個数が等しい。

$S_n$の共役類は$n$の分割に対応する。

この対応を幾何的に構成する。
$n$の分割は$GL_n$もしくは$SL_n$のJordan標準形に対応。
\end{frame}

\begin{frame}
intersection cohomologyとperverse sheaf

動機。特異点のある場合のPoincare双対など

derived categoryとperverse $t$-structure

semi-small mapとpush-forward

decomposition theorem

weight filtration

perverse sheafとweight

Weil予想、ゼータ関数のゼロ点とFrobenius固有値
\end{frame}


\begin{frame}
Fourier変換、Springer対応

$W$をGalois群に持つ被覆
正則表現の分解
intermediate extension
Fourier変換
するとSpringer fiberのコホモロジーが出てくる
\end{frame}

\begin{frame}
convolution代数としての$\Z[W]$の構成。

Lusztig
Steinberg多様体の上の構成可能関数とその合成石で定まる代数が$\Z[W]$と同型。

Fourier変換との関係は?
\end{frame}

\begin{frame}
Kazhdan-Lusztig多項式
$W$の群環とIwahori-Hecke代数
$T_w$を基底にもち$\Z[q^{1/2}, q^{-1/2}]$上生成される環で、積は
\begin{align*}
T_yT_w=T_{yw}& l(yw)=l(y)+l(w)\\
(T_s+1)(T_s-q)=0 & s\in S
\end{align*}
で乗法が定まる。$q \to 1$とすると

Kazhdan-Lusztig基底とKazhdan-Lusztig多項式
\begin{align*}
C_w=q^{-l(w)/2}\sum_{y\leq q}P_{y,w}T_y
\end{align*}
とすると、これが基底になって、involution $D$で不変。

Lie代数の表現の指標、Verma加群

Kazhdan-Lusztig予想
$w\in W$に対し、$w(\rho)-\rho$を最高weightに持つVerma module $M_w$と
$L_w$を最高weight加群とする。
この時、これらの指標の関係式がKazhdan-Lusztig多項式を用いて
\begin{align*}
ch(L_w)=\sum_{y\leq w}(-1)^{l(w)-l(y)}P_{y,w}(1)ch(M_w)
\end{align*}
と書ける。
\end{frame}

\begin{frame}{Kazhdan-Lusztig予想の証明}
Beilinson-BernsteinとBrylinski-Kashiwaraによる。

Bruhat分解$G=\coprod_{w\in W}BwB$とSchubert多様体$G/B=\coprod_{w\in W}X_w$

Beilinson-Bernstein localization
$\fg$の表現と旗多様体の$D$加群の対応
$\lambda$を整ウェイトで任意の$i\in I$について$\lambda(h_i)\in\Z_{>0}$を満たすとする。
この時$\Gamma(X,-):RH^0_l(D^\lambda_X) \to M(\fg)$は完全関手で
$B_w(\lambda)\mapsto M(w\circ\lambda)^*, M_w(\lambda)\mapsto M(w\circ\lambda), L_w(\lambda)\mapsto(w\circ\lambda)$を満たす。

Riemann-Hilbert対応
正則ホロノミック$D$加群とperverse sheafの対応
$Sol$が正則ホロノミック$D$加群をperverse sheafに移す。
$B_w(\lambda)\mapsto \C_{X^w}[-l(w)], M_w(\lambda)\mapsto D(\C_{X^w}[-l(w)],L_w(\lambda)\mapsto {}^\pi\C_{X^w}$となる。

Kazhdan-Lusztig多項式はSchubert多様体のintersectionコホモロジーを用いて
\begin{align*}
P_{y,w}(q)=\sum_iq^i\dim IH_{X_y}^{2i}(\overline{X}_w)
\end{align*}
と書ける(Kazhdan-Lusztig)

(柏原谷崎のKazhdan-Lusztig予想をめぐってを参考)
\end{frame}

\begin{frame}{Schubert多様体のintersection cohomologyの記述}
purityとdecomposition theoremを使う
\end{frame}

\begin{frame}{Soergel bimodule}
Kazhdan-Lusztigとの関係

categorification

Khovanovのtriply graded
\end{frame}

\begin{frame}
KhovanovのSpringer多様体。
Grassmannianとの関係、geometric Satake
\end{frame}

\begin{frame}
不変量の幾何的定義

結び目の図式からある多様体とその上の層の複体を定義する。

weight filtrationからspectre系列を作る。

$E_2$-pageが二重複体で、さらにここにweightでもう一つ次数が入って、三重次数複体。
\end{frame}
\end{document}