\documentclass{jsarticle}
\RequirePackage{amsmath,amssymb,amsthm, amscd, comment, multicol}
\usepackage[all]{xy}
\input{../tex/theorems}
\input{../tex/symbols}
\newcommand{\fC}{\mathfrak{C}}
\usepackage[dvipdfmx]{graphicx}
\title{Abelと楕円積分}
\author{@unaoya}
\date{\today}
\begin{document}
\maketitle
楕円積分の話
- Legendre relation
- レムニスケート周率
- 算術幾何平均
- 楕円関数の加法定理
- アーベルの定理

アーベルヤコビの定理
因子が関数の因子となる条件、周期積分

$1$と$\sqrt{2}$のAGMがレムニスケートになる話

\begin{align*}
\int^x_0\frac{dt}{\sqrt{1-t^4}}
\end{align*}
はレムニスケート積分。
これは楕円積分の特別な場合。

第一種完全楕円積分と第二種完全楕円積分
\begin{align*}
K(k)=\int^{\frac{\pi}{2}}_0\frac{dt}{\sqrt{1-k^2\sin^2t}}\\
E(k)=\int^{\frac{\pi}{2}}_0\sqrt{1-k^2\sin^2t}dt
\end{align*}

算術幾何平均との関係
$0<a<b$について
\begin{align*}
K(a,b)=\int^{\frac{\pi}{2}}_0\frac{dt}{\sqrt{a^2\cos^2t+b^2\sin^2t}}
\end{align*}
となる。
$k=\dfrac{\sqrt{a^2-b^2}}{a}$とおけば
\begin{align*}
K(a,b)=\frac{1}{a}\int^{\frac{\pi}{2}}_0\frac{dt}{\sqrt{1-k^2\sin^2t}}
\end{align*}
となり、第一種完全楕円積分。

\begin{thm}
$a'=\dfrac{a+b}{2}, b'=\sqrt{ab}$とすると
\begin{align*}
K(a,b)=K(a',b')
\end{align*}
\end{thm}
Landen変換
\begin{align*}
\tan x=\frac{\sin2t}{k+\cos2t}
\end{align*}
を用いて計算する。

\section{Abel-Jacobiの定理}
Abel-Jacobi mapとは$C$をリーマン面で種数$g$とする。
$C \to Jac(C)$は周期積分を用いて定まる。
$Jac(C)$は$\C^g/\Lambda$で、$\omega_1,\ldots,\omega_g$が正則$1$形式の基底とし、
サイクル$\gamma$上の積分で生成される格子$\Lambda$を考える。
すると、$x\in C$に対し、$(\int_{x_0}^x\omega_i)$は$\mod \Lambda$で決まる。
積分の起点$x_0$を決めることで上の写像は定まる。

$Pic$と$Jac$の関係。

$C$が楕円曲線、すなわち$g=1$の場合を考える。
定義式が$y^2=f(x)$とかける時、正則$1$形式として$\frac{dx}{y}$を取ることができる。
これでの積分が楕円積分。
この積分の因子$D$での値が$0$になることと、$D$が$div(f)$と書けることが同値?(AJ mapの単射性)
全射性がJacobiの逆問題?

\section{Griffiths}
%https://publications.ias.edu/sites/default/files/legacy.pdf
%https://publications.ias.edu/sites/default/files/variationsonatheorem.pdf

\section{theta関数}
wikipedia
%https://en.wikipedia.org/wiki/Theta_function

Jacobiのtheta functionとは、$z\in\C$と$\tau\in H$についての関数
\begin{align*}
\theta(z,\tau)=\sum_{n\in\Z}\exp(\pi in^2\tau+2\pi inz)
\end{align*}
のこと。
$\exp$の周期性から、$z$について周期関数である、つまり
\begin{align*}
\theta(z+1,\tau)=\theta(z,\tau)
\end{align*}
となる。
また、
\begin{align*}
\theta(z+\tau,\tau)=\sum\exp(\pi in^2\tau+2\pi inz+2\pi in\tau)
\end{align*}

さらに$\theta_{00}=\theta, \theta_{01}, \theta_{10}, \theta_{11}$という関数を
\begin{align*}
\theta_{01}=\theta(z+\frac{1}{2},\tau)\\
\theta_{10}=\exp(\frac{1}{4}\pi i \tau+\pi i z)\theta(z+\frac{1}{2}\tau, \tau)\\
\theta_{11}=\exp(\frac{1}{4}\pi i \tau+\pi i(z+\frac{1}{2}))\theta(z+\frac{1}{2}\tau+\frac{1}{2},\tau)
\end{align*}
と定める。

$z=0$としたとき、これらは保形形式となり、例えばFermat曲線のパラメータ表示
\begin{align*}
\theta_{00}(0,\tau)=\theta_{01}(0,\tau)^4+\theta_{10}(0,\tau)^4
\end{align*}
を与える。

Jacobiの恒等式。
$\tau$についての保形性。
\begin{align*}
\theta_{00}(\frac{z}{\tau},-\frac{1}{\tau})=(-i\tau)^{\frac{1}{2}}\exp(\frac{\pi}{\tau}iz^2)\theta_{00}(z,\tau)\\
\theta_{01}(\frac{z}{\tau},-\frac{1}{\tau})=(-i\tau)^{\frac{1}{2}}\exp(\frac{\pi}{\tau}iz^2)\theta_{10}(z,\tau)\\
\theta_{10}(\frac{z}{\tau},-\frac{1}{\tau})=(-i\tau)^{\frac{1}{2}}\exp(\frac{\pi}{\tau}iz^2)\theta_{01}(z,\tau)\\
\theta_{11}(\frac{z}{\tau},-\frac{1}{\tau})=-i(-i\tau)^{\frac{1}{2}}\exp(\frac{\pi}{\tau}iz^2)\theta_{11}(z,\tau)
\end{align*}
これを使って、$\zeta$の函数等式を示せる。

$\wp$との関係
\begin{align*}
\wp(z,\tau)=-(\log(\theta_{11}(z,\tau)))''+c

$\eta$との関係
\begin{align*}
\theta_{10}(0,\tau)=\frac{2\eta^2(2\tau)}{\eta(\tau)}\\
\theta_{00}(0,\tau)=\frac{\eta^2(\frac{1}{2}(\tau+1))}{\eta(\tau+!)}\\
\theta_{01}(0,\tau)=\frac{\eta^2(\frac{1}{2}\tau)}{\eta(\tau)}
\end{align*}

Heisenberg群との関係。
Jacobi thetaはHeisenberg群のある離散部分群で不変。
theta representaionという。
Heisenberg群とは
%https://en.wikipedia.org/wiki/Heisenberg_group
\begin{align*}
\begin{pmatrix}
1 & a & c\\
0 & 1 & b\\
0 & 0 & 1
\end{pmatrix}
\end{align*}
のなす群。
$a,b,c\in\R$のものを$H_3(\R)$とかく。
\begin{thm}[Stone-von Neumann]
$H_3(\R)$のについて、
与えられた指標に対し、それを中心の作用とする既約ユニタリ表現が一意に存在。
\end{thm}
この表現の実現として\Schr\"odinger modelは二乗可積分函数の空間に、
theta rerpresentaionは上半平面上の正則関数の空間に、それぞれ実現する。
このLie代数$\mathfrak{h}$は
\begin{align*}
\begin{pmatrix}
0 & a & c\\
0 & 0 & b\\
0 & 0 & 0
\end{pmatrix}
\end{align*}
のなすLie代数。
これは量子力学における、位置、運動量作用素の交換関係と同じ関係を満たす。

Schr\"odinger表現。
一般には$H_{2n+1}$の$L^2(\R)$への表現。
$n=1$でやる。
$h\in\R$に対して、既約ユニタリ表現$\Pi_h$を$\psi(x) \mapsto e^{ihc}e^{ibx}\psi(x+ha)$で定める。

theta representation
%https://en.wikipedia.org/wiki/Theta_representation
$f(z)$を正則関数とし、$a,b$を実数とする。
\begin{align*}
(S_af)(z)=f(z+a)=\exp(a\partial_z)f(z)\\
(T_bf)(z)=\exp(i\pi b^2\tau+2\pi ibz)f(z+b\tau)=\exp(2\pi ibz+b\tau\partial_z)f(z)
\end{align*}
と定める。
これらはそれぞれ$\R$の表現を定め、交換関係は
\begin{align*}
S_aT_b=\exp(2\pi iab)T_bS_a
\end{align*}
となる。
これにより$H=U(1)\times\R\times\R$の表現を定める。
\begin{align*}
U_\tau(\lambda,a,b)f(z)=\lambda(S_aT_bf)(z)
\end{align*}
Hilbert空間にこれを実現する。
$H_\tau$をノルム
\begin{align*}
\abs{f}_\tau^2\int_\C\exp)\frac{-2\pi y^2}{Im\tau}\abs{f(+iy)}^2dxdy
\end{align*}
と定める。
これはHilbert空間であり、$H$が作用する。
%https://ncatlab.org/nlab/show/Heisenberg+group

Stone-von Neumann theorem
%https://en.wikipedia.org/wiki/Stone\UTF{2013}von_Neumann_theorem

Metaplectic group
%https://en.wikipedia.org/wiki/Metaplectic_group
Spの二重被覆を閉部分群にもつ。

Weil表現
Weil Representation, Howe Duality, and the Theta correspondence
Dipendra Prasad
%http://www.math.tifr.res.in/~dprasad/montreal.pdf
The Weil Representation
Charlotte Chan
%http://www-personal.umich.edu/~charchan/TheWeilRepresentation.pdf
Weil表現とHowe duality 松本久義
%http://www.sci.kumamoto-u.ac.jp/~narita/5_Matsumoto_weil_representation.pdf
重さ半整数の Siegel モジュラー形式と Jacobi 形式, 高瀬幸一 
%http://www.sci.kumamoto-u.ac.jp/~narita/6_Takase_Saito-Kurokawa_1.pdf
Reductive Dual PairとWeil表現
西山 享
%http://rtweb.math.kyoto-u.ac.jp/preprint/sss.pdf
HEISENBERG GROUPS, THETA FUNCTIONS AND THE WEIL REPRESENTATION
JAE-HYUN YANG
%https://arxiv.org/pdf/0905.1865.pdf

Representations of Heisenberg Groups
Matt Collins
%https://pdfs.semanticscholar.org/7582/4b4bc258254a42774f712d05abb15032bb35.pdf

Quadratic Reciprocity and the Sign of the Gauss Sum via the Finite Weil Representation
Shamgar Gurevich1, Ronny Hadani2, and Roger Howe3
%https://www.math.wisc.edu/~shamgar/QR-IMRN.pdf

リー群の表現から見たテータ関数* 落合啓之
%http://www.kurims.kyoto-u.ac.jp/~kyodo/kokyuroku/contents/pdf/1476-26.pdf

nLab
Jacobi theta function
%https://ncatlab.org/nlab/show/Jacobi+theta+function

nLab
theta function
%https://ncatlab.org/nlab/show/theta+function

Non-abelian theta functions
Arnaud Beauville
%https://math.unice.fr/~beauvill/conf/chennai.pdf

Abel積分の理論、土屋
%http://www.kurims.kyoto-u.ac.jp/~yanagida/198603T.pdf

The Theta group of a line bundle.
%https://www.math.ru.nl/~bmoonen/BookAV/ThetaGr.pdf
\end{document}