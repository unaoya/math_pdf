この文章では商集合の構成と基本的な性質、特に普遍性について説明する。
商集合というのは、与えられた集合$X$の要素の間に新しい同一視の基準を作ることにより定まる新しい集合のことである。

例えば、日本人全体の集合$X$の要素の間に、住んでいる都道府県が同じという基準を作ることにより
$X$を$47$個のグループに分けることができる。\footnote{ここでは外国に住んでいる人などは除く。}
このグループたちを要素とする集合がこの新たな基準により要素(つまり今は日本人)を同一視して得られる商集合$Y$である。
東京都民のなす集合や北海道民のなす集合など、日本人全体の集合$X$の部分集合$47$個を集めることでこの集合$Y$を作ることができる。
$Y$は$X$の部分集合$47$個を要素に持つ集合であり、いわば集合の集合である。
しかしこの$Y$の要素は都道府県であると言ってしまっても、おおよそその扱いに差がない。
このように、実際の集合としては異なってもある種の振る舞いが同じであり、
またその振る舞いからある意味で集合が決まってしまうという性質を説明するのが普遍性という概念である。

商集合は数学のいたるところに現れるものであるが、初めて見ると複雑な概念であるように感じられる。
そこでできるだけ具体的なイメージを持てるよう、例を紹介しながら説明するよう心がけた。
同値関係を導入した上で商集合を定義するという流れの説明がよくあると思うが、
同値関係という概念を定義するというのが少々抽象的に感じられると思う。
そこで、この文章では少し順番を変えてまずは集合の分割とそれによる商集合の定義を説明し、
そのあとで同値関係について説明した。

また、数学にとって基本的な構成であることが伝わるよう、最後に商集合を用いた整数と有理数の構成を説明した。

\section{準備}

ここでは逆像と冪集合という二つの基本的な概念について説明する。



\section{集合の分割}

まず商集合を作るため、集合の分割の概念を紹介する。
集合$X$の分割を一つ与えることで、$X$の商集合を一つ定めることができる。

\begin{dfn}[集合の分割]
  集合$X$の(添え字集合$I$を持つ)分割とは各$i\in I$に対する$X$の部分集合$X_i\subset X$(すなわち$X$の冪集合$P(X)$の要素$X_i\in P(X)$)の族$(X_i)/{i\in I}$であって、
  \begin{enumerate}
  \item $i\in I$に対して$X_i\neq\emptyset$
  \item $i,j\in I$が$i\neq j$ならば$X_i\cap X_j=\emptyset$
  \item $\bigcup_{i\in I}X_i=X$
  \end{enumerate}
  をみたすもの。
\end{dfn}

冒頭で述べた日本人全体の集合$X$に対して$I$を都道府県とし$X_i$を$i$県民全体とすることで$X$の分割が得られる。
次の例も集合の分割を与える。

\begin{eg}
  集合$X$をトランプのカード$53$枚の集合としよう。
  各カードを記号と数字の列$\heartsuit1$とか$\clubsuit12$とか$\diamondsuit8$などとあらわす。
  絵札$J,Q,K$は数字$11, 12, 13$としよう。
  またジョーカーは$Jo$とあらわすことにする。
  つまり
  \begin{align*}
    X=\{\heartsuit1,\ldots,\heartsuit13,\spadesuit1,\ldots,\spadesuit13,\diamondsuit1,\ldots,
    \diamondsuit13,\clubsuit1,\ldots,\clubsuit13,Jo\}
  \end{align*}
  の$53$個の要素からなる集合。

  例えばこの集合を記号によって次のように分割しよう。
  $I=\{\heartsuit,\spadesuit,\diamondsuit,\clubsuit,Jo\}$として、
  \begin{align*}
    X_\heartsuit&=\{\heartsuit1,\ldots,\heartsuit13\}\\
    X_\spadesuit&=\{\spadesuit1,\ldots,\spadesuit13\}\\
    X_\diamondsuit&=\{\diamondsuit1,\ldots,\diamondsuit13\}\\
    X_\clubsuit&=\{\clubsuit1,\ldots,\clubsuit13\}\\
    X_{Jo}&=\{Jo\}\\
  \end{align*}
  とする。
  これら$X_i$たちは全て$X$の部分集合であるから、$X_i\in P(X)$である。
  これは集合の分割の条件を満たしている。
  例えば$X_\heartsuit\cap X_\clubsuit=\emptyset$だし、
  \begin{align*}
    \bigcup_{i\in I}X_i&=
    X_\heartsuit\cup X_\spadesuit\cup X_\diamondsuit\cup X_\clubsuit \cup X_{Jo}\\
    &=X
  \end{align*}
  である。
\end{eg}

このように$X$をグループ分けを与えるのが$X$の分割という概念である。
このグループ分けにおいて、同じグループに属する要素を同じとみなすことで新しく集合を作る。
$X$の分割$(X_i)_{i\in I}$は各々が$P(X)$の要素なので、これらを集めて$P(X)$の部分集合$Y$を定める。
つまり、この集合$Y$の要素は$X$の部分集合であり、これらが$X$の分割を与える。
\begin{dfn}[分割により定まる商集合]
\end{dfn}

このような$X_i$は各$x\in X$に対してただ一つ存在することが分割の定義から従う。
\begin{proof}
ただ一つであることは、$x\in X_i$かつ$x\in X_j$であれば$x\in X_i\cap X_j$であるから、
特に$X_i\cap X_j\neq\emptyset$であり、一つ目の条件から$i=j$となる。
次に$x\in X$に対して$x\in X_i$となる$X_i$が存在することは、二つ目の条件から
$x\in\bigcup_{i\in I}X_i$だから、和集合の定義よりよい。
\end{proof}

これに対して写像$q:X\to Y$を$x$に対して$x\in X_i$であるものを対応させることで$q(x)=X_i$と定める。
このようにして定まる$q:X\to Y\subset P(X)$を商写像という。

\begin{eg}
  上のトランプのカードと記号による分割の例でいうと、商集合は
  \begin{align*}
    Y=\{X_\heartsuit,X_\spadesuit,X_\diamondsuit,X_\clubsuit,X_{Jo}\}
  \end{align*}
  という$5$つの要素からなる集合である。

  この時、商写像は$q(\heartsuit3)=X_\heartsuit, q(\clubsuit7)=X_\clubsuit$などのようにして定まる写像。    
\end{eg}

この商写像は次のような性質をもつ。
ここではまず具体例で説明しよう。

\begin{eg}
  $X$を上のトランプ$53$枚の集合とする。
  $c:X\to\{\mbox{赤},\mbox{黒},Jo\}$をカードの色を見るという写像とする。
  例えば$c(\heartsuit4)=\mbox{赤}, c(\spadesuit10)=\mbox{黒}, c(Jo)=Jo$である。
  これは商写像を経由する。
  可換図式

  $n:X\to\{1,2,3,4,5,6,7,8,9,10,11,12,13,Jo\}$をカードの数字を見るという写像とする。
  例えば$n(\clubsuit7)-7, n(\diamondsuit9)=9$などである。
  これは商写像を経由しない。
  可換図式
\end{eg}

一般的に述べると次のような命題になる。
\begin{eg}
  $X$の分割$\{X_i\}_{i\in I}$により定まる商集合を$Y=\{X_i\mid i\in I\}\subset P(X)$とする。
  写像$f:X \to Y$が次の条件を満たすとする。
  任意の$i\in I$と、$x\in X_i$に対し$f(x)$
\end{eg}


\section{写像による分割}

前節ではあらかじめ$X$が直接グループに分けて記述されており、それらをそのまま商集合
ここでは、写像を用いてグループ分けを定める方法を紹介する。
また、それにより商集合が定まることを見る。
このように述べると、商が

写像$f:X \to I$を考える。
この$f$での値が同じものを同一視することで、新しく集合$X/\sim_f$を作ろう。

まず$f$を用いて集合$X$を分割する。
$i\in I$に対して、$X_i=f^{-1}(i)$とする。
ここで$X_i=\emptyset$であるようなものは除外して、$I'\subset I$を考える。

これにより、$\{X_i\}_{i\in I'}$が添え字集合$I'$を持つ集合の分割となることが次のように証明できる。
\begin{enumerate}
\item
  $i\neq j$とする。$x\in X_i\cap X_j$とする。
  $x\in X_i=f^{-1}(i)$なので、$f(x)=i$である。
  また、$x\in X_j=f^{-1}(j)$なので、$f(x)=j$である。
  写像の行き先は一意的に定まるから、$i=j$である。
  これは矛盾。
\item
  $\bigcup_{i\in I}X_i=X$であること。
  $X_i\subset X$なので、それらの和集合$\bigcup_{i\in I}X_i\subset X$である。
  逆に$X\subset \bigcup_{i\in I}X_i$であることを示す。
  $x\in X$とする。
  $f(x)\in I$なので、$f(x)=i\in I$とする。
  このとき$x\in f^{-1}(i)=X_i$である。
  よって和集合の定義から$x\in\bigcup_{i\in I}X_i$である。
\end{enumerate}


\begin{eg}
  上の例で考えたトランプ$53$枚の集合$X$から
  集合$S=\{\heartsuit,\spadesuit,\diamondsuit,\clubsuit,Jo\}$への
  写像$s:X\to S$をそのカードの記号を対応させる写像として定める。
  例えば$s(\heartsuit4)=\heartsuit, s(\clubsuit9)=\clubsuit$などである。

  このとき、この写像からさだまる分割は前節の例で述べた分割に一致する。
  すなわち
  \begin{align*}
    X_\heartsuit&=s^{-1}(\heartsuit)=\{\heartsuit1,\ldots,\heartsuit13\}\\
    X_\spadesuit&=s^{-1}(\spadesuit)=\{\spadesuit1,\ldots,\spadesuit13\}\\
    X_\diamondsuit&=s^{-1}(\diamondsuit)=\{\diamondsuit1,\ldots,\diamondsuit13\}\\
    X_\clubsuit&=s^{-1}(\clubsuit)=\{\clubsuit1,\ldots,\clubsuit13\}\\
    X_{Jo}&=s^{-1}(Jo)=\{Jo\}\\
  \end{align*}
  である。
\end{eg}

したがって、この分割を用いて商集合を定めることができる。
この商集合の要素は$X_i$という集合だが、これは$i\in I$と思うこともできる。
(ただし$X_i=\emptyset$の場合を除いて。)

\begin{eg}
  この集合は
  \begin{align*}
    \{\heartsuit,\spadesuit,\diamondsuit,\clubsuit,Jo\}
  \end{align*}
  という集合であるかのように扱える。

  商写像
\end{eg}

\section{同値関係}
ここでは分割を与える別の方法として、同値関係を紹介する。
実際に商集合を導入する際には同値関係で述べられることが多い。
同値関係とは、集合の要素の間の関係で、「同じとみなす」というべき条件を抽象化したものである。

商集合と同値関係、写像からの誘導
与えられた集合から、その要素を同一視することで新しい集合を作る。
例、学校をクラス分け
写像があればそこから定義できる。

ここまでの話では、あらかじめ写像$f$があって(つまり$X$の要素に対して何らかのラベルをつけることができて)、
それに基づいて$X$の要素を同一視する基準を決めた。

別の方法で同一視する基準を記述できる。
それが同値関係というものである。

まずは同値関係の考え方を説明する。
普通の等号は$x=y$でこれが$x$と$y$が同じであるか否かを判定する。
これと同様に$x\sim y$で$x$と$y$が何らかの意味で同じか否かを判定することにしよう。

例えば集合$X=\{\heartsuit,\spadesuit,\diamondsuit,\clubsuit\}$という集合において、
要素$x,y\in X$に対する$=$が成立する場合を$\circ$とし、不成立な場合を$\times$として表にすると、
次のようなになる。

\begin{table}[htb]
  \begin{tabular}{|c||c|c|c|c|} \hline
    $=$ & $\heartsuit$ & $\spadesuit$ & $\diamondsuit$ & $\clubsuit$ \\ \hline \hline
    $\heartsuit$ & $\circ$ & $\times$ & $\times$ & $\times$ \\ \hline
    $\spadesuit$ & $\times$ & $\circ$ & $\times$ & $\times$ \\ \hline
    $\diamondsuit$ & $\times$ & $\times$ & $\circ$ & $\times$ \\ \hline
    $\clubsuit$ & $\times$ & $\times$ & $\times$ & $\circ$ \\ \hline
  \end{tabular}
\end{table}

次に、$x$と$y$の色が同じであることを$x\sim y$とあらわすことにしよう。
\begin{table}[htb]
  \begin{tabular}{|c||c|c|c|c|} \hline
    $\sim$ & $\heartsuit$ & $\spadesuit$ & $\diamondsuit$ & $\clubsuit$ \\ \hline \hline
    $\heartsuit$ & $\circ$ & $\times$ & $\circ$ & $\times$ \\ \hline
    $\spadesuit$ & $\times$ & $\circ$ & $\times$ & $\circ$ \\ \hline
    $\diamondsuit$ & $\circ$ & $\times$ & $\circ$ & $\times$ \\ \hline
    $\clubsuit$ & $\times$ & $\circ$ & $\times$ & $\circ$ \\ \hline
  \end{tabular}
\end{table}
となる。

別の例として集合$X=\{1,2,3,4,5,6,7,8,9,10,11,12,13\}$に$3$で割ったあまりが等しいことを$x\sim y$で表そう。
この場合を表にすると

このようにして、同じとみなすもの同士を集めて集合の分割を作る。
つまり$x\sim y$であるとき、またその時に限り$x\in X_i$かつ$y\in X_i$であるような分割を作ろう。
これはまず$x\in X$に対して$X_x=\{y\in X\mid y\sim x\}$としておく。
これは$X$の部分集合であり、$P(X)$の要素である。
このように記述できる$P(X)$の要素を集める。
ただし、異なる$x\in X$に対して$X_x=X_{x'}$となりうることに注意せよ。

この方法で必ず分割が作れるわけではない。
しかし無条件にこのような表を作ることはできない。
例えば次のような表を考えてみよう。
これから上のようにして分割を作ろうとしても、次のような不都合が生じる。

このことを踏まえて、同値関係の定義を次のように与える。
\begin{enumerate}
\item まず$x$は自分自身と同じとみなす。
\item 次に$x\sim y$なら$y\sim x$である。
\item 最後に$x\sim y$かつ$y\sim z$なら$x\sim z$である。
\end{enumerate}

通常の等号$=$は上の三つの条件を満たしている。

表で言うと、左上から右下に引いた対角線が全て$\circ$であるということが一つ目の条件に、
またこの対角線について右上と左下が線対称であるということが二つ目の条件に対応する。
(三つ目の条件を表で言うのは難しい?行列やグラフなど?)

以上のような条件を満たす表(正確には$X\times X$の部分集合)を同値関係という。
このようにして同値関係から分割を定め、商集合を定義できる。

商写像の記述。
$q:X \to P(X)$を$x\in X$に対して$q(x)=\{y\in X\mid x\sim y\}\in P(X)$として定める。
これらは一致するか交わりを持たないかのいずれかであり、

写像の誘導について。


ここまでは同値関係や写像から分割を定めるという説明をしたが、
逆に分割や写像から同値関係を定めることができることが以下のようにわかる。

\begin{prop}
  $X$の分割から$X$の同値関係を$x\sim y$であることを$x$と$y$が同じ$X_i$に属することと定義することで、
  これは同値関係になる。
\end{prop}

\begin{prop}
  写像$f:X\to Y$から$X$の同値関係を$x\sim y$であることを$f(x)=f(y)$と定める。
  これは同値関係になる。
\end{prop}


\section{誘導された写像}
上の構成でも紹介したが、
写像$f:X\to Y$が$x\sim x'$ならば$f(x)=f(x')$とする。
このときこれは商写像$q:X \to X/\sim$を経由する。
つまり、$g:X/\sim \to Y$が存在して$gq=f$となる。
また、このような$g$は一意的である。

$g$の存在。
$x\in X/\sim$とする。
$X/\sim$を$X$の分割から与えられるとして、$x$は$X$のある部分集合である。
この部分集合に属する要素は全て$y\sim y'$である。


\section{商集合の普遍性}
集合$X$の同値関係$\sim$による商集合は次の性質を持つことは上で述べた通り。
任意の集合$Y$と写像$f:X \to Y$であって$x\sim x'$ならは$f(x)=f(x')$を満たすものに対し、
一意的に写像$g:X/\sim\to Y$であって、$f(x)=g([x])$であるものがが存在する。

逆にこのような性質を持つ集合は$X/\sim$に限ることがわかる。
正確に言うと、次のようになる。
\begin{thm}[商集合の普遍性]
  $X$とその同値関係$\sim$に対し、集合$Z$及び$q:X \to Z$が$q(x)=q(x')$であり、
  さらに次の性質を持つとする。
  任意の集合$Y$と任意の写像$f:X \to Y$であって$x\sim x'$ならば$f(x)=f(x')$を満たすものに対し、
  ただ一つの写像$g:Z \to Y$であって$gq=f$であるものが存在する。

  このとき、$Z$は$X/\sim$と全単射であり、しかもその全単射は$X$からの射として一意的。
\end{thm}

\section{数の構成}
商集合の考え方を使って、整数全体の集合、有理数全体の集合を作ってみよう。
ただし、ここでは自然数全体の集合$\N=\{0,1,2,\ldots\}$と自然数の足し算は与えられているものとする。

\subsection{整数}
$\Z$だが、直積集合$\N\times\N$に次のような関係を定める。
$(n,m)\sim(n',m')$であることを$n+m'=n'+m$により定める。
この条件は自然数の足し算のみによって記述されていることに注意しよう。
これが同値関係であることを確かめる。
\begin{enumerate}
\item $(n,m)\sim(n,m)$であること。これは$n+m=n+m$よりよい。
\item $(n,m)\sim(n',m')$ならば$(n',m')\sim(n,m)$であること。
$\sim$の定義に戻ると$n+m'=n'+m$ならば$n'+m=n+m'$を示すことになる。
これは等号の性質よりよい。
\item $(n,m)\sim(n',m')$かつ$(n',m')\sim(n'',m'')$ならば$(n,m)\sim(n'',m'')$であること。
$\sim$の定義に戻ると$n+m'=n'+m$かつ
\end{enumerate}

よって、商集合が定まる。
この商集合がどのような集合か?
$[(n,m)]=n-m$と思うことで普通の整数と同じように扱うことができる。

足し算、掛け算、引き算ができる。

\subsection{有理数}

次に$\Q$を直積集合$\Z\times\Z$に次のような同値関係を定める。

足し算、掛け算、引き算、わり算ができる。

$[(n,m)]=\dfrac{n}{m}$と思うことで普通の有理数と同じように扱うことができる。
