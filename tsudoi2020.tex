\documentclass[dvipdfmx]{beamer}
\usetheme{metropolis}

\input{../tex/theorems}
\input{../tex/symbols}
\usepackage{bxdpx-beamer}
\usepackage{comment}

\title{Weil予想とエタールコホモロジー}
\author{梅崎直也@unaoya}
\date{2020年3月1日関東すうがく徒のつどい}
\begin{document}

\begin{frame}
  \maketitle
\end{frame}

\begin{frame}{合同ゼータ関数}
  定義
\end{frame}

\begin{frame}{Weil予想}
  予想の主張を述べる
\end{frame}

\begin{frame}{エタール基本群}
  ガロア理論について復習
  $\pi_1(X,\overline{x}), \pi_1(X,\bar{x})$
\end{frame}

\begin{frame}{fpqc降下}
  ベクトル束や層の貼り合わせについて復習
  降下データ$\phi:p_1^*M'\to p_2^*M'$
\end{frame}

\begin{frame}{エタール層}
  位相の説明、層の定義、茎
  $F$
\end{frame}

\begin{frame}{局所系}
  茎$F_{\bar{x}}$をとることで$\pi_1(X,\bar{x})$の表現が得られる。
\end{frame}

\begin{frame}{コホモロジー}
  導来関手、チェック
\end{frame}

\begin{frame}{体のコホモロジー}
  ガロアコホモロジーと一致すること。
  $H^i(specK,F)=H^i(Gal(K_x/K),M)$

  $G_m$のコホモロジー、高次の消滅
\end{frame}

\begin{frame}{曲線のコホモロジー}
  $H^i(X_{\bar{k}},F)$
\end{frame}

\begin{frame}{底変換}
  固有底変換と滑らか底変換
\end{frame}

\begin{frame}{高次元のコホモロジー}
  ファイブレーションとスペクトル系列
\end{frame}

\begin{frame}{有限性}
  構成可能性、$Rf_*, Rf_!$
\end{frame}

\begin{frame}{コホモロジー消滅}
  アフィン消滅、一般には$2d$次まで
\end{frame}

\begin{frame}{比較定理}
  複素多様体としての特異コホモロジーと自然に同型。
  $H^i(X,\Z/n)=H^i(X(\C),\Z/n)$
\end{frame}

\begin{frame}{Poincar\'e双対}
  $Rf^!$
\end{frame}

\begin{frame}{サイクル類}
  $H^{2d}(X)$とtrace map
\end{frame}

\begin{frame}{Lefschetz跡公式}
  $Tr$
\end{frame}

\begin{frame}{ここまででわかること}
  ベッチ数、関数等式、有理性
\end{frame}

\begin{frame}{six operations}
\end{frame}

\begin{frame}{リーマン予想類似}
  重さの概念。
  Deligneの証明。
\end{frame}


\end{document}
