\documentclass{jsarticle}
\setlength{\textwidth}{\fullwidth}
\setlength{\evensidemargin}{\oddsidemargin}
\RequirePackage{amsmath,amssymb,amsthm, amscd, comment, multicol}
\usepackage[all]{xy}
\input{../tex/theorems}
\input{../tex/symbols}
\usepackage[dvipdfmx]{graphicx}
\usepackage{tikz}
\usepackage{tkz-euclide}
\usetkzobj{all}
\usetikzlibrary{intersections, calc}



\title{Selberg trace formula}
\author{@unaoya}
\date{\today}
\begin{document}
\maketitle

$G=\SL_2(\R)$とし$\Gamma=\SL_2(\Z)$とする。
$K=SO(2)\subset G$とし$\Delta$を$G/K=\h$のLaplacianとする。
\begin{align*}
\sum_{\gamma\in\Prim(\Gamma)}M(\gamma)=\sum_{\lambda\in\spec\Delta}W(\sqrt{\frac{1}{4}-\lambda})
\end{align*}
$M$と$W$は適当な関数でFourier変換で結びつく。

$f$を$G$上の適当な関数とし、$G$の右正則表現$(R, L^2(\Gamma\backslash G))$への$f$の作用$R(f):L^2(\Gamma\backslash G)\to L^2(\Gamma\backslash G)$を
\begin{align*}
\phi\mapsto(x\mapsto\int_Gf(g)\phi(xg)dg)
\end{align*}
と定める。
つまり
\begin{align*}
R(f)\phi=\int_Gf(g)R(g)\phi dg
\end{align*}
と定義する。これは$G$-homである。

この作用を積分作用素の形で記述する。
\begin{align*}
(R(f)\phi)(x)&=\int_Gf(g)\phi(xg)dg=\int_Gf(x^{-1}g)\phi(g)dg\\
&=\int_{\Gamma\backslash G}\sum_{\gamma\in\Gamma}f(x^{-1}\gamma g)\phi(g)dg
\end{align*}
となる。
つまり
\begin{align*}
K_f(x,y)=\sum_{\gamma\in\Gamma}f(x^{-1}\gamma y)
\end{align*}
を核関数とする積分作用素である。
この時、関数解析の一般論から
\begin{align*}
\tr(R(f))=\int_{\Gamma\backslash G}K_f(x,x)dx
\end{align*}
となる。

Hecke作用?
traceのふた通りの表現。
指標群としての理解と共役類上の関数としての理解。
有限群の場合における類関数と指標の関係の類似?

上の話を次のように解釈する。
$H^o \subset H$をconvlutionにより積を定めたsmoothコンパクト台関数の環の両側$K$不変な関数のなす部分環とする。
これは可換環になる(一般には佐武同型?)
$V$を$G$の表現とし、$H$の$V$への作用を
\begin{align*}
\pi(\phi)v=\int_G\phi(g)\pi(g)vdg
\end{align*}
と定義する。
($\R$上のFourier変換と比較せよ)

\begin{align*}
K_\phi(x,y)=\sum_{\gamma\in\Gamma}\phi(x^{-1}\gamma y)
\end{align*}
と定義する。
\begin{prop}
\begin{align*}
(\rho(\phi)f)(x)=\int_{\Gamma\backslash G}K_\phi(x,g)f(g)dg
\end{align*}
特に$\rho(\phi)$は$L^2(\Gamma\backslash G)$のコンパクト作用素。
特に$\{\phi_i\}$は$\rho(\phi)$の固有関数からなる正規直交基底で$\rho(\phi)\phi_i=\mu_i\phi_i$として$\mu_i\to0$を満たす。
\end{prop}


\begin{thm}
$\phi$を適切なtest関数として
\begin{align*}
\tr\rho(\phi)=\sum_{\gamma\in\Gamma}\int_{Z(\gamma)\backslash G}\phi(g^{-1}\gamma  g)dg
\end{align*}
となる。ここで$Z(\gamma)=\{g\in G\vert g\gamma=\gamma g\}$
\end{thm}

$f_i$を$L^2(\Gamma\backslash\h)$の$\rho(\phi)$と$\Delta$の固有関数からなる直交基底とし、
\begin{align*}
K_\phi(z,w)=\sum\mu_i f_i(z)\overline{f_i(w)}
\end{align*}
とFourier展開する。
ここで$\mu_i$は$\rho(\phi)$の固有値。
これを積分することで以下を得る。
\begin{thm}
$\phi$が適切なtest関数とすると、$\rho(\phi)$はtrace classで
\begin{align*}
\tr\rho(\phi)=\int_{\Gamma\backslash\h}K_\phi(z,z)\frac{dx\wedge dy}{y^2}
\end{align*}
\end{thm}

$\rho(\phi)$はtrace classである。

Laplacianのresolventを積分核がGreen関数である積分作用素で書く。
このときLaplacianのresolventのtraceは関数解析の一般論から記述ある。
\begin{align*}
\tr R(\Delta, \lambda)
\end{align*}
これを$\Delta$の固有値の逆数の和でかけるか?

Green関数$G_\rho$を用いて
\begin{align*}
k_h(z,w)=\frac{1}{\pi i}\int^\infty_{-\infty}G_\rho(z,w)\rho h(\rho)d\rho
\end{align*}
と定義し、
\begin{align*}
L(h)f(z)=\int_{\h}k_h(z,w)f(w)d\mu(w)
\end{align*}
と定義する。

Green関数について補足。
微分作用素$D$について$Df=g$という微分方程式を考える。
$DG_x=\delta_x$となる関数$G_x$が存在するとすると、
$g(x)=\int\delta_x(y)g(y)dy$となるので、$D\int G_xg(y)dy=\int DG_xg(y)=\int \delta_xg(y)dy=g(x)$となり、
$\int G_xg(y)dy=f$が解となる。
このような$G_x$は$DG_x=\delta_x$の両辺をFourier変換することで求める。
熱方程式やPoisson方程式の解法を見よ。

\begin{prop}
$f$を適当な条件を満たすtest関数で$-\lambda=-(\rho^2+\dfrac{1}{4})$を固有値に持つ$\Delta$の固有関数とする。
この時
\begin{align*}
L(h)f=h(\rho)f
\end{align*}
\end{prop}

resolventのtraceとzeta関数の関係。
Riemann zetaやWeil予想の場合にどのようになっているか?

\section{Poisson和公式}
Poisson和公式は$G=\R, \Gamma=\Z$の場合と考えられる。
あるいは空間が$S^1$でLaplacianが$\frac{d^2}{dx^2}$の場合。

$\R/\Z$も群。
これの既約表現は$1$次元表現で$\chi_n$とかける。
\begin{align*}
\pi(f)\chi_n=\int_{\R}f(x)\chi_n(x)dx=\hat{f}(n)
\end{align*}
である。
よって$\pi(f)$の$L^2$への作用のtraceは$\sum_{n\in\Z}\hat{f}(n)$である。
これを軌道積分側(積分作用素で書く?)を書くと?

\section{Selberg trace formula}
これに対して$G=\SL(2,\R)$とその離散部分群、上半平面$\h$とそのLaplacianを考える。

$k(z,w)$を核関数に持つ積分作用素。
Green関数を使う。
これとLaplacianのresolventとの関係。
$k(z,w)$のLaplacianの固有関数によるFourier展開を考える。

\begin{thm}
$g$が適当な条件を満たすtest関数。
$h$をそのFourier変換、$\frac{1}{4}+t_i^2$が$\Delta$の$\Gamma\backslash\h$の固有値、
$\log(N)$が$\Gamma$のclosed geodesicの長さとする。
\begin{align*}
\sum_ih(t_i)=\frac{\vol(\Gamma\backslash\h)}{4\pi}\int^\infty_{-\infty}th(t)\tan h(\pi t)dt+\sum_N\frac{\log(N_0)}{N^{1/2}-N^{-1/2}}g(\log(N))
\end{align*}
この右辺の第二項も双曲線関数でかける、zetaのlog微分、軌道積分でもかける。
左辺は表現のtraceとしてかける。
\end{thm}


\begin{align*}
\sum^\infty_{n=0}m(\pi_{r_n})h(r_n)=\frac{\vol(\Gamma\backslash\h)}{4\pi}\int^\infty_{-\infty}rh(r)\tanh(\pi_r)dr+\sum_{\gamma\in\Gamma_{hyp}}\frac{\log(N(\gamma_0)}{N(\gamma)^{1/2}-N(\gamma)^{-1/2}}g(\log N(\gamma))
\end{align*}
ここで$g(u)$は$h(r)$のFourier変換。

ここで特に
\begin{align*}
h(r)=\frac{1}{r^2+(s-\frac{1}{2})^2}-\frac{1}{r^2+\beta^2}
\end{align*}
とすると(この関数はどこからきたのか?双曲線関数との関係?)
双曲元の寄与は
\begin{align*}
H(s)=\frac{d}{ds}\log(Z(s))
\end{align*}
となる。
(これとWeil予想との関係は?Weil予想の合同ゼータの計算とか、有理点の個数の誤差項と主要項の関係とか)

$\gamma\in\Gamma_{hyp}$にたいし$\Gamma$における$\gamma$の中心化群$\Gamma_\gamma$は巡回群であり、
その生成元を$\gamma_0$とした。

Weil予想の時と似たような計算?
\begin{align*}
\sum^\infty_{k=1}\sum_x\frac{\log x}{1-x^{-k}}x^{-ks}
\end{align*}
として
\begin{align*}
\frac{d}{ds}\log(Z(s))=\frac{Z'}{Z}
\end{align*}
と上の式の関係は?$\log N$が測地線の長さ。

双曲元の寄与の計算
\begin{align*}
\frac{\log(N(\gamma_0)}{N(\gamma)^{1/2}-N(\gamma)^{-1/2}}e^{-(s-\frac{1}{2})\log N(\gamma)}\\
=\frac{\log(N(\gamma_0))}{1-N(\gamma)^{-1}}N(\gamma)^{-s}
\end{align*}
と(Weylの?)指標公式の関係は?

$N(\gamma)=e^{\phi(\gamma)}$とすると、上の式は
\begin{align*}
\frac{\phi(\gamma)}{e^{\phi(\gamma)/2}-e^{-\phi(\gamma)/2}}e^{-\phi(\gamma)(s-\frac{1}{2}}
\end{align*}
となる。

\begin{align*}
\frac{1}{2\pi}\int_\R\tr(\pi_{ir}(f))N(\gamma)^{-ir}dr=e^{-(s-\frac{1}{2})\log(N(\gamma))}
\end{align*}
となるよう$\tr(\pi_{ir}(f))$を決める。
この積分はFourier変換。
\begin{thm}
$G$を半単純Lie群、$\Gamma\subset G$をココンパクト離散部分群とする。
$f\in C_c^\infty(G)$にたいし
\begin{align*}
\sum_{\pi\in\hat{G}}m_\Gamma(\pi)\tr(\pi(f))=\sum_{\gamma\in\Conj(\Gamma)}\vol(\Gamma_\gamma\backslash G_\gamma)I(\gamma,f)
\end{align*}
の両辺が収束し、等号が成立する。
\end{thm}

この公式を$G$が有限群の場合、$G=\R, \Gamma=\Z$の場合に考えてみる。


これを$G=\SL(2,\R)$で$\Gamma$が楕円元を含まず$f$が両側$K$不変な場合に計算していく。

$G$の既約ユニタリ表現$\pi$がclass oneもしくはsphericalとは$K=SO(2)$への制限が自明表現を持つこと。
\begin{prop}
$\pi$がclass oneとする。
このとき$f\in C^\infty_c(K\backslash G/K)$に対し$\pi(f)=0$となる。
\end{prop}

Selberg trace formulaの証明。

traceのふた通りの表示。
$R(f)$の$L^2(\Gamma\backslash G)$への作用を考える。
左辺は表現の既約分解を用いた計算。

テスト関数$f\in C^\infty_c(G)$の作用。
$(\pi, H_\pi)$を$G$の(既約ユニタリ?)表現とする。
$\pi(f):H_\pi \to H_\pi$を
\begin{align*}
v\mapsto\int_Gf(g)\pi(g)vdg
\end{align*}
で定める。

Fourier変換との関係。
$G=\R/\Z$の一次元表現$\chi(x)=\exp(2\pi x)$を$\pi$とすると、
\begin{align*}
\int_0^1f(x)\exp(2\pi x)dx
\end{align*}
となる。

(これを$G=\R, \Gamma=\Z$として解釈できる?)

有限群の場合にも同様。



これを軌道積分の計算を用いて計算する。
$G=\SL_2(\R)$の軌道積分の計算。
\begin{align*}
I(\gamma,f)=\int_{G_\gamma\backslash G}f(x^{-1}\gamma x)dx
\end{align*}
を軌道積分という。
ここで$\gamma, G_\gamma, f\in C^\infty_c(K\backslash G/K), dx$とする。

$G=NAK$と分解して、特に$\gamma=a_t\in A$を考える。
これに対し、簡単な行列の計算から$G_\gamma =A$とわかる。

\section{Selberg zeta}
Selberg zetaの定義と諸性質
\begin{dfn}
$s\in\C, Re(s)>1$に対し
\begin{align*}
Z_\Gamma(s)=\prod_{p \in\Prim(\Gamma)}\prod^\infty_{k=0}(1-N(p)^{-(k+s)})
\end{align*}
\end{dfn}

解析接続、零点と極、函数等式

これらをSelberg trace formulaを用いて証明する。
\section{応用}
\begin{thm}[素測地線定理]
$\Gamma\subset \SL(2,\R)$を$\vol(\Gamma\backslash\h)<0$なる離散部分群とし、
$\pi_\Gamma(x)=\sharp\{p\in \Prim(\Gamma)\mid N(p)\leq x\}$と定める。
これに対し
\begin{align*}
\pi_\Gamma(x)=\li(x)+\sum_{\frac{3}{4}<t_k<1}\li(x^{t_k})+O(x^{\frac{3}{4}}(\log x)^{-\frac{1}{2}})~(x\to \infty)
\end{align*}
が成り立つ。
ここで
\begin{align*}
li(x)=\int^x_2\frac{dt}{\log t}
\end{align*}
であり、$\lambda_k=t_k(1-t_k)$はLaplacian $\Delta$の$L^2(\Gamma\backslash\h)$の例外固有値
\end{thm}

\begin{thm}[数論的表示]
$\Gamma=\SL(2,\Z)$に対するSelbergゼータ$Z_\Gamma(s)$は
\begin{align*}
Z_\Gamma(s)=\prod_{d\in D}\prod^\infty_{k=0}(1-\epsilon_d^{-2(s+k)})^{h(d)}
\end{align*}
\end{thm}

この二つを合わせて、実二次体の類数の分布の公式が得られる。




\section{$\SL(2,\R)$の表現論}
$G$の上半平面への作用。
$i$の固定部分群が$K=\SO(2)$が極大コンパクト。

$G=NAK$と直積分解できる。
ここで
\begin{align*}
N=\{\begin{pmatrix}1&x\\0&1\end{pmatrix}\}, A=\{\begin{pmatrix}\sqrt{y}&0\\0&\sqrt{y}^{-1}\end{pmatrix}\}
\end{align*}

$G$上の両側不変測度として
\begin{align*}
\frac{dxdy}{y^2}\frac{d\theta}{2}=d\mu
\end{align*}
を取れる。

$K$の既約表現は、整数$n$に対して
\begin{align*}
\chi_n(k(\theta))=e^{in\theta}
\end{align*}

主系列表現。
$G=NAK$とし、$M=Z_A(K)=\{\pm E_2\}$とする。
$NAM$の表現を$N$については自明、$A, M$の指標$\chi_s, \chi_\epsilon$の積であるものを考え、これの誘導表現を$\pi_{\epsilon,s}$とする。
これは$G/NA\cong K$上の関数空間としての実現を持つ。
これについて調べていく。

$G$の表現$\pi$について、その$K$への制限の直和分解$\oplus \chi_n^{m_n}$を$\pi$の$K$-spectrumなどという。
また$\g$を$G$のLie代数とし、$\fZ\subset U(\g_\C)$を普遍包絡代数の中心とする。
$\pi$には$\g$の作用が$\dfrac{d}{df}\exp(tX)v\vert_{t=0}$として定まる。
$G=\SL(2,\R)$の場合、$\fZ$はCasimir元$C$で生成される。
これの作用の様子を調べる。

この二つが表現を調べるための手がかりとなる。

Plancherelの公式。
既約表現の分布の様子。Fourier変換のPlancherelの公式との関係は?



\section{Selberg zeta}
ラプラシアンの固有値の和が対数微分?
測地線の長さとの関係、オイラー積表示

上半平面の等長変換として$\SL(2,\R)$が作用する。
$\gamma\in\Gamma$で$z\in \h$をうつしたものたちの中で、双曲計量から定まる距離が正の最小となるもの。
つまり$z$と$\gamma z$を結んでできる最短測地線の長さ。
商がコンパクトを仮定すると存在?
この長さを$l_\gamma$と書くことにする。

素測地線とは。
他の元のベキで書けないもの。
双曲元$\gamma\in\Gamma_{hyp}\subset \Gamma$にたいしその中心化群$\Gamma_\gamma$は巡回群になるので、その生成元。

素元$\gamma$に対し
\begin{align*}
Z_\gamma(s)=\prod^\infty_{m=0}(1-\exp(l_\gamma(s+m))
\end{align*}
と定める。
これの対数微分
\begin{align*}
\frac{d}{ds}\log(Z_\gamma(s))&=\sum^\infty_{m=0}\frac{d}{ds}\log(1-\exp(l_\gamma(s+m)))\\
&=-l_\gamma
\end{align*}

これの表現論的意味とは?
\begin{align*}
\frac{1}{e^{t/2}-e^{-t/2}}
\end{align*}
双曲線関数?
円盤の双曲計量の極座標表示に双曲線関数が出てくる。


Poisson和公式とRiemann zetaの関係。
解析接続、関数等式、特殊値とか?

\section{まとめ}
関数解析、微分幾何的側面

表現論、数論






\begin{thm}
$X=\Gamma\backslash G$とする。
$L^2(X)$は$\Delta$の固有関数からなる基底を持つ。
$L^2(\Gamma\backslash G)$は$G$の既約許容表現の直和に分解する。
\end{thm}
$L^2(\Gamma \backslash G)$には$K$-fixed vectorを持たない既約表現が存在する。
これは正則保型形式を用いて構成できる。
正則離散系列表現。



\begin{dfn}[admissible representation]
$(\pi, V)$を$G$の表現とし$V(\rho)\subset V$を$K$が$\rho$で作用する部分とする。
admissibleとは任意の$\rho$で$V(\rho)$が有限次元なこと。
\end{dfn}

$G=\SL(2,\R)$の時には既約admissibleなら任意の$\rho$について$V(\rho)$はたかだか$1$次元



$G=\SL(2,\R), K=SO(2)$とすると$\h=G/K$である。
双曲計量(群から決まる?等長変換と$G$の関係)を適当に決めるとLaplacianが
\begin{align*}
\Delta=-y^2(\frac{\partial^2}{\partial x^2}+\frac{\partial^2}{\partial y^2})
\end{align*}

$\Gamma\subset G$を離散ココンパクトとし(どんな例がある?四元数とか?)
$X=\Gamma\backslash\h$はコンパクトリーマン面。
$\Delta$は$C^\infty(\Gamma\backslash\h)$に作用し、$L^2(\Gamma\backslash\h, \frac{1}{y^2}dxdy)$の作用に伸びる。
これは自己共役作用素なのでスペクトル分解定理が使える。

$G$の$C^\infty(G)$への作用から$\g$の$C^\infty(G)$への作用が
\begin{align*}
Xf(g)=\frac{d}{dt}f(ge^{tX})\vert_{t=0}
\end{align*}
により定まる。

$U(\g)$のcenterの生成元を
\begin{align*}
H=\begin{pmatrix}1&0\\0&-1\end{pmatrix},
L=\begin{pmatrix}0&0\\1&0\end{pmatrix},
R=\begin{pmatrix}0&1\\0&0\end{pmatrix},
-4\Delta=H^2-2RL+2LR
\end{align*}
により定める。
これとLaplacianとの関係は?

$\g$の作用を書いてみる。
$\exp(tH)=\begin{pmatrix}e^t&0\\0&e^t\end{pmatrix}$であり、
\begin{align*}
Hf(x)=\frac{d}{dt}(f(\begin{pmatrix}e^t&0\\0&e^t\end{pmatrix}x))\vert_{t=0}
\end{align*}
$\exp(tR)=\begin{pmatrix}1&e^t\\0&1\end{pmatrix}$である。
これらが上半平面を$x+iy$とかき、一次分数変換の作用により、$\frac{\partial}{\partial y}, \frac{\partial}{\partial x}$に対応する。

$\SL(2,\R)=NAK$と分解する。
\begin{align*}
N=\{\begin{pmatrix}1&x\\0&1\end{pmatrix}\}\\
A=\{\begin{pmatrix}\sqrt{y}&0\\0&\sqrt{y}\end{pmatrix}\}\\
K=\{\begin{pmatrix}\cos\theta&\sin\theta\\-\sin\theta&\cos\theta\end{pmatrix}\}
\end{align*}
$A$の$i$への作用は$i\mapsto yi$であり$N$の作用は$i\mapsto xi$である。

$G=\R$のとき。
$\g=\R$であり、$X\in\g$の作用は$f\mapsto f''$である。
これの固有関数は$e^{\pm i\lambda x}$である。
$\g$の作用から$\Delta$の作用が定まり、これの固有関数で$L^2$を展開する。

\section{Green関数について}
Green関数とは。
レゾルベントに対応する積分作用素。
$\delta$関数との関係。Fourier変換すると?
レゾルベントのFourier変換は?Laplace変換は前に出てきたきがする。

$G(z,w;\lambda)$をLaplacianのresolventの積分核、つまり
\begin{align*}
(\Delta+\lambda)^{-1}f(z)=\int G(z,w;\lambda)f(w)d\mu(w)
\end{align*}
なるものとする。
これは次のような性質を持つ。
\begin{align*}
(\Delta+\lambda)G(z,w;\lambda)=\delta(z,w)
\end{align*}
$G(z,w;\lambda)$は$z,w$についてはそれらの距離$d(z,w)$のみによって決まる。
$d(z,w)\to\infty$で$G(z,w;\lambda)\to0$である。

形式的に計算すると
\begin{align*}
f(z)=\int\delta(z,w)f(w)dw=\int(\Delta+\lambda)Gf(w)dw)
\end{align*}
となる。

$\R$上の場合、黒田本pp163, 221, 244のGreen作用素$G_\zeta=R(L,\zeta)$を見よ。

これを用いて
\begin{align*}
k(z,w)=\frac{1}{\pi i}\int^\infty_{-\infty}G_\rho(z,w)\rho h(\rho)d\rho
\end{align*}
と定義し、
\begin{align*}
Lf(z)=\int_{\h}k(z,w)f(w)d\mu(w)
\end{align*}
と定義する。

これを用いて
\begin{align*}
K(z,w)=\sum_{\gamma\in\Gamma}k(z,\gamma w)
\end{align*}
とする?
こうすれば$\Gamma$不変になるので$\Gamma\setminus\h$上の関数になる。

\begin{prop}
$f$を適当な条件を満たすtest関数で$-\lambda=-(\rho^2+\dfrac{1}{4})$を固有値に持つ$\Delta$の固有関数とする。
この時
\begin{align*}
Lf=h(\rho)f
\end{align*}



\end{prop}
微分方程式
\begin{align*}
W''(r)+\frac{1}{r}W'(r)+\frac{4\lambda}{(1-r^2)^2}W(r)=0
\end{align*}
を考える。
これは$r=0,1$で特異点を持つ。
これの回のうちで$r=0$に特異点を持つものを$g_\lambda$とする。
disc上の関数で極座標$(r,\theta)$にたいし$g_\lambda(r)$により定める。
また
\begin{align*}
g_\lambda(z,w)=g_\lambda(\abs{\frac{z-w}{z-\overline{w}}})
\end{align*}
で$\h\times\h$上の関数を定義する。
さらに
\begin{align*}
G_\lambda(z,w)=\sum_\gamma g_\lambda(z,\gamma(w))
\end{align*}
で定義する。

表現論的な意味は?
左$K$-invで$\Delta$の固有関数であるようなもの?

\section{話の流れ}
$\Delta$の固有値の和を考える。
$\Delta$は$U(\g)$の中心の生成元ということで一般化できる。
$\g$は$L^2(G)$に作用。



双曲計量と群作用の関係。
等長変換として作用する。
$S^1, S^2$の場合は?$G, K$は何になるか。
固有関数は三角関数や球面調和関数になるはず。



\section{有限群}
$G$が有限群の場合。
$(R_G,\C^G)$を考える。
\begin{align*}
R_G(f)v=\sum_{g\in G}f(g)R_G(g)v
\end{align*}
$v:G\to \C$なので
\begin{align*}
(R_G(f)v)(h)=\sum_{g\in G}f(g)v(hg^{-1})
\end{align*}
となる。

$G$が巡回群の場合。
積分作用素側は?

$G=S_n, D_n, G(\F_q)$などの場合にどのようになるか。
test関数$f$の取り方、$\Gamma$の設定とか、誘導関数での解釈とか、$G$の$G$への共役作用による解釈とか

\section{$L$関数と等分布}
\begin{align*}
\prod_{v\in\Sigma}\frac{1}{1-(Nv)^{-s}} 
\end{align*}
が$Re s>1$で収束し、$Re s\geq 1$で有理型、$s=1$で$1$位の極を持ち、それ以外には極も零点も持たないとする。
$\rho$を$G$の既約表現とし、$\chi$を指標とする。
\begin{align*}
L(s,\rho)=\prod_{v\in\Sigma}\frac{1}{\det(1-\rho(x_v)(Nv)^{-s})}
\end{align*}
が$Re s>1$で収束し、$Re s\geq 1$で有理型、$s=1$で$-c_\chi$位の極を持ち、それ以外には極も零点も持たないとする。

\begin{thm}
$\sharp\{v\in\Sigma, Nv\leq n\}$と$\dfrac{n}{\log n}$は$n\to \infty$で同じ。

$\chi$を$G$の既約指標とすると
\begin{align*}
\sum_{Nv\leq n}\chi(x_v)=\frac{c_\chi n}{\log n}+o(\frac{n}{\log n}), n\to\infty
\end{align*}
\end{thm}

さらに仮定として、ある$C$が存在して、任意の$n\in\Z$に対し$\sharp\{v\in\Sigma,Nv=n\}\leq C$とする。
このとき$\Sigma$の順序によらない。

$E/K$を楕円曲線とし、$F_v$の$T_\ell E$への作用の固有値を$\pi_v, \bar{\pi}_v$とする。
このとき$\abs{\pi_v}=\sqrt{Nv}$となる。
$0\leq\phi_v\leq\pi$を用いて、
\begin{align*}
\frac{\pi_v}{\sqrt{N_v}}=e^{i\phi_v}
\end{align*}
と書くことにする。

$G=SU(2)$とし、
\begin{align*}
G/\sim=X=\{\begin{pmatrix}e^{i\phi}&0\\0&e^{-\i\phi}\end{pmatrix},0\leq\phi\leq\pi\}
\end{align*}
とする。
$G$のHaar measureの$X$での像は$\dfrac{2}{\pi}\sin^2\phi d\phi$である。
(四元数を用いて解釈する)

$\phi:G \to GL_2(\C)$を自然表現とし$\phi_m=Sym^{\otimes m}\phi$とする。
\begin{align*}
x_v=\begin{pmatrix}e^{i\phi_v}&0\\0&e^{-\phi_v}\end{pmatrix}\in X
\end{align*}
とする。
これに対し
\begin{align*}
L(\rho_m,s)=\prod_v\prod_{a=0}^m\frac{1}{1-e^{i(m-2a}\phi_v(Nv)^{-s}}
\end{align*}

$X$を位相空間とし$C(X)$を$X$上の$\C$値連続関数全体とする。
\begin{align*}
\norm{f}=\sum_{x\in X}\abs{f(x)}
\end{align*}
とする。
$\delta_x:C(X) \to \C; f\mapsto f(x)$をDirac measureとし、
$\{x_n\}$に対して
\begin{align*}
\delta_n=\frac{1}{n}\sum_{i=1}^n\delta_{x_i}
\end{align*}
と定める。

\begin{dfn}
$\mu$を$C(X)$上のRadon measureすなわち連続写像$C(X)\to\C$とする。
$\{x_n\}$が$\mu$-equidistributedとは
任意の$f\in C(X)$に対し$n\to \infty$で$\mu_n(f) \to \mu(f)$となること、すなわち
\begin{align*}
\lim_{n\to\infty}\frac{1}{n}\sum^n_{i=0}f(x_i)=\mu(f)
\end{align*}
となること
\end{dfn}

$G$をコンパクト群とし$X=G/\sim$を共役類とする。
\begin{prop}
$X$の元の列$\{x_n\}$が$\mu$-equidistributedであることと、
任意の$G$の既約指標$\chi$について
\begin{align*}
\lim_{n\to\infty}\frac{1}{n}\sum^n_{i=1}\chi(x_i)=\mu(f)
\end{align*}
となることが同値。
\end{prop}
Peter-Weylの定理?

\subsection{Wiener-Ikehara}


\end{document}