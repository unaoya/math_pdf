\documentclass[dvipdfmx]{beamer}
\input{../tex/theorems}
\input{../tex/symbols}
\usepackage{bxdpx-beamer}
\title{導来代数幾何入門}
\author{梅崎直也@unaoya}
\date{2019/3/29 第3回関東すうがく徒のつどい}
\begin{document}

\begin{frame}
\maketitle
\end{frame}

\begin{frame}
\begin{enumerate}
\item 導入
\item 代数幾何
\item 導来代数幾何
\item BenZvi-Francis-Nadler
\item 応用
\end{enumerate}
\end{frame}

\begin{frame}{目標}
\end{frame}

\begin{frame}{代数幾何におけるファイバー積}

スキーム論は一般の環に対して空間を構成する、またそれの貼り合わせ
相対的な議論を扱える枠組み
例、高校数学の線束、mod p、無限小変形
テンソル積が大事

$\spec A_1\otimes_BA_2\simeq\spec A_1\times_{\spec B}\spec A_2$
\end{frame}

\begin{frame}{derived stack}
affine derived stackとその貼り合わせ(どの圏ではり合わせるか?)
\end{frame}

\begin{frame}{mapping stack}
$\Sigma$が位相空間や単体的集合の時、internal hom $X^\Sigma=Map(\Sigma,X)$がderived stackとして定まる。
\end{frame}

\begin{frame}{classifying space}
\end{frame}

\begin{frame}{derived loop stack}
$LX=X^{S^1}=Map(S^1,X)$はinternal homで定める。
$LX \simeq X\times_{X\times X}X$である。

$X$が位相空間から定まるconstant stackの場合、$LX$は通常のloop spaceから定まるconstant stack

$X=BG$のとき$LX=LBG=G/G$

$X$がsmooth scheme over char $0$ fieldの時は$T_X[-1]$
\end{frame}

\begin{frame}{$Mod_A$の定義}

まず$Vect$を定義する。
一般にabel圏から$\infty$圏を構成する方法がLurieのHAにある。
これがmonoidal

Gaitsgory
$AssocAlg(Vect))^{op} \to DGCat_{cont}, A \mapsto A-Mod$が定まる。
ここでVectは$\infty$-cat of chain complexes of $k$-vecctor spaces
$DGCat_{cont}$は$1-Cat^{St, cocmpl}_{cont}$における$Vect$-moduleたちのなす$\infty$-cat
Gaitsgory1.10.1
HAでのアーベル圏$A$から$(\infty,1)$-cat $D^-(A)$を作り、
これのright completionを作る。

LurieのHAでの扱いは?

$A$-modの圏から直接作るのと一致する?
\end{frame}

\begin{frame}{$QC(X)$の定義}
$X=\spec A$がaffine derived schemeの時、$QC(X)=Mod_A$とする。

一般のderived stackについては、$X$をaffine derived stackのcolimitで書き、
同じ図式で$QC$のlimitを$\infty$-cat of $\infty$-catsでとる。

$X$がqcでaffine diagonalを持てば、cosimplical diagramのtotalizationでかける。

$O_X$-modのような作り方はできない?
\end{frame}

\begin{frame}{perfect stack}
\begin{dfn}
\begin{enumerate}
\item $A$をderived commutative ringとする。
$A$加群$M$がperfectとは、$Mod_A$のsmallest $\infty$ categoryでfinite colimitとretractでとじたものに属すること。
\item derived stack $X$に対し、$Perf(X)$は$QC(X)$のfull $\infty$-subcategoryであって、
任意のaffine $f:U \to X$への制限$f^*M$がperfect moduleであるものからなるもの。
\item derived stack $X$がprefect stackとは$QC(X)\cong IndPerf(X)$であること。
\item $f:X \to Y$がperfectとは、任意のaffine $U \to Y$について、$X \times_Y U$がperfectなこと。
\end{enumerate}
\end{dfn}

compactとdualizableとperfectの関係。
特に$X$がaffine diagonalを持つ時の$QC(X)$における同値性。
\end{frame}

\begin{frame}{base changeとprojection formula}
Gaitsgoryにも注意がある?

\begin{prop}[BFN, proposition 3.10]
$f:X \to Y$をperfectとする。
この時
\begin{enumerate}
\item $f_*:QC(X) \to QC(Y)$はsmall colimitと交換し、projection formulaを満たす
\item 任意のderived stackの射$g:Y' \to Y$に対し、base chage map $g^*f_* \to f'^*g'_*$は同値
\end{enumerate}
\end{prop}
\end{frame}

\begin{frame}
symmetric monoidal categoryとalgebraとmodule
\end{frame}

\begin{frame}
\begin{prop}[BFN, Proposition 4.6]
$X_1, X_2$ perfect, $\boxtimes:QC(X_1)^c\otimes QC(X_2)^c \cong QC(X_1\times X_2)^c$
\end{prop}
\begin{enumerate}
\item $\otimes$とpullbackはdualizableを保ち、$X=X_1\times X_2$がperfectなことから、外部積がcompactを保つ
\item $QC(X_1\times X_2)^c$が外部積で生成
\item projection formula
\end{enumerate}
により証明。
さらに
\begin{enumerate}
\item $Ind:st \to Pr^L$がsummetric monoidal
\item $IndQC(X)^c\simeq QC(X)$
\end{enumerate}
から、$\boxtimes:QC(X_1)\otimes QC(X_2) \simeq QC(X_1\times X_2)$が成立。
\end{frame}

\begin{frame}
\begin{thm}[BFNのTheorem 4.7]
$X_1, X_2, Y$がperfectの時、$QC(X_1 \times_Y X_2) = QC(X_1) \otimes_{QC(Y)}QC(X_2)$
\end{thm}

$Y$が一般の時の証明の方針(どこに$Y$がperfectを使う?)
\begin{enumerate}
\item $QC(X_1\times_YX_2)=Mod_{T_{geom}}(QC(X_1\times X_2))$ by Barr-Beck
\item $QC(X_1)\otimes_{QC(Y)}QC(X_2)=Mod_{T_{alg}}(QC(X_1\times X_2))$ by Barr-Beck
\item $T_{alg}=T_{geom}$ by base change
\end{enumerate}
\end{frame}

\begin{frame}
self-duality
$Fun$のやつ
\end{frame}

\begin{frame}
上を一般化。
4.1の話

これをどう使うか
\end{frame}

\begin{frame}
\begin{thm}[BFNのTheorem 4.14]
$X, Y$ derived stack with affine diagonal、$f:X \to Y$をperfectとする。
$g:X' \to Y$は任意のderived stackの射とする。
この時、$QC (X \times_Y X') \simeq Fun_Y(QC(X),QC(X'))$は$\infty$圏の同値
\end{thm}
\begin{enumerate}
\item 関手の構成$M\mapsto \tilde{f}_*(M\otimes\tilde{g}^*-)$とする。
$\tilde{f}$がperfectなのでcolimitを保ち$QC$に移る。
またprojection formulaにより$QC(Y)$線形になる。

\item $X'$についてlocalなので($\times, \lim, \colim, QC$の交換関係)、affineに帰着する。
$QC(X\times_Y\spec A) \simeq Fun_Y(QC(X),Mod_A)$を示す。

\item $Y=\spec B$の時。
前の系4.8から$QC(X)$は$Mod_B$上self dualで、前の命題4.13から$QC$と$\otimes$の交換がわかるので
$Fun_B(QC(X),Mod_A)\simeq Fun_B(Mod_B,QC(X)^\vee\otimes_BMod_A)\simeq QC(X)\otimes_BMod_A$
$QC(X\times_B\spec A)\simeq QC(X)\otimes_BMod_A$
と計算できる。

\item $Y$が一般の時。
\end{enumerate}
\end{frame}

\begin{frame}{応用}
symmetric monoidal $\infty$-cats of qcsとs$\infty$-cats of linear endofunctors with monoidal str by compositionのcenterとtraceと$E_n$-analogueを計算する。
\begin{enumerate}
\item Hecke category
\item TFT
\end{enumerate}

\end{frame}

\begin{frame}{Hecke category}
$X \to Y$に対して$D(X\times_YX)$
特に$BB \to BG$に対して$X\times_YX=B\backslash G/Y$

Hecke categoryはHecke algebraのcategorification
\end{frame}

\begin{frame}{affine Hecke category}
cf. Bezrukavnikov

$H^{aff}_G$を$St_G=\tilde{G}\times_G\tilde{G}$上の$G$同変準連接層のなす$\infty$-categoryとする。
ここで$\tilde{G} \to G$はGrothendieck-Springer resolutionで$G$は簡約群。
\end{frame}

\begin{frame}
$St_G=\tilde{G}\times_G\tilde{G}$とする。
$Z(QC(X\times_YX)) \simeq QC(LY)$を$X=\tilde{G}/G \to Y=G/G=LBG$に適用することで
$Z(H^{aff}_G)=Z(QC(St_G))\simeq Z(QC(X\times_YX))\simeq QC(LY)\simeq QC(LLBG)\simeq QC(Loc_G(T^2))$となる。
\end{frame}

\begin{frame}{finite Hecke algebra}
cf. BN2でやった
coherent $D$-moduleの圏?$D(B\backslash G/B)$のDrinfeld centerと$G$上の指標層の圏の同一視。
さらに指標層のLanglands双対
BN2のTheorem1.8
$H_G$と$\tilde{H}_G$はsemi-rigidでcanonical pivotal, CY strを持つ。
冪単指標層のなすdg圏$Ch_G$は$H_G$のmonoidal centerおよび$\tilde{H}$のmonoidal traceに標準的に同型。

さらにKoszul双対から$\tilde{H}^{per}_G$と$H^{per}_{G^{\vee}}$が同値なことが言えて、
これにより上の定理の系として、Langlands dualのtwo-periodic dg cat of unip ch shが同値なことが言える。
\end{frame}

\begin{frame}{TFT}
\begin{dfn}
\end{dfn}

\begin{prop}
perfect stack $X$に対し 2d TFT $\exists Z_X$ s.t. $Z_X(S^1)=QC(LX), Z_X(\Sigma)=\Gamma(X^\Sigma, O_{X^\Sigma})$
\end{prop}
証明すべきことは?

Costelloのcategorified analogue。
$X$に対する仮定なしに構成できる。
\end{frame}

\begin{frame}{Deligne-Kontsevich conjecture}
monoidal stable categoryのDerinfeld centerはassociative (or $A_\infty$)-algのHochshild cohomologyのcategorical analogueである。
Deligneの予想は、Hochshild cochain complexはGerstenhaber algebranの持ち上げである$E_2$-algebraの構造を持つこと。
これのcyclic versionとして、Frobeinus algebraのHochshild cochainはframed $E_2$ (or ribbon) algebraの構造を持つ。
さらにKontsevichはこの高次版として、$E_n$-algebraのHochshild cochainは$E_{n+1}$-algebraの構造を持つことを予想。

CostelloとKontsevich-Soibelmanはalgebra $A$に対して$Z_A(S^1)=HC(A)$となるようなTFTから予想が従うことを説明した。

これの圏論類似として、monoidal $\infty$-categoryのDrinfeld centerが$E_2$-categoryであること。
\end{frame}

\begin{frame}{参考文献}
\begin{itemize}
\item BFN:
\item BN2:
\end{itemize}
\end{frame}

\end{document}