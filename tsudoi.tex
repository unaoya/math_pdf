\documentclass[dvipdfmx]{beamer}
\usetheme{metropolis}
%\usecolortheme{beaver} 

\input{../tex/theorems}
\input{../tex/symbols}
\usepackage{bxdpx-beamer}
\usepackage{tikz-cd}
\usepackage[all]{xy}
\RequirePackage{amsmath,amssymb,amsthm, amscd, comment, multicol}

%https://tex.stackexchange.com/questions/178800/creating-sections-each-with-title-pages-in-beamers-slides
\AtBeginSection[]{
  \begin{frame}
  \vfill
  \centering
  \begin{beamercolorbox}[sep=8pt,center,shadow=true,rounded=true]{title}
    \usebeamerfont{title}\insertsectionhead\par%
  \end{beamercolorbox}
  \vfill
  \end{frame}
}

\newcommand{\QC}{\mathrm{QC}}
\newcommand{\spec}{\mathrm{Spec}}
\newcommand{\Mod}{\mathrm{Mod}}

\title{導来代数幾何入門}
\author{梅崎直也@unaoya}
\date{2019/3/29 第3回関東すうがく徒のつどい}
\begin{document}

\begin{frame}
\maketitle
\end{frame}

\begin{frame}{はじめに}
この講演の内容は、いくつかの解説と論文のイントロの非常に粗い位相での貼り合わせです。
参考にしたものは最後に参考文献として一覧にしてあります。

講演者は証明や正確な定義をフォローしていません。\\
(次回作に期待)

スライドは公開しています。
(twitter @unaoya)
\end{frame}

\begin{frame}{目次}
\tableofcontents
\end{frame} 

\begin{frame}{今日の内容}
代数幾何における空間に対して、適切な極限や余極限の操作を与える枠組みが欲しい。
通常のスキームは$Alg_k \to Set$という関手で適切な条件を満たすものである。
これを拡張して、導来スキームを関手$dAlg_k \to sSet$であって適切な条件を持つものとして定める。

これらはホモトピーを取り入れた構造を持つ圏であり、
ホモトピーを考慮した適切な極限操作を定義することができる。

応用上の動機として、例えばup to equivalenceで物事を分類するような問題を考えたい。
例えば導来圏の対象で適切な条件を持つものを分類する。
\end{frame}

\section{代数幾何とホモトピー}

\begin{frame}{この節の目標}
代数幾何においてホモトピーを考えたくなる状況。

\begin{enumerate}
\item 関手が完全でないとき、導来関手を考える。複体のホモトピーを用いて定義する。
\item スキームの交点を考えると、重複度が現れる。「重複度付きの空間」を直接扱いたい。
\item up to equivalenceなモジュライ理論を考える。例えば導来圏の対象を分類するなど。
\item ループ空間を考える。$S^1$から$X$への射を分類する空間を構成したい。
\end{enumerate}

\end{frame}

\begin{frame}{homotopy limit and homotopy colimit}
通常の極限や余極限はホモトピーと相性が悪い。

\begin{multicols}{2}
\[
\begin{CD}
\ast@<<<\ast\\
@AAA@AAA\\
\ast@<<<\ast\amalg\ast
\end{CD}
\]

\[
\begin{CD}
S^1@<<<[0,1]\\
@AAA@AAA\\
[0,1]@<<<\ast\amalg\ast
\end{CD}
\]
\end{multicols}


\begin{multicols}{2}
\[
\begin{CD}
\ast@>>>\ast\\
@VVV@VVV\\
\ast@>>>S^1
\end{CD}
\]

\[
\begin{CD}
\Z@>>>\ast\\
@VVV@VVV\\
\R@>>>S^1
\end{CD}
\]
\end{multicols}

これを克服するためにホモトピー極限とホモトピー余極限という概念を定義する。
\end{frame}

\begin{frame}{代数幾何におけるファイバー積}
$k$を可換環とし$Alg_k$を可換な$k$代数の圏とする。

$k$代数$A$に対して空間$\spec A \to \spec k$を構成する。
これは$B \mapsto Hom_{k}(A,B)$により関手$Alg_k \to Sets$を与える。
これの貼り合わせが一般のスキーム$X$である。
これは相対的な議論$X\to S$を扱うための枠組みを与える。
\begin{multicols}{2}
\[
\begin{CD}
X_p@>>>X\\
@VVV@VVV\\
\spec\F_p@>>>\spec\Z
\end{CD}
\]

\[
\begin{CD}
X_0@>>>X\\
@VVV@VVV\\
\spec k@>>>\spec k[t]/t^2
\end{CD}
\]
\end{multicols}

スキームのファイバー積に対応する操作が代数のテンソル積。
\begin{align*}
\spec(A_1\otimes_BA_2)\simeq \spec A_1\times_{\spec B}\spec A_2
\end{align*}
\end{frame}

\begin{frame}{交点理論}
加群の複体のホモトピーとその極限、余極限も同様の問題。

$k\otimes_{k[x]}k\simeq k$だが$k\otimes_{k[x]}(k[x][-1]\oplus k[x])\simeq k[-1]\oplus k$となる。
\[
\begin{CD}
0@>>>k[x]@>{1\mapsto x}>>k[x]@>>>0\\
@.@VVV@VVV@.\\
0@>>>0@>>>k@>>>0
\end{CD}
\]
$Tor^1_{k[x]}(k,k)=k$であり、$k\otimes^L_{k[x]}k\simeq k[\epsilon_{-1}]=k\oplus k[-1]$とup to homotopyで定まる。
\[
\begin{CD}
\spec B\otimes_AC@>>>\spec B\\
@VVV@VVV\\
\spec C@>>>\spec A
\end{CD}
\]
ではなく、直接$\spec A\otimes^L_BC$を幾何的な対象としたい。
ただし$A\otimes^L_BC$はup to homotopyでしか決まらないことに注意。
\end{frame}

\begin{frame}{変形理論}
$A$の微分加群は$\Omega^1_{A/k}=I/I^1, I=\ker(A\otimes_kA \to A)$であり、
$X$の接空間は$Hom_k(\spec k[t]/t^2, X)$であった。
ホモトピー極限を用いて、$X$に対し、
\[
\begin{CD}
LX@>>>X\\
@VVV@V\Delta VV\\
X@>\Delta >>X\otimes X
\end{CD}
\]
と定義すると$LX=\spec Sym_{O_X}(L_X[1])$となる。

この$L_X$はcotangent complexと呼ばれるもので、$X$の変形をコントロールする複体。

$A$がsmooth $k$-algebraなら$L_A\simeq \Omega^1_{A/k}$であり、$\pi_0(L_A)=\Omega^1_{\pi_0A}$となる。
\end{frame}

\begin{frame}[fragile]{stack}
スキーム$X$は集合に値を持つ層$Alg_k \to Set$を定める。
例えば$A$に対し可逆元全体の集合$A^\times$を対応させるものは$\G_m=\spec k[x,x^{-1}]$で表現される。

このとき層$F$の貼り合わせ条件は、$S$の被覆$U_\bullet \to S$に対して次が完全であること。
\[
\begin{tikzcd}
   F(S) \arrow[r]
    & F(U) \arrow[r, shift left]
        \arrow[r, shift right]
& F(U\times_SU)
\end{tikzcd}
\]

例えば$G$-torsor全体を分類する空間$BG$を考える。
つまり関手$BG(S)=\{$S$\text{上の}$G$\text{-torsor全体}\}$を考える。
同型類を適切に処理するために、SetではなくGrpd(全ての射が同型である圏のなす圏)に値を持つ層を考える。
このようなものをstackという。

stack $F$の貼り合わせ条件はコサイクル条件を考えて
\begin{tikzcd}
   F(S) \arrow[r]
    & F(U) \arrow[r, shift left]
        \arrow[r, shift right]
& F(U\times_SU) \arrow[r]
\arrow[r, shift left=2] \arrow[r, shift right=2]
        & F(U\times_SU\times_SU) 
\end{tikzcd}

SetはGrpdに離散的なものとして埋め込める。
\end{frame}

\begin{frame}[fragile]{higher stack}
GrpdのnerveをとることでsSet(単体的集合のなす圏)が定まる。
単体的集合は、おおよそ点や線分、三角形、四面体などを適切に貼り合わせたものとしてイメージしておく。

これに広げることでより広いmoduli問題を考えることができる。
特にup to euivalenceで分類したい場合がある。
例えば$S$上の適切な条件を満たす層の複体を分類したい場合など。

このような問題を考えるためにhigher stackをsSetに値を持つ層として定める。
貼り合わせ条件は高次のコサイクル条件を考えなければいけない。

\begin{tikzcd}
  F(S) \arrow[r]
    & F(U) \arrow[r, shift left]
        \arrow[r, shift right]
& F(U\times_SU) \arrow[r]
\arrow[r, shift left=2] \arrow[r, shift right=2]
& F(U\times_SU\times_SU) \arrow[r, shift left=3]
\arrow[r, shift left=1] \arrow[r, shift right=1] \arrow[r, shift right=3]
        & \cdots
\end{tikzcd}

\end{frame}


\begin{frame}{ループ空間}
空間$X$に対して$S^1$から$X$への連続写像全体を適切に位相空間と思ったものがループ空間$LX$である。

ホモトピー論におけるループ空間
$Map(S^1,X)=Map(B\Z,X)$
$S^1\simeq B\Z\simeq \ast\amalg^h_{\ast\amalg^h\ast}\ast$とできる。

代数幾何においてループ空間を作る。
$B\Z$はスタックとしては定義できるがmapping stackは自明なものになってしまう。

derived mapping stackを考える。
つまり$T \mapsto Map(T\times M,X)$ではなく$T\mapsto Map(T\times^hM,X)$を考える。

ループ空間は$X \times^h_{X\times X}X$として定めることができる。
\end{frame}

\begin{frame}{ループ空間とChern指標}
$LBG=Map(S^1,BG)=G/G$となる。
これは$S^1$上の$G$-torsorを考えると、貼り合わせが$e$をどこにはり合わせるかで決まること、
$G$-torsorの同型が$G$同変であることとtorsorの作用を考えると、同型を与えるのが$e\mapsto h$としたとき、
$hg'=gh$となる。

$V/X$と$\gamma:S^1\to X$に対し、
\[
\begin{CD}
\gamma^*V@>>>V\\
@VVV@VVV\\
S^1@>\gamma>>X
\end{CD}
\]
のモノドロミーのtraceを対応させることで、$Ch(V)\in O(LX)^{S^1}$が定まる。
これが$Ch:K(X) \to O(LX)^{S^1}=H^{ev}_{DR}(X)$を与える。

特に$X=BG$とすると、$LX=LBG=[G/G]$であり、$V$は$G$の表現、$O(LX)^{S^1}=\C(G/G)$は類関数で、
$Ch$は表現のtraceである。
\end{frame}

\begin{frame}[fragile]{この節のまとめ}

\[\begin{tikzcd}[column sep=small]
Alg_k \ar[rr, "\rm{scheme}"] \ar[rrd, "\rm{stack}"] \ar[rrdd, left, "\rm{higher stack}"']& & Sets \ar[d]\\
& & Grpd \ar[d]\\
& & sSets 
\end{tikzcd}\]

ループ空間を正しく定義するためには、空間のホモトピー極限が必要である。
このもとで
\begin{align*}
LX=Map(S^1,X)=X\times^h_{X\times X}X
\end{align*}
と定義できるはず。
\end{frame}

\section{derived stack}

\begin{frame}[fragile]{この節の目標}

\[\begin{tikzcd}[column sep=small]
Alg_k \ar[dd] \ar[rr, "\rm{Sch}"] \ar[rrd, "\rm{St}"] \ar[rrdd, left, "\rm{hSt}"']& & Sets \ar[d]\\
& & Grpd \ar[d]\\
dAlg_k \ar[rr, "\rm{dSt}"']& & sSets 
\end{tikzcd}\]

右側をssetにするとmoduli problemをより広いものを扱うことができる。
例えば導来圏の対象を分類する、導来圏を分類するなどup to equivで分類したい場合などに必要。

左側をderiveすると「正しく」極限をとることができ、「正しい」空間を定義できる。

両側にホモトピーが定まっていて、それについて整合的な関手。
\end{frame}

\begin{frame}{derived topology}
(higher) stackは$Alg_k \to sSet$で層になるものだった。
これを拡張して$dAlg_k \to sSet$で層になるものとしてderived stackを定義する。

$dAlg_k$は例えば可換dg $k$代数の圏。
$dg$代数とは$\oplus_iA_i$で次数$-1$の射$d$であって$d^2=0$なるもの。

層を定義するためには位相が必要。
\begin{dfn}
$dAlg_k^{op}$にderived \'etale topologyを以下で定める。
$\{A \to B_i\}_i$が\'etale coveringとは、
$\{\pi_0(A) \to \pi_0(B)\}$が通常の可換環として\'etale coveringであり、$\pi_nA \otimes_{\pi_0A}\pi_0B_i \to \pi_nB_i$が同型。
\end{dfn}

これはinfinitesimal liftingで特徴付けることもできる。
\end{frame}

\begin{frame}[fragile]{derived stack}
層は前層であって、貼り合わせ条件を満たすもの。

\begin{dfn}
derived stackとは関手$F:dAlg_k \to sSet$であって、weak equivalenceを保ち、次のdescent条件を満たす。

任意のetale h-hypercovering $B_\bullet \to A$に対して$F(A) \to holim F(B_\bullet)$が$Ho(sSet)$における同値

\begin{tikzcd}
  F(A) \arrow[r]
    & holim (F(B) \arrow[r, shift left]
        \arrow[r, shift right]
& F(B\otimes^L _AB) \arrow[r]
\arrow[r, shift left=2] \arrow[r, shift right=2]
 &\cdots)
\end{tikzcd}
\end{dfn}
\end{frame}

\begin{frame}{derived affine stack}
$\R \spec:dAlg_k \to dSt_k$が定まり忠実充満。(derived Yoneda)
これは$A \mapsto (B \mapsto Map(A,B))$で定める。

\begin{align*}
\R \spec B \times^h_{\R \spec A}\R \spec C \simeq \R \spec(B \otimes^L_AB)\\
Map(F,G):H\mapsto Map(F\times^hH,G)
\end{align*}
などとして、internal homやholimが定まる。

一般のderived stackはaffine derived stackのcolimitでかける。

定義域を制限することで$t_0:dSt \to St$が定まり、afiineを貼り合わせることで$i:St \to dSt$が定まる。
$t_0(\R \spec A)=\spec \pi_0(A)$となる。
また$it_0X \to X$は閉埋め込みで、$X$と$t_0(X)$のsmall etale siteは一致する。
しかし$i$はholimやMapを保たず、derived tangentやderived fibered productは真にderivedな情報を含む。
\end{frame}

\begin{frame}{derived mapping stack}
$Map_{dSt_k}(F,G):H \mapsto Hom_{dSt_k}(F\times^hH, G)$と定める。
これが$dSt_k$におけるinternal homである。

$\Sigma$が位相空間や単体的集合の時、internal hom $X^\Sigma=Map(\Sigma,X)$がderived stackとして定まる。
ここで$\Sigma$はconstant stack。

このとき$i:St_k \to dSt_k$は$Map$と交換しない、
つまり$iMap(F,G)\simeq \R Map(iF, iG)$となるとは限らない。

一方で$t_0:dSt_k \to St_k$とは交換する。
つまり$t_0\R Map(F,G) \simeq Map(t_0F, T_0G)$となる。
特に$F, G$が$St(k)$から来るとき、$t_0\R Map(iF, iG)\simeq Map(F,G)$である。($t_0iF\simeq F$であることに注意)
つまりderived mapping stackはmapping stackを太らせたもの。

mapping spaceがずれる例として、次のloop stackの例を見る。
\end{frame}

\begin{frame}{derived loop stack}
$LX=X^{S^1}=Map(S^1,X)$はinternal homで定める。
$LX \simeq X\times^h_{X\times X}X$である。

$X$が位相空間から定まるconstant stackの場合、$LX$は通常のloop spaceから定まるconstant stack

stackとしての$Map(B\Z, X)$は$X$そのものになるが、
derived stackとしての$Map(B\Z, X)=X\times^h_{X\times^hX}X$となる。
$\ast \times_{\A^1}\ast\simeq k[\epsilon_{-1}]$の計算

$X=BG$のとき$LX=LBG=G/G$

$X$がsmooth scheme over char $0$ fieldの時は$T_X[-1]$
\end{frame}

\begin{frame}{cotangent complex}
schemeのcotangent complex, $\Omega^1$との関係、変形理論

derived ring $D_i=\R \spec k[\epsilon]=\R \spec (k\oplus k[i])$とする。
degree 0と-iにある。

このとき$Ext^i_k(L_{X,x},k)\simeq\R Hom_*(D_i,(X,x))$となる。
$Ext^i$はderived stackにおいては表現可能。

derived tangent stackを$TX=Map(\spec k[\epsilon], X)$とする。

$Y$がschemeなら$TiY\simeq \R \spec_Y(Sym_{O_Y}L_Y)$となる。

$Vect_n(X)$は非自明なderived extensionを持つ。
$\R Vect_n(X)$とする。
\end{frame}

\begin{frame}{この節のまとめ}
derived affine stack $\R \spec A$とその貼り合わせでderived stack $X$が得られる。
これは関手$X:dAlg_k \to sSet$であって、ホモトピーを保ち、貼り合わせ条件を満たすもの。

この枠組みにおいて、
\begin{enumerate}
\item ループ空間$LX$
\item cotangent complex $L_X$
\item 交点積$X \times^h_X$
\end{enumerate}
が正しく定義できる。
\end{frame}

\section{$\QC(X)$と積分変換}

\begin{frame}{この節の目標}
まずderived stack $X$上の層の圏$QC(X)$を定義する。
またある種の有限性条件を満たすものとしてperfect stack $X$を定義する。

この下で積分変換の圏が準連接層の圏と同値になることをみる。
$X \to Y, X' \to Y$に対して
\begin{align*}
\QC(X\times_YX') &\simeq Fun_Y(\QC(X), \QC(X'))\\
K &\mapsto (F \mapsto (f_*(g^*F\otimes K)))
\end{align*}

積分変換は$X\times Y$上の核関数$K(x, y)$を用いて$Y$上の関数$f(y)$から$x$上の関数を定める。
\begin{align*}
K(x,y) \mapsto (f(y) \mapsto (x \mapsto \int_Yf(y)k(x,y)dy))
\end{align*}
\end{frame}

\begin{frame}{$\QC(X)$の定義}
一般にアーベル圏 $A$からその複体のなすdg圏$Ch(A)$を作り、
さらにそこから$\infty$圏$N_{dg}(Ch(A))$を定めることができる。
これを$Mod_A$とする。
$X=\spec A$がaffine derived schemeの時、$\QC(X)=Mod_A$とする。

一般のderived stackについては、$X$をaffine derived stackのcolimitで書き、
同じ図式で$\QC$のlimitを$\infty$-cat of $\infty$-catsでとる。

$X$がqcでaffine diagonal $\Delta:X \to X\times X$を持てば、cosimplical diagramのtotalizationでかける。
\end{frame}

\begin{frame}{perfect stack}
\begin{dfn}
\begin{enumerate}
\item $A$をderived commutative ringとする。
$A$加群$M$がperfectとは、$Mod_A$のsmallest $\infty$ categoryでfinite colimitとretractでとじたものに属すること。
\item derived stack $X$に対し、$Perf(X)$は$\QC(X)$のfull $\infty$-subcategoryであって、
任意のaffine $f:U \to X$への制限$f^*M$がperfect moduleであるものからなるもの。
\item derived stack $X$がprefect stackとは$\QC(X)\cong IndPerf(X)$であること。
\item $f:X \to Y$がperfectとは、任意のaffine $U \to Y$について、$X \times_Y U$がperfectなこと。
\end{enumerate}
\end{dfn}
\end{frame}

\begin{frame}{有限性条件}
compactとdualizableとperfectの関係。
stable $\infty$-category $C$の対象$M$について

\begin{enumerate}
\item compactとは$Hom_C(M,-)$がcoproductと交換すること。
\item dualizabuleとはある$M^\vee$と$u:1\to M\otimes M^\vee, \tau:M\otimes M^\vee \to 1$が存在して、
$M \to M \otimes M^\vee \otimes M \to M$が$\id_M$となるもの。
\end{enumerate}

$Vect/k$における有限次元ベクトル空間。
$V^\vee = \Hom(V,k)$とする。
$1\to V\otimes V^\vee$を対角行列、$V\otimes V^\vee \to 1$をtraceとすると、
上の条件を満たす。

特に$X$がaffine diagonalを持ちperfectなとき、$\QC(X)$においてdualizableとcompactとperfectは同値。
\end{frame}

\begin{frame}{base changeとprojection formula}

\begin{prop}[BFN, proposition 3.10]
$f:X \to Y$をperfectとする。
この時
\begin{enumerate}
\item $f_*:\QC(X) \to \QC(Y)$はsmall colimitと交換し、projection formulaを満たす
\item 任意のderived stackの射$g:Y' \to Y$に対し、base chage map $g^*f_* \to f'_*g'^*$は同値
\begin{multicols}{2}
\[
\begin{CD}
X'@>g'>>X\\
@Vf'VV@VfVV\\
Y'@>g>>Y
\end{CD}
\]

\[
\begin{CD}
\QC(X')@<g'^*<<\QC(X)\\
@Vf'_*VV@Vf_*VV\\
\QC(Y')@<g^*<<\QC(Y)
\end{CD}
\]
\end{multicols}
\end{enumerate}
\end{prop}
\end{frame}

\begin{frame}{$\otimes$と$\times$}
\begin{prop}[BFN, Proposition 4.6]
$X_1, X_2$ perfect, $\boxtimes:\QC(X_1)^c\otimes \QC(X_2)^c \cong \QC(X_1\times X_2)^c$
\end{prop}
\begin{enumerate}
\item $\otimes$とpullbackはdualizableを保ち、$X=X_1\times X_2$がperfectなことから、外部積がcompactを保つ
\item $\QC(X_1\times X_2)^c$が外部積で生成
\item projection formula
\end{enumerate}
により証明。
さらに
\begin{enumerate}
\item $Ind:st \to Pr^L$がsummetric monoidal
\item $Ind\QC(X)^c\simeq \QC(X)$
\end{enumerate}
から、$\boxtimes:\QC(X_1)\otimes \QC(X_2) \simeq \QC(X_1\times X_2)$が成立。
\end{frame}

\begin{frame}{$\otimes$と$\times$}
\begin{thm}[BFNのTheorem 4.7]
$X_1, X_2, Y$がperfectの時、$\QC(X_1 \times_Y X_2) = \QC(X_1) \otimes_{\QC(Y)}\QC(X_2)$
\end{thm}

$Y$が一般の時の証明の方針(どこに$Y$がperfectを使う?)
$X_1\times_YX_2 \to X_1\times Y^\bullet\times X_2$から$\QC(X_1\times_YX_2) \gets \QC(X_1\times Y^\bullet X_2)$を作る。
すでに証明したことから$\QC(X_1)\otimes \QC(Y)^\bullet\otimes \QC(X_2)$となり、
これのgeometric realizationで$\QC(X_1)\otimes_{\QC(Y)}\QC(X_2)$が計算できる。
\begin{enumerate}
\item $\QC(X_1\times_YX_2)=Mod_{T_{geom}}(\QC(X_1\times X_2))$ by Barr-Beck
\item $\QC(X_1)\otimes_{\QC(Y)}\QC(X_2)=Mod_{T_{alg}}(\QC(X_1\times X_2))$ by Barr-Beck
\item $T_{alg}=T_{geom}$ by base change
\end{enumerate}
\end{frame}

\begin{frame}{self-duality}
\begin{cor}[BFN, Corollary 4.8]
$\pi:X \to Y$ map of perfect stacksとする。
$\QC(X)$はself dual $\QC(Y)$-modである。
つまり$Fun_{QC(Y)}(QC(X),QC(X')) \simeq QC(X)\otimes_{QC(Y)}QC(X')$となる。
\end{cor}

\end{frame}


\begin{frame}{積分変換}
\begin{thm}[BFNのTheorem 4.14]
$X, Y$ dst with affine diagonal、$f:X \to Y$をperfect、$g:X' \to Y$は任意。
この時$\QC (X \times_Y X') \simeq Fun_Y(\QC(X),\QC(X'))$は同値。
\end{thm}
\begin{enumerate}
\item 関手の構成$M\mapsto \tilde{f}_*(M\otimes\tilde{g}^*-)$とする。
$\tilde{f}$がperfectなのでcolimitを保ち$\QC$に移る。
またprojection formulaにより$\QC(Y)$線形になる。

\item $X'$についてlocalなので($\times, \lim, \colim, \QC$の交換関係)、affineに帰着する。
$\QC(X\times_Y\spec A) \simeq Fun_Y(\QC(X),Mod_A)$を示す。

\item $Y=\spec B$の時。
前の系4.8から$\QC(X)$は$Mod_B$上self dualで、前の命題4.13から$\QC$と$\otimes$の交換がわかるので
$Fun_B(\QC(X),Mod_A)\simeq Fun_B(Mod_B,\QC(X)^\vee\otimes_BMod_A)\simeq \QC(X)\otimes_BMod_A$
$\QC(X\times_B\spec A)\simeq \QC(X)\otimes_BMod_A$
と計算できる。

\item $Y$が一般の時。
\end{enumerate}
\end{frame}

\begin{frame}{この節のまとめ}
\begin{enumerate}
\item derived affine scheme $X=\R \spec A$に対し$\QC(X)=Mod_A$を$\infty$圏として定義した。

\item 一般のderived scheme $X$に対し$\QC(X)$を$X=colim_i\R \spec A_i$のとき$Mod_{A_i}$の貼り合わせで定義した。
これは加群である。(stable symmetric monoidal category)

\item perfectというクラスを定義した。
有限性の条件

\item 積分変換のなす圏がファイバー積の$\QC$と同値であることを示した。
$X$が$Y$上perfectなとき
\begin{align*}
\QC(X\times_YX') &\simeq Fun_Y(\QC(X), \QC(X'))\\
K &\mapsto (F \mapsto (f_*(g^*F\otimes K)))
\end{align*}
\end{enumerate}
\end{frame}

\section{表現論とTFTへの応用}
\begin{frame}{応用}
\begin{enumerate}
\item Hecke category
\item 位相的場の理論
\end{enumerate}
\end{frame}

\begin{frame}{affine Hecke category}
$G$は簡約群。
$H^{aff}_G$を$St_G=\tilde{G}\times_G\tilde{G}$上の$G$同変準連接層のなす$\infty$-categoryとする。
ここで$\tilde{G} \to G$はGrothendieck-Springer resolutionで$\tilde{G}=\{(g,B), g\in B, B\text{はBorel}\}$とする。

$St_G=\tilde{G}\times_G\tilde{G}$とする。
$Z(\QC(X\times_YX)) \simeq \QC(LY)$を$X=\tilde{G}/G \to Y=G/G=LBG$に適用することで
$Z(H^{aff}_G)=Z(\QC(St_G))\simeq Z(\QC(X\times_YX))\simeq \QC(LY)\simeq \QC(LLBG)\simeq \QC(Loc_G(T^2))$となる。
\end{frame}

\begin{frame}{finite Hecke algebra}
$X \to Y$に対して$D(X\times_YX)$を考える。
特に$BB \to BG$に対して$X\times_YX=B\backslash G/B$となる。

Hecke categoryはHecke algebraのcategorification

またLoop spaceとしての解釈から$D(B\backslash G/B)\simeq Coh_{[(B\times B)/G]}(St^u/G)^{S^1}{loc}$として
affine Hecke catgooryとfinite Hecke categoryを結びつけることができる。

coherent $D$-moduleの圏$D(B\backslash G/B)$のDrinfeld centerと$G$上の指標層の圏の同一視。
さらに指標層のLanglands双対がある。
\end{frame}

\begin{frame}{TFT}
extended TFTとは$(\infty,2)$-catの間のsymmetric monoidal functor $Z:2Cob \to 2Alg$のこと。
$2Cob$は点を$0$対象、点の間の$1$次元bordismが$1$対象、$1$次元bordismの間の$2$次元bordismが$2$対象。
$2Alg$は代数が$0$対象、bimoduleが$1$対象、その間の射が$2$対象。

\begin{prop}
perfect stack $X$に対しextended 2d TFT $\exists Z_X$が
$Z_X(S^1)=\QC(LX), Z_X(\Sigma)=\Gamma(X^\Sigma, O_{X^\Sigma})$として定まる。
\end{prop}

\begin{align*}
Z_X((S^1)^{\amalg m})=QC((LX)^{\times m})\simeq QC(LX)^{\otimes m} = Z_X(S^1)^{\otimes m}
\end{align*}
となり、symmetric monoidalになる。

特に$X=BG$の場合が数理物理的にも興味を持たれる。
\end{frame}


\begin{comment}
\begin{frame}{Deligne-Kontsevich conjecture}
monoidal stable categoryのDerinfeld centerはassociative (or $A_\infty$)-algのHochshild cohomologyのcategorical analogueである。
Deligneの予想は、Hochshild cochain complexはGerstenhaber algebranの持ち上げである$E_2$-algebraの構造を持つこと。
これのcyclic versionとして、Frobeinus algebraのHochshild cochainはframed $E_2$ (or ribbon) algebraの構造を持つ。
さらにKontsevichはこの高次版として、$E_n$-algebraのHochshild cochainは$E_{n+1}$-algebraの構造を持つことを予想。

CostelloとKontsevich-Soibelmanはalgebra $A$に対して$Z_A(S^1)=HC(A)$となるようなTFTから予想が従うことを説明した。

これの圏論類似として、monoidal $\infty$-categoryのDrinfeld centerが$E_2$-categoryであること。
\end{frame}
\end{comment}

\begin{frame}{参考文献}
\begin{itemize}
\item D. Ben-Zvi, J. Francis and D. Nadler, Integral transforms and Drinfeld centers in derived algebraic geometry.
\item D. Ben-Zvi and D. Nadler, Loop Spaces and Connections.
\item D. Ben-Zvi and D. Nadler, The character theory of a complex group.
\item D. Ben-Zvi and D. Nadler, Loop Spaces and Langlands Parameters.
\item D. Ben-Zvi and D. Nadler, Loop Spaces and Representations.
\item B. To\"en, Higher and Derived Stacks: a global overview.
\item B. To\"en and G. Vezzosi, A note on Chern character, loop spaces and derived algebraic geometry.
\item D. Gaitsgory and N. Rozenblyum, A study in derived algebraic geometry
\item J. Lurier, Higher Algebra.
\end{itemize}
\end{frame}

\end{document}