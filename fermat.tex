\documentclass{jsarticle}
\RequirePackage{amsmath,amssymb,amsthm, amscd, comment, multicol}
\usepackage[all]{xy}
\input{../tex/theorems}
\input{../tex/symbols}
\usepackage[dvipdfmx]{graphicx}
\title{Fermat予想}
\author{@unaoya}
\date{\today}
\begin{document}
\maketitle
\begin{enumerate}
\item Cornell-Silverman-Stevens
\item  Darmon-Diamond-Taylor
\item conradのseminar http://math.stanford.edu/~conrad/modseminar/
\item 藤原先生
\end{enumerate}

\section{Darmonのsurvey}
特別な場合に簡略化した証明をする。
\begin{thm}
楕円曲線$E/\Q$が$5$でgood reductionでGalois表現が$E_5\cong X_0(17)_5$とする。
このとき$E$は保型的。
\end{thm}

$X=X_0(17)$について調べることで以下がわかる。
\begin{lem}
$X_0(17)$の$\mod 5$表現$\bar{\rho}_0\colon G_\Q\to GL_2(\F_5)$は以下をみたす。
\begin{enumerate}
\item $\det(\bar{\rho}_0)$は円分指標$\bar{\epsilon}\colon G_\Q\to \F_5^\times$
\item $\bar{\chi}_2$を位数$4$の不分岐指標として
\[
\bar{\rho}_0\vert_{D_5}\cong\begin{pmatrix}\bar{\chi}_1&*\\0&\bar{\chi}_2\end{pmatrix},
\bar{\rho}_0\vert_{I_5}\cong\begin{pmatrix}\bar{\epsilon}&*\\0&1\end{pmatrix}
\]
\item 非自明指標$\bar{\Psi}\vert_{I_{17}}$により
\[
\bar{\rho}_0\vert_{D_{17}}\cong\begin{pmatrix}\bar{\epsilon}&\bar{\Psi}\\0&1\end{pmatrix}
\]
\item $\bar{\rho}_0$は全射
\end{enumerate}
\end{lem}

\begin{dfn}
$A$を完備ネーター局所$\Z_5$代数で剰余体が$\F_5$なるものとする。
$\bar{\rho}_0$の変形$\rho\colon G_\Q\to GL_2(A)$が許容的とは以下をみたすこと。
\begin{enumerate}
\item $\det{\rho}$が円分指標$\epsilon\colon G_\Q\to\Z_5^\times\subset A$
\item 
\[
\bar{\rho}_0\vert_{D_5}\cong\begin{pmatrix}\chi_1&*\\0&\chi_2\end{pmatrix},
\bar{\rho}_0\vert_{I_5}\cong\begin{pmatrix}\epsilon&*\\0&1\end{pmatrix}
\]
\item 
\[
\bar{\rho}_0\vert_{D_{17}}\cong\begin{pmatrix}\epsilon&\Psi\\0&1\end{pmatrix}
\]
\end{enumerate}
\end{dfn}
\begin{prop}
$E/\Q$が$5$等分点で保型的なら$T_5E$はadmissible
\end{prop}
このことから以下を証明すればよい
\begin{thm}
全ての許容的な$\bar{\rho}_0$の変形は保型的
\end{thm}
保型的な変形と許容的な変形の間の全単射を構成するが、そのままでは無限集合になる。
変形の係数環と分岐条件を決めて有限集合の比較をする。
言い換えると$MD_\Sigma(A)\to AD_\Sigma(A)$が各$A, \Sigma$について全単射であることを示す。
もし$AD_\Sigma$および$MD_\Sigma$が表現可能関手であればそれらを表現する間の同型を示せばよい。
\begin{thm}[Mazur]
$AD_\Sigma$は有限生成局所$\Z_5$代数で剰余体が$\F_5$である$R_\Sigma$により表現される
\end{thm}
$T_5X\in AD_\Sigma(\Z_5)$なのでこれに対応する射$\pi_{R_\Sigma}\colon R_\Sigma\to \Z_5$が存在する。

$MD_\Sigma$の表現可能性をあらかじめ示すことはできないが、その候補を定義する。
$N_\Sigma=17\prod_{p\in\Sigma}p^2$とし、それに対応するHecke環を$T(\Sigma)$とする。
これは$\ell\nmid N_\Sigma$なる$T_\ell$と$q\in\Sigma\cup\{17\}$なる$U_q$で生成される。
$f$を$X$に対応する正規固有形式とし$T(\Sigma)$のeigenformをinductiveに
\[
f_\emptyset=f, f_{\Sigma\cup\{q\}}=f_\Sigma(\tau)-a_qf_\Sigma(q\tau)+qf_\Sigma(q^2\tau)
\]
と定義する。
さらにイデアル$m_\Sigma\subset T(\Sigma)$を$5,T_\ell-a_\ell, U_q, U_{17}+1$で生成されるものとし、
$T_\Sigma$を$T(\Sigma)$の$m_\Sigma$による完備化とする。
これは有限平坦局所$\Z_5$代数で剰余体が$\F_5$である。
さらに$\Z_5$代数の射$T_\Sigma\to O$に対して、
$\Gamma_0(N_\Sigma)$の$5$進固有形式で$f_\Sigma$と$\mod 5$で合同なるものが対応する。
特に$f_\sigma$に対応する$\pi_{T_\Sigma}\colon T_\Sigma\to \Z_5$が存在する。
\begin{thm}[Eichler-Shimura, Carayol]
\end{thm}

$T(\Sigma)$と$X_0(N_\Sigma), J_0(N_\Sigma)$の関係。

上の定理と$R_\Sigma$の普遍性から$\phi_\Sigma\colon R_\Sigma\to T_\Sigma$が定義される。
これは$\pi$と整合的。
目標は以下を示すこと。
\begin{thm}
$\phi_\Sigma$は同型
\end{thm}

全射性は比較的容易に示せる。

\subsection{可換環論}
$C$を有限生成完備局所$\Z_p$代数$A$と全射$\pi\colon A\to\Z_p$の組$(A,\pi)$のなす圏とする。
$\Phi_A=\ker\pi/(\ker\pi)^2, \eta_A=\pi(Ann_A\ker\pi)$とする。
\begin{thm}
$R, T\in C$で$T$は有限生成ねじれ自由$\Z_p$加群であり、$\phi\colon R\to T$を全射とする。
$\abs{\Phi_R}\leq\abs{\Z_p/\eta_T}<\infty$ならば$\phi$は同型。
\end{thm}

\begin{thm}
$\phi\colon A\to B$を$C$の全射で$B$は完全交差とする。
$\Phi_\phi\colon\Phi_A\to\Phi_B$が同型で、これらが有限なら$\phi$は同型。
\end{thm}
\begin{proof}
$B$は完全交差なので$C$の全射$\nu_B\colon U=\Z_p[[X_1,\ldots,X_n]]\to B$で$\ker\nu_B=(f_1,\ldots,f_n)$なるものが取れる。
$b_i\in\ker\pi_B$に対し$\Phi_\phi$が同型から$a_i\in\ker\pi_A$で$\phi(a_i)=b_i$なるものが取れる。
これを使って$\nu_A\colon\Z_p[[X_1,\ldots,X_n]]\to A$を$X_i\mapsto a_i$として定めると中山の補題より$\nu_A$は全射。
$a_i$の定義から$\ker\nu_B\supset\ker\nu_A$である。
逆が言えるか?
$\Phi_{\nu_A}\colon\Phi_U\to\Phi_A$のkernelの生成元を持ち上げて$g_1,\ldots,g_n\in\ker\nu_A$とする。
この時$M\in M_n(U)$であって$(g_1,\ldots,g_n)=(f_1,\ldots,f_n)M$なるものが$f_i$が$\ker\nu_B$の生成元であることから取れる。
この$M$が正則であることを言えばよい。
$(\bar{g_1},\ldots,\bar{g_n})=(\bar{f_1},\ldots,\bar{f_n})\bar{M}$で$\Phi_A\cong\Phi_B$であることから$\det\bar{M}\in\Z_p^\times$となりよい。
したがって$\nu_A\nu_B^{-1}\colon B\to U\to A$が定義でき、これが$\phi$の逆を与える。
\end{proof}

\subsection{$\Phi_{R_\Sigma}$と$\eta_{T_\Sigma}$}
$X=X_0(17), T=end(T_5(X))=\{Trace(f)=0\}$とする。
これは$\rank 3$の自由$\Z_p$加群で$G_\Q$の共役作用を持つ。
$A=T\otimes\Q_p/\Z_p$とする。これは$G_\Q$加群。

$\Sigma$を素数の集合で$\{5,17\}$を含まないとする。
これに対し$J_r\subset H^1(\Q_r,A)$を素数$r$に対して以下で定める。
\begin{enumerate}
\item $r\notin \Sigma\cup\{5,17\}$のとき$J_r=\ker(H^1(\Q_r,A)\to H^1(I_r,A))$
\item $r\in\Sigma$のとき$J_r=H^1(\Q_r,A)$
\item $J_{17}=\ker(H^1(\Q_{17},A)\to H^1(\Q_{17},A/A^\circ_{(17)}))$
\item $J_5=\ker(H^1(\Q_5,A)\to H^1(I_5,A/A^\circ_{(5)}))$
\end{enumerate}
さらに
\[
S_\Sigma(\Q,A)=\ker(H^1(\Q,A)\to\prod_rH^1(\Q_r,A)/J_r
\]
とする。
つまり$S_\Sigma(\Q,A)=\{s\in H^1(\Q,A)\vert s_r\in J_r\}$とする。

$R_\Sigma$は普遍変形環なので
$\rho^{\rm univ}\colon G_\Q\to GL_2(R_\Sigma)$が存在する。
対応する普遍コホモロジー類を$u_\Sigma\in H^1(\Q,M_2(\Psi_{R_\Sigma}))$とする。
$\det(u_\Sigma)=1$なので$u_\Sigma$の像は$T\otimes\Psi_{R_\Sigma}$に入る。
これを用いて$\phi_\Sigma\colon Hom(\Psi_{\R_\Sigma},\Q_5/\Z_5)\to H^1(\Q,A)$が定義できる。
\begin{prop}
$\phi_\Sigma$は$S_\Sigma(\Q,A)$への同型
\end{prop}

\begin{thm}
\[
\abs{\Z_5/\eta_{T_\Sigma}}=\prod_{q\in\Sigma}(q-1)(a_q^2-(q+1)^2)
\]
\end{thm}
$\Sigma$の大きさに関する帰納法で示す。
\begin{enumerate}
\item $\eta_\emptyset=\Z_5$であること。
\item $\Sigma'$を一つ素数を追加したものとし$\eta_{T'/T}=(q-1)(T_q^2-(a+1)^2)$となること。
\end{enumerate}

$\Lambda=T_5(J)\otimes_{T}T, \Lambda'=T_5(J')\otimes_{T'}T'$であり、

\subsection{判定法の条件をみたすこと}
$\Sigma=\emptyset$の場合に帰着できる。

$\Sigma=\emptyset$の場合、右辺は$1$なので、$S_\emptyset(\Q,A)$が自明であることを示す。

\begin{prop}
$\abs{S_\emptyset(\Q,A_5^*)}=1$
\end{prop}
\begin{proof}
まず$s\in S$とgood prime $q$について$s_q=0$を示す。
次に$H^1(\Q,A_5^*)\to H^1(K,A_5^*)$が単射であることを示す。
これを用いて証明。

$\bar{s}\in\Hom(G_K,A_5^*)$とみて、$L/K$を$\bar{s}$が$U=\Gal(L/K)$を経由する最大拡大とする。
$\bar{s}\colon U\to  A_5^*$が$0$であることを言えばよい。

$\tau\in\Gamma$を適切に固定。
$h\in U$を任意にとり、$h\tau\in\Gamma$が$\Frob_q$なる素数$q$を選ぶ。
すると$\tau$の取り方から$q$はgood primeであり、$s_q=0$となる。
特に$q$の上にある$K$の素点$Q$に対し$\bar{s}(\Frob_Q)=0$となる。
$\tau$の取り方から$K_Q/\Q_q$の剰余次数は$4$で、$\Frob_Q=(h\tau)^4=h^+\in U^+$となる。
$h$は任意のなので$\bar{s}$は$U^+$を消す。
一方、$U$ヘの$\tau\in G$の作用は$\tau$の定義から固有値$1$を持つ、つまり$U^+$は非自明。
$\bar{s}$は$G$同変なので$0\neq U^+\subset\ker\bar{s}\subset U$は$G$部分加群で、
$U$は既約$G$加群なので$\ker\bar{s}=U$、すなわち$\bar{s}=0$である。
\end{proof}

$X=X_0(17)$の$5$等分点への$G_{\Q_5}, G_{\Q_{17}}$作用、
つまりMLTで持ち上げたい$\mod 5$表現$\bar{\rho_0}$の定義。
$p=17$では$\F_5^2$に$\begin{pmatrix}\bar{\epsilon}&\bar{\Psi}\\0&1\end{pmatrix}$で作用。
$\bar{\Psi}$は分岐指標。
これの$End^0$に定まる表現を計算する。
$End^0$の基底を$\begin{pmatrix}1&0\\0&-1\end{pmatrix}, \begin{pmatrix}0&1\\0&0\end{pmatrix}, \begin{pmatrix}0&0\\1&0\end{pmatrix}$でとると、これらの移り先は
\begin{align*}
\begin{pmatrix}\bar{\epsilon}&\bar{\Psi}\\0&1\end{pmatrix}\begin{pmatrix}1&0\\0&-1\end{pmatrix}\begin{pmatrix}\bar{\epsilon}&\bar{\Psi}\\0&1\end{pmatrix}^{-1}
&=\begin{pmatrix}\bar{\epsilon}&-\bar{\Psi}\\0&-1\end{pmatrix}\begin{pmatrix}\bar{\epsilon}^{-1}&-\bar{\epsilon}^{-1}\bar{\Psi}\\0&1\end{pmatrix}\\
&=\begin{pmatrix}1&-2\bar{\Psi}\\0&-1\end{pmatrix}\\
\begin{pmatrix}\bar{\epsilon}&\bar{\Psi}\\0&1\end{pmatrix}\begin{pmatrix}0&1\\0&0\end{pmatrix}\begin{pmatrix}\bar{\epsilon}&\bar{\Psi}\\0&1\end{pmatrix}^{-1}
&=\begin{pmatrix}0&\bar{\epsilon}\\0&0\end{pmatrix}\begin{pmatrix}\bar{\epsilon}^{-1}&-\bar{\epsilon}^{-1}\bar{\Psi}\\0&1\end{pmatrix}\\
&=\begin{pmatrix}0&\bar{\epsilon}\\0&0\end{pmatrix}\\
\begin{pmatrix}\bar{\epsilon}&\bar{\Psi}\\0&1\end{pmatrix}\begin{pmatrix}0&0\\1&0\end{pmatrix}\begin{pmatrix}\bar{\epsilon}&\bar{\Psi}\\0&1\end{pmatrix}^{-1}
&=\begin{pmatrix}\bar{\Psi}&0\\1&0\end{pmatrix}\begin{pmatrix}\bar{\epsilon}^{-1}&-\bar{\epsilon}^{-1}\bar{\Psi}\\0&1\end{pmatrix}\\
&=\begin{pmatrix}\bar{\Psi}\bar{\epsilon}^{-1}&-\bar{\epsilon}^{-1}\bar{\Psi}^2\\\bar{\epsilon}^{-1}&-\bar{\epsilon}^{-1}\bar{\Psi}\end{pmatrix}
\end{align*}
となる。
従って$A_5$への$G_{\Q_{17}}$作用は、$1$次元ずつのfiltration $A^0_5\subset A^1_5\subset A_5$があって、
それぞれのgrに$\epsilon, 1, \epsilon^{-1}$で作用する。
\begin{prop}
$h_{17}=1$
\end{prop}
\begin{proof}
$A_5$への$G_{\Q_{17}}$作用は、$1$次元ずつのfiltration $A^0_5\subset A^1_5\subset A_5$があって、
それぞれのgrに$\chi, 1, \chi^{-1}$で作用する。
ここで$\chi$は円分指標。

$0\to A^0_5\to A_5\to A_5/A^0_5\to 0$からGaloisコホモロジーの長完全列をかくと、
\begin{align*}
0&\to H^0(\Q_{17}, A_5^0)\to H^0(\Q_{17},A_5)\to H^0(\Q_{17},A_5/A_5^0)\\
&\to H^1(\Q_{17}, A_5^0)\to H^1(\Q_{17},A_5)\to H^1(\Q_{17},A_5/A_5^0)\\
&\to H^2(\Q_{17}, A_5^0)\to H^2(\Q_{17},A_5)\to H^2(\Q_{17},A_5/A_5^0)\to 0\\
\end{align*}

$h_{17}=\abs{H^0(\Q_{17},A^*_5)}/[H^1(\Q_{17},A_5):\ker H^1(\Q_{17},A_5)\to H^1(\Q_{17},A_5/A_5^0)]$を求める。
上の完全列から
$0\to H^1(\Q_{17},A_5)/\ker\to H^1(\Q_{17},A_5/A_5^0)
\to H^2(\Q_{17}, A_5^0)\to H^2(\Q_{17},A_5)\to H^2(\Q_{17},A_5/A_5^0)\to 0$が完全なので
\[
[H^1(\Q_{17},A_5):\ker]=\frac{h^1(A_5/A^0_5)h^2(A_5)}{h^2(A^0_5)h^2(A_5/A_5^0)}
\]
となる。

双対性から$h^2(V)=h^0(V^*)$である?
\end{proof}

$p=5$では$\F_5^2$に$\begin{pmatrix}\bar{\epsilon}&\bar{\Psi}\\0&1\end{pmatrix}$で作用。
$\bar{\Psi}$は分岐指標。
これの$End^0$に定まる表現を計算する。
$End^0$の基底を$\begin{pmatrix}1&0\\0&-1\end{pmatrix}, \begin{pmatrix}0&1\\0&0\end{pmatrix}, \begin{pmatrix}0&0\\1&0\end{pmatrix}$でとると、これらの移り先は
\begin{align*}
\begin{pmatrix}\bar{\epsilon}&\bar{\Psi}\\0&1\end{pmatrix}\begin{pmatrix}1&0\\0&-1\end{pmatrix}\begin{pmatrix}\bar{\epsilon}&\bar{\Psi}\\0&1\end{pmatrix}^{-1}
&=\begin{pmatrix}\bar{\epsilon}&-\bar{\Psi}\\0&-1\end{pmatrix}\begin{pmatrix}\bar{\epsilon}^{-1}&-\bar{\epsilon}^{-1}\bar{\Psi}\\0&1\end{pmatrix}\\
&=\begin{pmatrix}1&-2\bar{\Psi}\\0&-1\end{pmatrix}\\
\begin{pmatrix}\bar{\epsilon}&\bar{\Psi}\\0&1\end{pmatrix}\begin{pmatrix}0&1\\0&0\end{pmatrix}\begin{pmatrix}\bar{\epsilon}&\bar{\Psi}\\0&1\end{pmatrix}^{-1}
&=\begin{pmatrix}0&\bar{\epsilon}\\0&0\end{pmatrix}\begin{pmatrix}\bar{\epsilon}^{-1}&-\bar{\epsilon}^{-1}\bar{\Psi}\\0&1\end{pmatrix}\\
&=\begin{pmatrix}0&\bar{\epsilon}\\0&0\end{pmatrix}\\
\begin{pmatrix}\bar{\epsilon}&\bar{\Psi}\\0&1\end{pmatrix}\begin{pmatrix}0&0\\1&0\end{pmatrix}\begin{pmatrix}\bar{\epsilon}&\bar{\Psi}\\0&1\end{pmatrix}^{-1}
&=\begin{pmatrix}\bar{\Psi}&0\\1&0\end{pmatrix}\begin{pmatrix}\bar{\epsilon}^{-1}&-\bar{\epsilon}^{-1}\bar{\Psi}\\0&1\end{pmatrix}\\
&=\begin{pmatrix}\bar{\Psi}\bar{\epsilon}^{-1}&-\bar{\epsilon}^{-1}\bar{\Psi}^2\\\bar{\epsilon}^{-1}&-\bar{\epsilon}^{-1}\bar{\Psi}\end{pmatrix}
\end{align*}
となる。
従って$A_5$への$G_{\Q_{17}}$作用は、$1$次元ずつのfiltration $A^0_5\subset A^1_5\subset A_5$があって、
それぞれのgrに$\epsilon, 1, \epsilon^{-1}$で作用する。
\begin{prop}
$h_{5}\leq\dfrac{1}{25}$
\end{prop}
\begin{proof}
$\phi_1\colon H^1(\Q_5,A_5)\to H^1(\Q_5,A_5/A_5^0), \phi_2\colon H^1(\Q_5,A_5/A_5^0)\to H^1(I_5,A_5/A_5^0)$とし、
その合成を$\phi$とする。
$\abs{Im\phi}\geq25$を示したい。
$\abs{Im\phi}\geq\abs{Im\phi_1}/\abs{\ker\phi_2}$である。
$\abs{Im\phi_1}=125$であることは上の補題と同様にして長完全列を書いて示せる。
$\abs{\ker\phi_2}=5$はinflation-restriction系列をかく。
\end{proof}

$p=\infty$の場合。
\begin{lem}
$h_\infty=\abs{H^0(\R,A_5^*)}=\abs{(A_5^*)^{G_\R}}=25$
\end{lem}
\begin{proof}
これは$\bar{\rho}_0$がoddであることから複素共役の固有値が$-1, 1, 1$である。
従って固定部分が$\F_5$上$2$次元。
\end{proof}
\begin{prop}
$\abs{S_\emptyset(\Q,A_5)}=1$
\end{prop}
\begin{proof}
セルマー群の双対性から(Tate-Poitou完全列?)。
\[
\frac{\abs{S_\emptyset(\Q,A_5)}}{\abs{S_\emptyset(\Q,A_5^*)}}=h_\infty h_5h_{17}\leq1
\]
\end{proof}

話の流れ。
$\Sigma=\emptyset$の時には判定法の条件を満たすので、$R_\Sigma=T_\Sigma$が証明できる。
一般の$\Sigma$はこれに帰着する。
\section{Galois cohomology}
etale cohomologyとの関係、とくにTate-Poitou dualityとPoincare dualityの関係。

有限体のBrauer群。
$G=\hat{\Z}$の群コホモロジーの計算。
local fieldsでは$H^q(G,A)=\colim H^q(G/nG,A^{nG})$で定義する。
(これは$M\mapsto M^G$のderived functorではない?)
$A$が有限もしくは可除の場合に$H^2=0$を示す。
(実際には$2$以上全て消える?)

(副有限)群の(連続)コホモロジーをサイトの非可換コホモロジーとして解釈したい。
群コホモロジーと分類空間のコホモロジー、etale cohomologyとGalois cohomologyの対応

\subsection{巡回群の有限係数コホモロジー}
位数$n$の巡回群はlens空間を分類空間にもつ。
$\Z$のresoutionとして

\subsection{表現の変形とSelmer群}
$\bar{\rho}\colon G\to GL_d(\F)$とその変形$\rho\colon G\to GL_d(O/\lambda^n)$に対し、
$Ext^1_{O/\lambda^n[G]}(ad\rho,ad\rho)$と$H^1(G,ad\rho)$は同型となる。
これの部分群$Ext^1_{D\cap O/\lambda^n[G]}(ad\rho,ad\rho)$に対応する部分群として
$H^1_D(G,ad\rho)\subset H^1(G,ad\rho)$を定義し、さらに$H^1_D(G,ad\rho)=H^1_D(G,ad\rho)\cap H^1(G,ad^0\rho)$を定義する。
この時
\begin{prob}
$Def^\chi_{\bar{\rho}}(\F[\epsilon]/\epsilon^2)$と$H^1_D(G,ad^0\rho)$と$Hom_{\F}(m_{R_D}/(\lambda,m^2_{R_D}),\F)$と
$Hom_{O-alg}(R_D,\F[\epsilon/\epsilon^2)$は同型。
\end{prob}

一般論としてdeformationと$H^1$の関係をSGAや小平に沿って理解する。
\end{document}